\documentclass[11pt,a4paper,twoside,onecolumn]{article}
\usepackage[left=20mm,right=20mm,top=25mm,bottom=20mm]{geometry}

%------------------------------------------------------------------------------------------------
% Additional packages
%------------------------------------------------------------------------------------------------

\usepackage[colorlinks=true]{hyperref}
\usepackage[hyperref,dvipsnames]{xcolor}

\usepackage[T1]{fontenc} 							
\usepackage[english]{babel} 					
\usepackage{amssymb,amsmath,amsthm,amsbsy}		
\usepackage{array} 						
\usepackage{pdfpages}								
\usepackage{graphicx} 						
\usepackage{caption}[2004/07/16]					
\usepackage{subfigure}							
\usepackage{mathrsfs} 	
\usepackage{empheq}								
\usepackage{nicefrac}								
\usepackage{stmaryrd}								
\usepackage{fixltx2e} 							
\usepackage{fix-cm}								
\usepackage{bm}
\usepackage{booktabs}
\usepackage{here}								
\usepackage{url}
\usepackage{enumitem}
\usepackage{thmbox}
\usepackage{shadethm}		
\usepackage{siunitx}
\usepackage{todonotes}	
\usepackage{bm}						

\newcommand{\revcommentA}[1]{\textbf{\textcolor{blue}{#1}}~}
\newcommand{\revcommentB}[1]{\textbf{\textcolor{blue}{#1}}~}
\newcommand{\revcommentC}[1]{\textit{\textcolor{blue}{#1}}~}
\newcommand{\reply}[1]{~\\[0.25cm] \textcolor{green!40!black}{#1}~}
\newcommand{\todiscuss}[1]{\textcolor{red!60!orange}{\underline{To discuss:} #1}~}
\newcommand{\tocheck}{\textcolor{red!60!black}{\underline{ToCheck}}~}
\newcommand{\fixed}{\textcolor{green!50!black}{\underline{fixed}}~}

\renewcommand{\baselinestretch}{1,3}
\setlength{\parindent}{0pt}

%--------------------------------------------------ENVIRONMENTS---------------------------------------
\newtheorem[style=S,underline=true,bodystyle=\normalfont\noindent]{thmDef}{Definition}[section]
\newtheorem[style=S,cut=false]{thmCor}[thmDef]{Corollary}
\newtheorem[style=S,cut=false,headstyle=\normalsize\bfseries\boldmath####1~####2]{thmLem}[thmDef]{Lemma}
\newtheorem[style=S]{thmCor}[thmDef]{Corollary}
\newshadetheorem{thm}[thmDef]{Theorem}
\newenvironment{thmThe}[1][]{%
 	\definecolor{shadethmcolor}{gray}{0.9}%
  	\definecolor{shaderulecolor}{rgb}{0.0,0.0,0.0}%
  	\setlength{\shadeboxrule}{0.0pt}%
  	\begin{thm}[#1]%
}{\end{thm}}			

\newenvironment{thmProof}
                [0]
                { \begin{example}[Proof] \normalsize}
                { $\hfill\blacksquare$ \end{example} }	
	
\newenvironment{thmRem}
                [0]
                { \begin{example}[Remark] \normalsize}
                { $\hfill\blacktriangle$ \end{example} }							
%-----------------------------------------------------------------------------------------------------

\makeatletter % Mit \vhrulefill{x.xpt} lassen sich horizontale Striche mit variabler Dicke machen
  \def\vhrulefill#1{\leavevmode\leaders\hrule\@height#1\hfill \kern\z@}
\makeatother	


\usepackage{fancyhdr}
\fancyhf{}
\cfoot{\thepage}
\renewcommand{\headrulewidth}{0pt}
\pagestyle{fancy}

\usepackage{csquotes}

\begin{document}

%\pagestyle{empty}
\setlength{\footskip}{20pt}

%------------------------------------------------------------------------------------------------
%------------------------------------------------------------------------------------------------
\section*{Response to Reviewers \\
Communications on Applied Mathematics and Computation
}	
%------------------------------------------------------------------------------------------------
%------------------------------------------------------------------------------------------------
\vspace{-5mm}
\vhrulefill{1pt}


\begin{flushright}
  Erik Burman, Janosch Preuss, Tim van Beeck\\[0.2cm]
  Department of Mathematics, \\
  University College London, United Kingdom \\[0.2cm]
  Inria Project-Team Makutu, \\
  Université de Pau et des Pays de l’Adour, France \\[0.2cm]
  Institute for Numerical and Applied Mathematics, \\
  Georg-August-Universit\"{a}t G{\"o}ttingen, Germany \\
\end{flushright}

\vspace{5mm}

Professor Chi-Wang Shu \\
Editor-in-Chief \\
Communications on Applied Mathematics and Computation

\vspace{5mm}
\hfill \today
\vspace{5mm}

Dear Professor Shu, \\

First of all, thank you for considering our article "Variational data assimilation for the wave equation in heterogeneous media: Numerical investigation of stability" for publication in \textit{Communications on Applied Mathematics and Computation}. We also thank the reviewers for their valuable comments and suggestions for improvement. Below, we address their points one after another. \\



%------------------------------------------------------------------------------------------------
%------------------------------------------------------------------------------------------------


%------------------------------------------------------------------------------------------------
%------------------------------------------------------------------------------------------------
\subsection*{Referee 1}	
%------------------------------------------------------------------------------------------------
%------------------------------------------------------------------------------------

\begin{enumerate}
  \item The Geometric Control Condition (GCC) was originally introduced in [8]. However, the
  manuscript does not clarify how GCC satisfaction is verified in the specific numerical
  simulations. For completeness, I recommend including a precise definition of the GCC.
  \item On page 7, could the authors clarify whether $\bm{Z}_h$ belongs to the space $\mathcal{W}_{h, \Delta t}^{k^{\ast},q^{\ast}}$?
  \item In the numerical experiments, what values are chosen for the parameters $k, q, k^{\ast}$ and
  $q^{\ast}$? The current description appears to only specify the value of $k$.
  \item How is the “best approximation” determined in the numerical experiments? Please
  explain the criteria or method used to obtain it.
  \item On page 18, the black line in Figure 5, does it represent the exact solution at time 0.5
  or at time 1? It seems to depict the solution at $T = 1.0$, where the approximation
  matches the exact solution well. The exact solution at $T = 0.5$ does not appear to be
  shown.
  \item In the two-dimensional case, only a square data domain is considered. A more inter-
  esting scenario would involve a curved interface, such as a circular interface. Could the
  authors provide related numerical results for such a case?
\end{enumerate}


\subsection*{Referee {2}}	
%------------------------------------------------------------------------------------------------
%------------------------------------------------------------------------------------------------
\begin{enumerate}
  \item How did you numerically verify Assumption 1? Specifically, how were the two
  inequalities in Assumption 1 verified?
  \item What is the relationship between the GCC and condition (1.3)? In the numerical
  tests in Section 4.1, you use condition (1.3) to imply the GCC. It appears that
  you have verified stability under a much stronger condition. I am wondering
  what would happen if you directly let the data satisfy the GCC, as you did in
  Section 4.2. Moreover, could you please explain why $T = 0.75$ was chosen in
  Section 4.2? Would a smaller $T$ work?
  \item Could you please explain why you adopted the discontinuous-in-time FEM
  method? What are the advantages of this method for the problem under consideration?
  \item Please clarify the choices of $q$, $k^{\ast}$ , and $q^{\ast}$ in the numerical experiments.
  \item Please clarify the conditions of Lemma 3.1. Does it require the GCC or condition
  (1.3)?
  \item The authors are encouraged to provide a detailed proof of Lemma 3.3.
  \item What would happen if $c_1$ becomes larger?
  \item Please clarify the meaning of “the $L^2$ -best approximation errors” in the numerical section.
  \item Typo: last line on page 19, “$L^2$ -bestapproximation” $\to$ “$L^2$ -best approximation”.
\end{enumerate}



% \vspace{1cm}
Yours sincerely,\\
Erik Burman, Janosch Preuss, and Tim van Beeck

\end{document}
