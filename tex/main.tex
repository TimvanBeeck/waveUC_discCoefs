%3 May 2012
%\documentclass[10pt,reqno]{amsart}
\documentclass[sn-mathphys-num]{sn-jnl}

%\setlength{\topmargin}{0cm}
%\setlength{\textheight}{21cm}
%\setlength{\oddsidemargin}{0in}
%\setlength{\evensidemargin}{0in}
%\setlength{\textwidth}{6.5in}
%\setlength{\parindent}{.25in}

%\pagestyle{plain}
\usepackage{stmaryrd}
\usepackage{amssymb, amsmath, amsthm}
\usepackage{mathtools}
\usepackage{xcolor}
\usepackage{hyperref}
%\usepackage{showkeys}
\usepackage{enumerate}
\usepackage{subfig}
\usepackage{enumitem} 
\usepackage{todonotes}
\usepackage{booktabs}
\usepackage[]{algorithm2e}
\usepackage[capitalize]{cleveref}
\usepackage{placeins}
\crefname{equation}{}{}
\usepackage{verbatim}
\usepackage{bbm}

%\usepackage{graphicx}
%% expansion of width
%\textwidth=15.7cm
%\textheight=22.5cm
%\parskip=3pt
%\parindent=8mm
%\oddsidemargin=2mm
%\evensidemargin=0mm
%\topmargin=-0.5cm
%\marginparwidth=1cm


%% definition of theorem-type environments
\newtheorem{thm}{Theorem}[section]
\newtheorem{lem}[thm]{Lemma}
\newtheorem{cor}[thm]{Corollary}
\newtheorem{prop}[thm]{Proposition}
\newtheorem{defn}[thm]{Definition}
\newtheorem{rmk}{Remark}
\newtheorem{exa}{Example}
\newtheorem{assum}{Assumption}
\numberwithin{equation}{section}
\newcommand{\bel}{\begin{equation} \label}
\newcommand{\ee}{\end{equation}}
\def\beq{\begin{equation}}
\def\eeq{\end{equation}}
\newcommand{\jump}[1]{\llbracket#1\rrbracket}
\newcommand{\bea}{\begin{eqnarray}}
\newcommand{\eea}{\end{eqnarray}}
\newcommand{\beas}{\begin{eqnarray*}}
\newcommand{\eeas}{\end{eqnarray*}}
\newcommand{\pd}{\partial}
\newcommand{\mdiv}[1]{\ensuremath{\mathrm{div} \left( #1 \right)}}
\newcommand{\dd}{\mbox{d}}

\newcommand{\ep}{\varepsilon}
\newcommand{\la}{\lambda}
\newcommand{\va}{\varphi}
\newcommand{\ppp}{\partial}
\newcommand{\chch}{\chi_{\eta}}
\newcommand{\walpha}{\widetilde{\alpha}}
\newcommand{\wbeta}{\widetilde{\beta}}

\newcommand{\re}{\mathfrak R}

\newcommand{\im}{\mathfrak I}
\newcommand{\pdif}[2]{\frac{\partial #1}{\partial #2}}
\newcommand{\ppdif}[2]{\frac{\partial^2 #1}{{\partial #2}^2}}
\newcommand{\R}{\mathbb{R}}
\newcommand{\C}{\mathbb{C}} 
\newcommand{\N}{\mathbb{N}} 
\newcommand{\ooo}{\overline}
\newcommand{\uu}{\mathbf{u}}
\renewcommand{\v}{\mathbf{v}}
\newcommand{\y}{\mathbf{y}}
\newcommand{\RR}{\mathbf{R}}
\newcommand{\Y}{\mathbf{Y}}
\newcommand{\w}{\mathbf{w}}
\newcommand{\z}{\mathbf{z}}
\newcommand{\G}{\mathbf{G}}
\newcommand{\cB}{\mathcal{B}}
\newcommand{\cD}{\mathcal{D}}
\newcommand{\cL}{\mathcal{L}}
\newcommand{\cO}{\mathcal{O}}
\newcommand{\Hin}{\mathcal{H}_{\mathrm{in},T_0}}
\newcommand{\Gi}{S_{\mathrm{in}}}
\newcommand{\Go}{S_{\mathrm{out}}}
\newcommand{\dom}{\mathrm{Dom}}

\newcommand{\sdd}{\mathcal{E}}


\renewcommand{\baselinestretch}{1.5}
%
\renewcommand{\div}{\mathrm{div}\,}  %div
\newcommand{\grad}{\mathrm{grad}\,}  %grad
\newcommand{\rot}{\mathrm{rot}\,}  %rot

\newcommand{\supp}{\mathrm{supp}\,}  %supp
%\newcommand{\span}{\mathrm{span}\,} %span

\allowdisplaybreaks
%%  item
\renewcommand{\theenumi}{\arabic{enumi}}
\renewcommand{\labelenumi}{(\theenumi)}
\renewcommand{\theenumii}{\alph{enumii}}
\renewcommand{\labelenumii}{(\theenumii)}
\def\epsilon{\varepsilon}
%\def\phi {\varphi}
\def \la {{\lambda}}
\def \a {{\alpha}}
\def\t{\theta}
\def\fh{\frac{1}{h}}


\DeclareMathOperator{\dis}{dist}


\newcommand{\wop}{\square_c}

\newcommand{\tnorm}[1]{\vert\hspace{-0.3mm}\Vert#1\Vert\hspace{-0.3mm}\vert}

\providecommand{\abs}[1]{\left\lvert#1\right\rvert}
% pour les normes
\providecommand{\norm}[1]{\left\lVert#1\right\rVert}

\renewcommand{\leq}{\leqslant}
\renewcommand{\geq}{\geqslant}
\providecommand{\abs}[1]{\left\lvert#1\right\rvert}
% pour les normes
\providecommand{\norm}[1]{\left\lVert#1\right\rVert}
\def\thefootnote{{}}



\newcommand{\HOX}[1]{\marginpar{\footnotesize #1}}

\newcommand{\gammaGLS}{\gamma_{\text{GLS}}}
\newcommand{\gammaCIP}{\gamma}

\newcommand{\dT}{\mathrm{d}t}
\newcommand{\dX}{\mathrm{d}x}
\newcommand{\dS}{\mathrm{d}S}

\newcommand{\STdom}{Q}

\newcommand{\STdata}{\omega_T}
\newcommand{\STdataDisc}{\underline{\omega}_T}
\newcommand{\SemiDiscSpace}{\mathcal{W}_h}
\newcommand{\FullyDiscSpace}{W}

%\newcommand{\FullyDiscrSpaceDisc}[2]{ W_{ {#1},{#2}}^{ \text{dc} } }
%\newcommand{\FullyDiscrSpaceCont}[2]{ W_{ {#1},{#2}}^{ \text{c}  } }
%\newcommand{\ProdFullyDiscrSpaceDisc}[2]{  \mathcal{W}_{ {#1},{#2} }^{ \text{dc} } }
%\newcommand{\ProdFullyDiscrSpaceCont}[1]{  \mathcal{W}_{ {#1} }^{ \text{c} } }

\newcommand{\SemiDiscrSpace}[1]{ W^{ {#1}}_{h} }
\newcommand{\ProdSemiDiscrSpace}[1]{ \mathcal{W}^{ {#1} }_{h} }
\newcommand{\FullyDiscrSpace}[2]{ W^{ {#1},{#2}}_{h, \Delta t  } }
\newcommand{\FullyDiscrSpaceHat}[2]{ \hat{W}^{ {#1},{#2}}_{h, \Delta t  } }
\newcommand{\ProdFullyDiscrSpace}[2]{ \mathcal{W}^{ {#1},{#2}}_{h, \Delta t  } }

\DeclarePairedDelimiterX{\inp}[2]{(}{)}{#1, #2}
\newcommand{\tangular}[1]{ \llbracket\kern-0.5ex|#1|\kern-0.5ex\rrbracket} 
%\newcommand{\jump}[1]{\llbracket#1\rrbracket}
\newcommand{\avg}[1]{ \{\!\!\{#1\}\!\!\}}

\newboolean{includeextras}
\ifdefined\withextras
\setboolean{includeextras}{true}
\else
\setboolean{includeextras}{false}
\fi

\newcommand{\putextra}[1]{\ifthenelse{\boolean{includeextras}}{#1}{}}

\newcommand{\Uh}{\underline{\mathbf{U}}_h}
\newcommand{\Vh}{\underline{\mathbf{V}}_h}
\newcommand{\Yh}{\underline{\mathbf{Y}}_h}
\newcommand{\Zh}{\underline{\mathbf{Z}}_h}
\newcommand{\Wh}{\underline{\mathbf{W}}_h}

\newcommand{\ul}{\underline{u}}
\newcommand{\yl}{\underline{y}}
\newcommand{\zl}{\underline{z}}
\newcommand{\wl}{\underline{w}}

\newcommand{\Sud}{S^{\uparrow \downarrow}_{\Delta t}}


\newcommand{\dt}{\partial_t}
\newcommand{\dtt}{\partial_t^2}

%%% TvB: Added for plots, some might be redundant ##

%------------------------- Added for Plots ----
% Tikz package
\usepackage{tikz}
\usepackage{pgfplots,pgfplotstable}
\usepgfplotslibrary{colorbrewer,groupplots}
\usepackage{caption}
\pgfplotsset{compat=1.18}


%find colors at https://colorbrewer2.org/#type=qualitative&scheme=Set1&n=5
\pgfplotsset{
% initialize Set1-5:
cycle list/Set1-5,
% combine it with ’mark list*’:
cycle multiindex* list={
mark list*\nextlist
Set1-5\nextlist
},
}

\pgfplotsset{
    discard if not/.style 2 args={
        x filter/.append code={
            \edef\tempa{\thisrow{#1}}
            \edef\tempb{#2}
            \ifx\tempa\tempb
            \else
                \def\pgfmathresult{inf}
            \fi
        }
    }
}

%%%


\date{\today}
\begin{document}

\title{A numerical study of variational data assimilation for the wave equation in heterogeneous media}

\author[1]{\fnm{Erik} \sur{Burman}}\email{e.burman@ucl.ac.uk}
\author[1]{\fnm{Janosch} \sur{Preuss}}\email{j.preuss@ucl.ac.uk}
\author[2]{\fnm{Tim} \sur{van Beeck}}\email{t.beeck@math.uni-goettingen.de}

\affil[1]{\orgdiv{Department of Mathematics}, \orgname{University College London}, \orgaddress{\street{Gower Street}, \city{London}, \postcode{WC1E 6BT}, \country{United Kingdom}}}

\affil[2]{\orgdiv{Institute for Numerical and Applied Mathematics}, \orgname{University of G\"{o}ttingen}, \orgaddress{\street{Lotzenstr.}, \city{G\"{o}ttingen}, \postcode{37083}, \country{Germany}}}

\abstract{Todo...}

\keywords{Data assimilation, wave equation, discontinuous coefficients}

\maketitle

\section{Introduction}
\noindent In this article, we consider the discretization of a data assimilation problem for the wave equation in heterogeneous media. In the following, let $\Omega \subset \mathbb{R}^d$ be a bounded domain with smooth boundary $\partial \Omega$ such that there exists subdomains $\Omega_1, \Omega_2 \subseteq \Omega$ such that $\Omega = \overline{\Omega}_1 \cup \overline{\Omega}_2$ and $\Omega_1 \cap \Omega_2 = \emptyset$. We consider a scalar wave speed $c(x) = \mathbbm{1}_{\Omega_1} c_1(x) + \mathbbm{1}_{\Omega_2} c_2(x)$, $c_i \in C^\infty(\Omega_i)$, $i = 1,2$, which may be discontinuous across the interface $\Gamma = \Omega \setminus (\Omega_1 \cup \Omega_2)$, but constant in time. For a final time $T > 0$, we consider the space-time domain $Q = (0,T) \times \Omega$ with lateral boundary $\Sigma \coloneqq (0,T) \times \partial \Omega$. For any nonempty subset $\omega \subset \Omega$, we set $\omega_T \coloneqq (0,T) \times \omega$. In the following, let $\wop$ be the wave operator defined as 
\begin{equation*}
    \wop u := \partial_t^2 - \div(c^2 \nabla u).
\end{equation*}
In particular, we write $\square_1$ if $c_i = 1$, $i = 1,2$, i.e. if we are considering the case of homogeneous media. 
Then, we consider the problem: find $u : Q \rightarrow \mathbb{R}$ such that 
\begin{equation}\label{eq:waveEquation}
        (\wop u, u) = (0,0) \quad \text{ in } Q \times \Sigma.  
\end{equation}
To ensure that the problem is well posed, one usually prescribes initial data
\begin{equation}\label{eq:initialData}
    (u,\dt u) \vert_{t = 0} = (u_0,u_1) \text{ in } \Omega, \tag{IVP} 
\end{equation}
where $u_0 \in H^1(Q)$ and $u_1 \in L^2(Q)$ are given. For a detailed analysis of the well-posedness of the problem \eqref{eq:waveEquation}-\eqref{eq:initialData} in heterogeneous media we refer to \cite{StolkPhD}. 
Here we will consider a data assimilation problem instead, i.e. we assume that the initial data is unknown and that there are given measurements $u_{\omega} \in L^2(\omega_T)$ which are prescribed in the data domain $\omega_T$: 
\begin{equation}\label{eq:dataMatch}
    u = u_{\omega} \text{ in } \omega_T. \tag{DA}
\end{equation}


\noindent In a homogeneous medium, where $c_i = 1$, $i = 1,2$, the problem \eqref{eq:waveEquation}-\eqref{eq:dataMatch} is well-studied \cite{BFO20,BFMO21,DMS23}. In particular, it is known that if the boundary $\partial \Omega$ is strictly convex and the set $\omega_T \subset Q$ fulfills the geometric control condition (GCC) \cite{BLR88,BLR92} then the problem is Lipschitz stable. To be more precise, we have the following result.

\begin{thm}\label{thm:Lipschitz}
    Assume that $c_i = 1$, $i = 1,2$, and that $\partial \Omega$ is stricly convex. If $\omega_T \subset Q$ fulfills the GCC, then there exists a constant $C > 0$ such that for any $\phi \in H^1(Q)$ we have the following estimates
    \begin{align*}
        \Vert \phi \Vert_{L^\infty(0,T;L^2(\Omega))} + \Vert \dt \phi \Vert_{L^2(0,T;H^{-1}(\Omega))} &\le C \left(  \Vert \phi \Vert_{L^2(\omega_T)} + \Vert \phi \Vert_{L^2(\Sigma)} + \Vert \square_1 \phi \Vert_{H^{-1}(Q)} \right), \\
        \Vert \phi \vert_{t = 0} \Vert_{L^2(\Omega)} + \Vert \dt \phi \vert_{t = 0} \Vert_{H^{-1}(\Omega)} &\le C \left(\Vert \phi \Vert_{L^2(\omega_T)} + \Vert \phi \Vert_{L^2(\Sigma)} + \Vert \square_1 \phi \Vert_{H^{-1}(Q)} \right). 
    \end{align*}
\end{thm}

\begin{proof}
    See \cite[Thm. A.4]{BFMO21control} and \cite[Rem. A.5]{BFMO21control}.
\end{proof}
In that case, this estimate can be applied to $\phi = u-u_h$ to recover convergence rates in terms of the norm on the left-hand side. 

\noindent For the heterogeneous case where $c_1 \not =  c_2$, the problem is less studied. In the absence of the GCC, we have to assume that the final time $T$ is large enough to ensure that problem \eqref{eq:waveEquation}-\eqref{eq:dataMatch} is well-posed. To be precise, Holmgren's unique continuation theorem guarantees that \eqref{eq:waveEquation}-\eqref{eq:dataMatch} is uniquely solvable if
\begin{equation}\label{eq:Tcondition}
    T > 2 \sup_{x \in \Omega} \operatorname{dist}_{c}(x,\omega_T),
\end{equation}
where $\operatorname{dist}_c(x,\omega_T) := \inf_{y \in \omega_T} d_c(x,y)$ with $d_c$ being the distance function defined as the length of the shortest continuous path in $\Omega$ connecting $x$ and $y$ with respect to the metric adapted to the discontinuous wave speed $c$. While the condition \eqref{eq:Tcondition} guarantees unique solvability, the stability of problem \eqref{eq:waveEquation}-\eqref{eq:dataMatch} can be poor. In fact, it has been shown \cite[Thm. 5.17]{Filippas22} that the assumption \eqref{eq:Tcondition} only guarantees the following result:  
\begin{thm}[Thm. 5.17 of \cite{Filippas22}]\label{thm:filippas}
    Let $\omega_T \subset \Omega$ be nonempty and let $T$ be s.t. \eqref{eq:Tcondition} holds. Then, there exists $C,\kappa$ such that for any initial data $(u_0,u_1) \in H^1_0(\Omega) \times L^2(\Omega)$ and $u$ solving \eqref{eq:waveEquation}-\eqref{eq:initialData} one has for any $\mu > 0$ that 
    \begin{equation}\label{eq:FilippasEstimate}
        \Vert (u_0,u_1) \Vert_{L^2 \times H^{-1}} \le C e^{\kappa \mu} \left( \Vert u \Vert_{L^2(\omega_T)} + \Vert \wop u \Vert_{L^2(Q)} \right) + \frac{C}{\mu} \Vert (u_0,u_1) \Vert_{H^1 \times L^2}. 
    \end{equation}
    In particular, if $(u_0,u_1) \not = (0,0)$ we have that 
    \begin{equation}\label{eq:FilippasEstimateLog}
        \Vert (u_0,u_1) \Vert_{L^2 \times H^{-1}} \le C \frac{\Vert (u_0,u_1) \Vert_{H^1 \times L^2}}{\log \left( 1 + \frac{\Vert (u_0,u_1) \Vert_{H^1 \times L^2}}{ \Vert u \Vert_{L^2(\omega_T)} + \Vert \wop u \Vert_{L^2(Q)}} \right)}. 
    \end{equation} 
\end{thm} 



\noindent In this article, we introduce a discontinuous in time finite element method to solve the data assimilation problem \eqref{eq:waveEquation}-\eqref{eq:dataMatch} in the heterogeneous case. In particular, we want to investigate numerically if the following assumption is justified. 

\begin{assum}\label{assum:LipschitzStability}
    If $\omega_T \subset Q$ fulfills the GCC, then there exists a constant $C > 0$ such that for any $\phi \in H^1(Q)$ we have the following estimates
    \begin{align*}
        \Vert \phi \Vert_{L^\infty(0,T;L^2(\Omega))} + \Vert \dt \phi \Vert_{L^2(0,T;H^{-1}(\Omega))} &\le C \left(  \Vert \phi \Vert_{L^2(\omega_T)} + \Vert \phi \Vert_{L^2(\Sigma)} + \Vert \square_c \phi \Vert_{H^{-1}(Q)} \right), \\
        \Vert \phi \vert_{t = 0} \Vert_{L^2(\Omega)} + \Vert \dt \phi \vert_{t = 0} \Vert_{H^{-1}(\Omega)} &\le C \left(\Vert \phi \Vert_{L^2(\omega_T)} + \Vert \phi \Vert_{L^2(\Sigma)} + \Vert \square_c \phi \Vert_{H^{-1}(Q)} \right). 
    \end{align*}
\end{assum}


\section{Discretization (fully discrete \& dG in time)} 
\noindent In this section, we introduce a fully discrete discontinuous in time finite element method to discretize the data assimilation problem \eqref{eq:waveEquation}-\eqref{eq:dataMatch} for the wave equation in heterogeneous media. The discretization is based on the scheme introduced in \cite{BP24} for the homogeneous case. Before introducing the method in Section \ref{sec:method}, we introduce a geometric partition of the space-time domain $Q$ into time slabs in Section \ref{sec:spaceTimeDiscretization}. 

\subsection{Discretization of the space-time domain}\label{sec:spaceTimeDiscretization}
Let $\mathcal{T}_h$ be a quasi-uniform triangulation of $\Omega$ such that both, the data domain $\omega$ and the interface $\Gamma$ are fitted. 
Note that the triangulation can be constructed such that for $h$ small enough the following continuous trace inequality holds \cite[Sec. 4.2]{BFMO21control}: For all $v \in [H^1(K)]^d$ and $K \in \mathcal{T}_h$, we have
\begin{equation}\label{eq:traceInequality}
    \Vert v \Vert_{[L^2(\partial K)]^d} \le C \left(h^{-1/2} \Vert v \Vert_{[L^2(K)]^d} + h^{1/2} \Vert \nabla v \Vert_{[L^2(K)]^d} \right).  
\end{equation}



\noindent For $k \ge 1$, we define the $H^1$-conforming finite element space 
\begin{equation}
    V_h^k := \{ v \in H^1(\Omega) : v \vert_{K} \in \mathcal{P}^k(K) \ \forall K \in \mathcal{T}_h \},
\end{equation}
where $\mathcal{P}^k(K)$ denotes the space of polynomials of degree at most $k \in \mathbb{N}$ on $K \in \mathcal{T}_h$. Furthermore, we partition the time axis into $N$ subinterals $I_n = (t_n,t_{n+1})$, $n = 0, \dots, N-1$, where $0 = t_0 \le t_1 \le \dots \le t_N = T$. We assume that the intervals are of equal length and denote $\Delta t = \vert t_{n+1} -t_n \vert$. Then, we partition of $Q$ and $\Sigma$ into time-slabs 
\begin{equation}
    \begin{aligned}
        Q^n := &I_n \times \Omega, \quad \Sigma^n := I_n \times \Sigma, \quad n = 0, \dots, N-1, \\
        Q = &\bigcup_{n = 0}^{N-1} Q^n, \quad \phantom{:} \Sigma = \bigcup_{n = 0}^{N-1} \Sigma^n.
    \end{aligned}
\end{equation}
In the following, we denote the space-time integrals on the time slabs as 
\begin{equation*}
    (u,v)_{Q^n} := \int_{I_n} \int_{\Omega} uv \ \dX \dT, \quad (u,v)_{\Sigma^n} := \int_{I_n} \int_{\Sigma} uv \ \dS \dT,
\end{equation*}
and define $\Vert v \Vert^2_{Q^n} := (v,v)_{Q^n}$ and $\Vert v \Vert^2_{\Sigma^n} := (v,v)_{\Sigma^n}$. 
Furthermore, we set 
\begin{equation*}
    \omega^n := I_n \times \omega, \quad (u,v)_{\omega^n} := \int_{I_n} \int_{\omega} uv \ \dX \dT, \quad \Vert v \Vert^2_{\omega_T} := \sum_{n = 0}^{N-1} (v,v)_{\omega^n}.
\end{equation*}
Finally, we define the time jump operator 
\begin{equation*}
    v^n_{\pm} (x) := \lim_{s \rightarrow 0^+} v(x,t_n \pm s), \quad \jump{v^n} := v^n_+ - v^n_-. 
\end{equation*}

\subsection{A discontinuous in time FEM for the data assimilation problem}\label{sec:method}
\noindent We define the discontinuous in time finite element spaces 
\begin{equation}
    \FullyDiscrSpace{k}{q} := \otimes_{n = 0}^{N-1} \mathcal{P}^q(I_n) \otimes V_h^k, \quad \ProdFullyDiscrSpace{k}{q} :=  \FullyDiscrSpace{k}{q} \times \FullyDiscrSpace{k}{q}, \quad q \in \mathbb{N}_0, k \in \mathbb{N}. 
\end{equation}
For elements of $\ProdFullyDiscrSpace{k}{q}$, we use the notation $\Uh = (\ul_1,\ul_2) \in \ProdFullyDiscrSpace{k}{q}$. In the following, we denote
\begin{equation*}
    a(u,v)_{Q^n} := \int_{I_n} \int_{\Omega} c^2 \nabla u \cdot \nabla v \ \dX \dT,
\end{equation*}
and introduce a bilinear form $A$ that represents the wave-equation \eqref{eq:waveEquation} in mixed formulation:  
\begin{equation}
    \begin{aligned}
        A[\Uh,\Yh] := \sum_{n = 0}^{N -1} \Big\{ &(\dt \ul_2, \yl_1)_{Q^n} + a(\ul_1,\yl_1)_{Q^n} + (\dt \ul_1 - \ul_2,\yl_2)_{Q^n} \\
        &- (c^2 \nabla \ul_1 \cdot \mathbf{n}, \yl_1)_{\Sigma^n} \Big\},
    \end{aligned}
\end{equation} 
where $\Uh \in \ProdFullyDiscrSpace{k}{q}$ and $\Yh \in \ProdFullyDiscrSpace{k^\ast}{q^\ast}$ with $k, k^\ast, q \in \mathbb{N}$ and $q^\ast \in \mathbb{N}_0$. 
To approximate solutions of \eqref{eq:waveEquation}-\eqref{eq:dataMatch}, we search for stationary points of the Lagrangian 
\begin{align*}
    \cL_h (\Uh, \Zh) := &\frac{1}{2} \Vert \ul_1 - u_{\omega} \Vert^2_{\omega_T} + A[\Uh,\Zh] + \frac{1}{2} \gamma S_h(\Uh,\Uh) \\
    &- \frac{1}{2} \gamma^\ast S_h^\ast(\Zh,\Zh) + \frac{\gamma}{2} \Sud(\Uh,\Uh),
\end{align*}
where $S_h$, $S_h^\ast$, and $\Sud$ are stabilization terms yet to be defined. Note that the first and the second terms of $\cL_h$ incorporate the data and the PDE constraints, respectively. To define the stabilization terms $S_h$ and $S_h^\ast$, we introduce the following terms: 
\begin{align*}
    J(\Uh,\Wh) &:= \sum_{n = 0}^{N -1} \int_{I_n} \sum_{F \in \mathcal{F}_i} h (\jump{c^2 \nabla \ul_1}, \jump{c^2 \nabla \wl_1})_F \ \dT, \\
    G(\Uh,\Wh) &:= \sum_{n = 0}^{N -1} \int_{I_n} \sum_{K \in \mathcal{T}_h} h^2 (\dt \ul_2 - \div (c^2 \nabla \ul_1),\dt \wl_2 -\div( c^2 \nabla \wl_1))_K \ \dT, \\
    I_0(\Uh,\Wh) &:= \sum_{n = 0}^{N -1} (\ul_2 - \dt \ul_1, \wl_2-\dt \wl_1)_{Q_n}, \qquad 
    R(\Uh,\Wh) := \sum_{n = 0}^{N -1} h^{-1} (\ul_1,\wl_1)_{\Sigma^n}. 
\end{align*}
$J(\cdot,\cdot)$ is a continuous interior penalty term in space, $G(\cdot,\cdot)$ is a Galerkin least squares term enforcing the PDE locally on each element, $I_0(\cdot,\cdot)$ enforces that $\ul_2 = \dt \ul_1$, and $R(\cdot,\cdot)$ ensures boundary stability. 
Then we define the primal stabilizer $S_h$ as 
\begin{equation}
    S_h(\Uh,\Wh) := J(\Uh,\Wh) + I_0(\Uh,\Wh) + G(\Uh,\Wh) + R(\Uh,\Wh), 
\end{equation}
and the dual stabilizer $S_h^\ast$ through 
\begin{equation}
    S_h^\ast(\Yh,\Zh) := \sum_{n = 0}^{N-1} \left\{ (\yl_1,\zl_1)_{Q^n} + a(\yl_1,\zl_1)_{Q^n} + (\yl_2,\zl_2)_{Q^n} + h^{-1} (\yl_1,\zl_1)_{\Sigma^n} \right\}.
\end{equation}
The remaining stabilization term $\Sud$ imposes regularity on the discontinuities in time and is defined as 
\begin{equation}
    \Sud (\Uh,\Wh) := \underline{I}_1^{\uparrow \downarrow}(\Uh,\Wh) + \underline{I}_2^{\uparrow \downarrow}(\Uh,\Wh),
\end{equation}
where 
\begin{align*}
    \underline{I}_1^{\uparrow \downarrow}(\Uh,\Wh) &:= \sum_{n = 0}^{N-1} \left\{ \frac{1}{\Delta t} (\jump{\ul_1^n},\jump{\wl_1^n})_{\Omega} + \Delta t (c^2 \jump{\nabla \ul_1^n},c^2 \jump{\nabla \wl_1^n})_{\Omega}\right\}, \\
    \underline{I}_2^{\uparrow \downarrow}(\Uh,\Wh) &:= \sum_{n = 0}^{N-1} \frac{1}{\Delta t} (\jump{\ul_2^n},\jump{\wl_2^n})_{\Omega}.
\end{align*}

\noindent With the definition of $\cL_h$, the first order optimality conditions take the form: Find $(\Uh,\Zh) \in \ProdFullyDiscrSpace{k}{q} \times \ProdFullyDiscrSpace{k^\ast}{q^\ast}$ such that 
\begin{alignat*}{2}
    (\ul_1,\wl_1)_{\omega_T} \! + \! A[\Wh,\Zh] + \gamma S_h(\Uh,\Wh) + \Sud(\Uh,\Wh) &= (u_{\omega},\wl_1)_{\omega_T} \ &&\forall \Wh \in \ProdFullyDiscrSpace{k}{q} \\
    A[\Uh,\Yh] - \gamma^{\ast} S_h^\ast(\Yh,\Zh) &= 0 \ &&\forall \Yh \in \ProdFullyDiscrSpace{k^\ast}{q^\ast} 
\end{alignat*}
In a more compact form, we can write these conditions as: Find $(\Uh,\Zh) \in \ProdFullyDiscrSpace{k}{q} \times \ProdFullyDiscrSpace{k^\ast}{q^\ast}$ such that
\begin{equation}\label{eq:discreteProblem}
    B[(\Uh,\Zh),(\Wh,\Yh)] = (u_{\omega},\wl_1)_{\omega_T} \quad \forall (\Wh,\Yh) \in \ProdFullyDiscrSpace{k}{q} \times \ProdFullyDiscrSpace{k^\ast}{q^\ast},
\end{equation}
where 
\begin{equation}
    \begin{aligned}
        B[(\Uh,\Zh),(\Wh,\Yh)] \coloneqq \ &(\ul_1,\wl_1)_{\omega_T} + A[\Wh,\Zh]+ \gamma S_h(\Uh,\Wh) \\
        &+  \Sud(\Uh,\Wh) + A[\Uh,\Yh] - \gamma^{\ast} S_h^\ast(\Yh,\Zh).
    \end{aligned}
\end{equation}

\section{Error analysis}
\noindent In this section, we analyze the approximation of the continuous problem \eqref{eq:waveEquation}-\eqref{eq:dataMatch} through \eqref{eq:discreteProblem}. In particular, we show that the error between the first component $\ul_1$ of the discrete solution and the continuous solution $u$ is bounded by the best approximation error in suitable discrete norms defined below. Then, we show that under Assumption \ref{assum:LipschitzStability}, we can recover convergence rates in a physically relevant norm.
Let $\vert \cdot \vert_{S_h}, \vert \cdot \vert_{\uparrow \downarrow}$, and $\Vert \cdot \Vert_{S_h^{\ast}}$ be the semi-norms, respectively norms, induced by the stabilizers $S_h$, $\Sud$, and $S_h^{\ast}$:
\begin{equation}
    \vert \Uh \vert^2_{S_h} := S_h(\Uh,\Uh), \quad \vert \Uh \vert^2_{\uparrow \downarrow} := \Sud(\Uh,\Uh), \quad \Vert \Zh \Vert^2_{S_h^{\ast}} := S_h^{\ast}(\Zh,\Zh).
\end{equation}
\noindent Then, we define the discrete norm 
\begin{equation}
    \tnorm{ (\Uh,\Zh) }^2 := \gamma \vert \Uh \vert^2_{S_h} + \vert \Uh \vert^2_{\uparrow \downarrow} + \Vert \ul_1 \Vert^2_{\omega_T} + \gamma^{\ast} \Vert \Zh \Vert^2_{S_h^\ast},
\end{equation}
and its strengthened version
\begin{equation}\label{eq:tnormwop}
    \begin{aligned}
        \tnorm{(\Uh, \Zh)}^2_{\wop} := \tnorm{(\Uh, \Zh)}^2 &+ \sum_{n = 0}^{N -1} \Big\{ \Vert \dt u_2 \Vert^2_{Q^n} + \Vert c^2 \nabla u_1 \Vert^2_{Q^n} + \Vert \dt u_1 \Vert^2_{Q^n} \\
        &+ \Vert u_2 \Vert^2_{Q^n} + \int_{I_n} \sum_{K \in \mathcal{T}_h} h^2 \Vert c^2 u_1 \Vert^2_{H^2(K)} \dT \Big\}. 
    \end{aligned}
\end{equation}
%Note it suffices to show that $\tnorm{(\cdot,\cdot)}$ is a norm. We modify the steps from \cite{BP24} to show the following result. 

\begin{lem}
    The expressions $\tnorm{(\cdot,\cdot)}$ and $\tnorm{(\cdot,\cdot)}_{\wop}$ are norms on $\ProdFullyDiscrSpace{k}{q} \times \ProdFullyDiscrSpace{k^\ast}{q^\ast}$.  
\end{lem}

\begin{proof}
    It suffices to show that $\tnorm{(\cdot,\cdot)}$ is a norm. Since $\tnorm{(\cdot,\cdot)}$ is a semi-norm by definition, we only require that $\tnorm{(\Uh,\Zh)} = 0$ implies $\Uh=\Zh= 0$. Assume that $\tnorm{(\Uh,\Zh)} = 0$. Then, in particular, $\vert \Uh \vert_{\uparrow \downarrow} = 0$ and therefore $(\ul_1,\ul_2) \in [H^1(Q)]^2$. Furthermore, by definition of the stabilization terms $S_h$ and $S_h^{\ast}$, we have that $\Zh = 0$, $\ul_1 \vert_{\Sigma} = 0$, $\ul_1 \vert_{\omega_T} = 0$ and $\dt \ul_1 = \ul_2$. Similar to \cite[Lem. 2.1]{BP24}, partial integration yields 
    \begin{align*}
        \Vert \wop \ul_1 \Vert^2_{H^{-1}(Q)} &:= \sup_{\substack{  y \in H^1_0(Q), \\ \norm{y}_{H^1(Q) } = 1  }} \int_{Q} \left\{ -(\dt \ul_1) \dt y + c^2 \nabla \ul_1 \nabla y \right\}  \\
        &= \sup_{\substack{  y \in H^1_0(Q), \\ \norm{y}_{H^1(Q) } = 1  }} \Big\{ \sum_{n = 0}^{N-1} (\ul_2 - \dt \ul_1, \dt y)_{Q_n} \\
        &\qquad+ \sum_{n = 0}^{N-1} \int_{I_n} \sum_{K \in \mathcal{T}_h} (\dt \ul_2 - \div(c^2 \nabla \ul_1),y)_{K} \ \dT \\
        &\qquad + \sum_{n = 0}^{N-1} \int_{I_n} \sum_{F \in \mathcal{F}_i} (\jump{c^2 \nabla \ul_1}, y)_F \ \dT + \sum_{n = 0}^{N-1} (\jump{\ul_1^n},y)_{\Sigma^n} \Big\}.
    \end{align*}
    It follows by definition of the stabilizers and $\tnorm{(\Uh,\Zh)} = 0$ that $\Vert \wop \ul_1 \Vert_{H^{-1}(Q)} = 0$. Thus, $\ul_1$ solves \eqref{eq:waveEquation} and by taking $\mu \rightarrow \infty$ in Thm. \ref{thm:filippas}, we conclude that $u_0 = u_1 = 0$. As solutions to \eqref{eq:waveEquation}-\eqref{eq:initialData} depend continuously on the data \cite{StolkPhD}, it follows that $\Uh = 0$. 
\end{proof}

\noindent From this result it directly follows that the bilinear form $B$ enjoys inf-sup stability on $\ProdFullyDiscrSpace{k}{q} \times \ProdFullyDiscrSpace{k^\ast}{q^\ast}$ with respect to the $\tnorm{(\cdot,\cdot)}$-norm. 

\begin{cor}
    There exists a constant $C_B>0$ such that 
    \begin{equation}
        \sup_{(\Wh,\Yh) \in \ProdFullyDiscrSpace{k}{q} \times \ProdFullyDiscrSpace{k^\ast}{q^\ast}} \frac{B[(\Uh,\Zh),(\Wh,\Yh)]}{\tnorm{(\Wh,\Yh)}} \ge C_B \tnorm{(\Uh,\Zh)}.
    \end{equation}
\end{cor}

\begin{proof}
    Since $B[(\Uh,\Zh),(\Uh,-\Zh)] = \tnorm{(\Uh,\Zh)}^2 = \tnorm{(\Uh,\Zh)}\tnorm{(\Uh,-\Zh)}$, it directly follows that  
    \begin{align*}
        \sup_{(\Wh,\Yh) \in \ProdFullyDiscrSpace{k}{q} \times \ProdFullyDiscrSpace{k^\ast}{q^\ast}} &\frac{B[(\Uh,\Zh),(\Wh,\Yh)]}{\tnorm{(\Wh,\Yh)}} \\
        &\quad \ge \frac{B[(\Uh,\Zh),(\Uh,-\Zh)]}{\tnorm{(\Uh,-\Zh)}} = \tnorm{(\Uh,\Zh)}.
    \end{align*}
\end{proof}

 
\begin{lem}[Continuity of $A$]\label{lem:continuityA}
    For all $\mathbf{U} \in [H^1(Q_n) \cap L^2(0,T;H^2(\mathcal{T}_h))] \times H^1(Q_n)$ and all $\Yh \in \ProdFullyDiscrSpace{k^\ast}{q^\ast}$, we have that 
    \begin{equation*}
        A[\mathbf{U},\Yh] \le C \tnorm{(\mathbf{U}, 0)}_{\wop} \tnorm{(0,\Yh)}. 
    \end{equation*} 
\end{lem}

\begin{proof}
    First, note that $\tnorm{(0,\Yh)} = \Vert \Yh \Vert_{S_h^\ast}$. Using the Cauchy-Schwarz inequality, we obtain
    \begin{equation*}
        \sum_{n = 0}^{N-1} \left\{ (\dt u_2, \underline{y}_1)_{Q^n} + a(u_1,\underline{y}_1)_{Q^n} \right\} \le C \left( \sum_{n = 0}^{N-1} \Vert \dt u_2 \Vert^2_{Q^n} + \Vert c^2 \nabla u_1 \Vert^2_{Q^n} \right)^{1/2} \! \! \! \Vert \Yh \Vert_{S_h^{\ast}},
    \end{equation*}
    and 
    \begin{equation*}
        \sum_{n = 0}^{N-1} (\dt u_1 - u_2,\underline{y}_2)_{Q^n} \le C \left( \sum_{n = 0}^{N-1} \Vert \dt u_1 \Vert^2_{Q^n} + \Vert u_2 \Vert^2_{Q^n} \right)^{1/2} \Vert \Yh \Vert_{S_h^{\ast}}.
    \end{equation*}
    Since the interface $\Gamma$ is fitted by the triangulation, we have that $c^2 \nabla u_1 \in H^1(K)$ for each $K \in \mathcal{T}_h$. Thus, we can apply the trace inequality \eqref{eq:traceInequality} to obtain
    \begin{align*}
        \sum_{n = 0}^{N-1} (c^2 \nabla &u_1 \cdot n, \underline{y}_1)_{\Sigma^n} \\
        &\le \left( \sum_{n = 0}^{N-1} \int_{I_n} \sum_{K \in \mathcal{T}_h} h \Vert c^2 \nabla v \Vert_{\partial \Omega \cap \partial K}^2 \dT \right)^{1/2} \left( \sum_{n = 0}^{N-1} \int_{I_n} \sum_{K \in \mathcal{T}_h} h^{-1} \Vert \underline{y}_1 \Vert_{\partial \Omega \cap \partial K}^2 \dT \right)^{1/2} \\
        &\le C \left( \sum_{n = 0}^{N-1} \left\{ \Vert c^2 \nabla u_1 \Vert^2_{Q^n} + \int_{I_n} \sum_{K \in \mathcal{T}_h} h^2 \Vert c^2 u_1 \Vert_{H^2(K)} \dT \right\} \right)^{1/2} \Vert \Yh \Vert_{S_h^{\ast}}.
    \end{align*}
    Putting all estimates together, the claim follows. 
\end{proof}

\noindent Using the inf-sup stability of $B$ and the continuity of $A$, we can show that the approximation error in the $\tnorm{(\cdot,\cdot)}$-norm is bounded by the best approximation error in the $\tnorm{(\cdot,\cdot)}_{\wop}$-norm.

\begin{thm}\label{thm:bestapprox}
    Let $u$ be a sufficient regular solution of \eqref{eq:waveEquation} and $(\Uh,\Zh) \in \ProdFullyDiscrSpace{k}{q} \times \ProdFullyDiscrSpace{k^\ast}{q^\ast}$ be the solution to \eqref{eq:discreteProblem}. Set $\mathbf{U} := (u,\partial_t u)$. Then, there exists a constant $C>0$ such that
    \begin{equation}
        \tnorm{(\mathbf{U} - \Uh,\Zh)} \le \left( 1 + \frac{C}{C_B} \right) \inf_{\Vh \in \ProdFullyDiscrSpace{k}{q}} \tnorm{(\Uh - \Vh, 0)}_{\wop}.  
    \end{equation}
\end{thm}

\begin{proof}
    Let $\Vh \in \ProdFullyDiscrSpace{k}{q}$  and $(\Wh,\Yh) \in \ProdFullyDiscrSpace{k}{q} \times \ProdFullyDiscrSpace{k^\ast}{q^\ast}$ be arbitrary. The triangle inequality yields 
    \begin{equation*}
        \tnorm{(\mathbf{U} - \Uh,\Zh)} \le \tnorm{(\mathbf{U} - \Vh,0)} + \tnorm{(\Vh - \Uh,\Zh)}.
    \end{equation*}
    For the second term, we consider
    \begin{align*}
        B[&(\Uh - \Vh, \Zh),( \Wh, \Yh)] \\
        &= B[(\Uh,\Zh),( \Wh, \Yh)] - (\underline{v}_1, \underline{w}_1)_{\omega_T} - \gamma S_h(\Vh,\Wh) - \Sud(\Vh,\Wh) - A[\Vh,\Yh] \\
        &= (u_{\omega}, \underline{w}_1)_{\omega_T} - (\underline{v}_1, \underline{w}_1)_{\omega_T} - \gamma S_h(\Vh,\Wh) - \Sud(\Vh,\Wh) - A[\Vh,\Yh] \\
        &= (u - \underline{v}_1, \underline{w}_1)_{\omega_T}  + \gamma S_h(\mathbf{U} - \Vh,\Wh) + \Sud(\mathbf{U} - \Vh,\Wh) + A[\mathbf{U} - \Vh,\Yh],
    \end{align*} 
    where we use the fact that $\mathbf{U}$ is sufficiently smooth and $u = u_{\omega}$ on $\omega_T$. Then, we can apply Lem. \ref{lem:continuityA} to obtain 
    \begin{align*}
        (u - &\underline{v}_1, \underline{w}_1)_{\omega_T}  + \gamma S_h(\mathbf{U} - \Vh,\Wh) + \Sud(\mathbf{U} - \Vh,\Wh) + A[\mathbf{U} - \Vh,\Yh] \\
        &\le C(\tnorm{(\mathbf{U} - \Vh,0)}_{\wop} \tnorm{(0,\Yh)}  + \tnorm{(\mathbf{U} - \Vh,0)} \tnorm{(\Wh,0)} ) \\
        &\le C \tnorm{(\mathbf{U} - \Vh,0)}_{\wop} \tnorm{(\Wh,\Yh)}.
    \end{align*}
    Thus, the inf-sup condition on $B$ yields that 
    \begin{equation*}
        \tnorm{(\Vh - \Uh,\Zh)} \le \frac{C}{C_B} \tnorm{(\Uh - \Vh, 0)}_{\wop},
    \end{equation*}
    which gives the claim. 
\end{proof}


\begin{cor}\label{cor:tnormConvRates}
    Under the assumption of Thm. \ref{thm:bestapprox}, we have that
    \begin{equation*}
        \tnorm{(\mathbf{U} - \Uh,\Zh)} \le C h^s \Vert u \Vert_{H^m(Q)},
    \end{equation*}
    where $(s,m) := (\min\{ k,q \}, \max\{k,q\}+3)$. 
\end{cor}

\begin{proof}
    Due to the assumption that $c_i \in C^\infty(\Omega_i)$, $i = 1,2$, we have that 
    \begin{equation*}
        \tnorm{(\Uh - \Vh, 0)}_{\wop} \le C \tnorm{(\Uh - \Vh, 0)}_{\square_1},
    \end{equation*}
    where $\tnorm{(\cdot, \cdot)}_{\square_1}$ is defined by \eqref{eq:tnormwop} with $c_i = 1$, $i = 1,2$. Thus, we can apply the interpolation results derived in \cite[Lem. 2.4]{BP24} to obtain the claim. 
\end{proof}


\noindent In the following, we want to show how the assumption of Lipschitz stability, cf. Assumption \ref{assum:LipschitzStability} or Thm. \ref{thm:Lipschitz} in the homogeneous case, can be used to show convergence in a physically relevant norm. To apply these results, we deal with the fact that the discrete solution is discontinuous in time by introducing a suitable lifting operator $L_{\Delta t} : \FullyDiscrSpace{k}{q} \rightarrow C^0(0,T;V_h^k)$ such that $L_{\Delta t} \ul_1 \in H^1(Q)$ as discussed in \cite[Sec. 4]{BP24}. 



\begin{thm}
    Let Assumption \ref{assum:LipschitzStability} be satisfied. For $u$ being a sufficiently regular solution of \eqref{eq:waveEquation} and $(\Uh,\Zh) \in \ProdFullyDiscrSpace{k}{q} \times \ProdFullyDiscrSpace{k^\ast}{q^\ast}$ being the solution to \eqref{eq:discreteProblem}, there exists a constant $C>0$ such that
    \begin{equation}
        \Vert u - L_{\Delta t} \ul_1 \Vert_{L^\infty(0,T;L^2(\Omega))} + \Vert \dt (u - L_{\Delta t} \ul_1) \Vert_{L^2(0,T;H^{-1}(\Omega))} \le C h^s \Vert u \Vert_{H^m(Q)}. 
    \end{equation}
    In particular, we have that 
    \begin{equation}
        \Vert u_0 - \ul_1 \vert_{t = 0} \Vert_{L^2(\Omega)} + \Vert u_1 - (\dt \ul_1) \vert_{t = 0} \Vert_{H^{-1}(\Omega)} \le C h^s \Vert u \Vert_{H^m(Q)}.
    \end{equation}
\end{thm}

\begin{proof}
    The proof follows from the same argumentation as the proof of \cite[Thm. 4.4]{BP24}.
\end{proof}


\begin{rmk}[Homogenous media]\label{rem:homMedia}
    In the homogenous case where $c_i = 1$, $i = 1,2$, we know that assumption \ref{assum:LipschitzStability} is satisfied, cf. Thm. \ref{thm:Lipschitz}. Thus, we recover the result from \cite[Thm. 4.4]{BP24}.
\end{rmk}

\begin{rmk}
    If assumption \ref{assum:LipschitzStability} is not satisfied, e.g. if the GCC is not fulfilled, we can still apply the results from Thm. \ref{thm:filippas} if $T > 2 \sup_{x \in \Omega} \text{dist}_c(x,\omega)$. \textbf{For this we require:} 
    \begin{itemize}
        \item control of $\Vert (u_0,u_1) \Vert_{H^1 \times L^2}$ (discrete), e.g. through Tikhonov regularization,
        \item being able to rewrite the estimates from Thm. \ref{thm:filippas} with $\Vert \wop u \Vert_{H^{-1}(Q)}$ instead of $\Vert \wop u \Vert_{L^2(Q)}$,
    \end{itemize}
    Then, we obtain 
    \begin{equation}
        \Vert (u - \ul_1) \vert_{t = 0} \Vert_{L^2(\Omega)}  + \Vert \dt (u - \ul_1) \vert_{t = 0} \Vert_{H^{-1}(\Omega)} \le C C_{0,1} \log(1 + C_{0,1} h^{-s} \Vert u \Vert_{H^m(Q)}^{-1})^{-1},
    \end{equation} 
    where $C_{0,1} := \Vert (u_0,u_1) \Vert_{H^1 \times L^2}$.
\end{rmk}


\section{Numerical experiments}
\noindent In this section, we present numerical experiments carried out with the proposed method. The examples are implemented using the \texttt{FEniCSx} library \cite{BarattaEtal2023,BasixJoss,ScroggsEtal2022,AlnaesEtal2014}. Reproduction material is available at \textbf{insert reproduction files...}

\subsection{Examples in one space dimension}

\subsubsection{Simple Example}\label{sec:numex:1D:simple}
\noindent We set $\Omega \coloneqq [0,1]$ and define the subdomains $\Omega_1 = (0,0.5)$ and $\Omega_2 = (0.5,1.0)$ such that $\Omega = \overline{\Omega}_1 \cup \overline{\Omega}_2$. Then, we fix the wavespeed $c_2 := c \vert_{\Omega_2} = 1$ and consider different values for $c_1 := c \vert_{\Omega_1}$. As a first example, we consider the following exact solution considered in \cite{MHI08}
\begin{equation}\label{eq:1D:exact:simple}
    u(x,t) := \begin{cases}
        \cos(w_1 c_1 t) \cos(w_1(x-0.5)), & x \in \Omega_1, \\
        \cos(w_2 c_2 t) \cos(w_2(x-0.5)), & x \in \Omega_2,
    \end{cases}
\end{equation}
where we set $w_1 = 3 \pi$ and $w_2 = w_1 c_1 / c_2$. Furthermore, we define
\begin{equation}
    \omega \coloneqq [0.0,0.25] \cup [0.75,1.0], \quad \text{and} \quad \Omega_R \coloneqq \Omega \setminus \omega = [0.25,0.75].
\end{equation}
In one space dimension, the domain $\omega_T$ fulfills the GCC if and only if $T > 2 \sup_{x \in \Omega} \operatorname{dist}_c(x,\omega_T)$, where $\operatorname{dist}_c(x,\omega_T)$ denotes the distance of $x$ to $\omega_T$ with respect to the rescaled metric $d_c$. Let $d(x,y) \coloneqq \vert x - y \vert$, $x, y \in \Omega$, be the Euclidean metric. Then, we have that
\begin{equation}
    d_c(x,y) \coloneqq \mathbbm{1}_{\Omega_1} c_1(x)^{-1} d(x,y) + \mathbbm{1}_{\Omega_2} c_2(x)^{-1} d(x,y), \quad x,y \in \Omega.
\end{equation} 
Thus, for $\omega$ as above, $c_2 = 1$ fixed and $c_1 > 1.0$, we require that $T > 0.25(1+c_1^{-1})$. 


and \textbf{compute until $T = 1.0$, correct...} to ensure that the GCC is fulfilled. Furthermore, we set $h = 1/2^{1+L}$ and $\Delta t = h/2$ for different refinement levels $L \in \{1,2,3,4\}$. The results of the experiments can be found in Fig. \ref{fig:jumpCoefs:contrast:2.5} and Fig. \ref{fig:jumpCoefs:quality}. ... \\



 
\begin{figure}[!htbp]
    \begin{center}
        \begin{tikzpicture}[scale=0.72]
            \begin{groupplot}[%
                group style={%
                group size=2 by 2,
                horizontal sep=1.25cm,
                vertical sep=2cm,
                },
            ymajorgrids=true,
            grid style=dashed,
            %ymin = 1e-3, ymax = 0.5e1,
            ]    
            \nextgroupplot[width=8cm,height=6.2cm,domain=1:4,xmode=linear,ymode=log, ylabel={}, xlabel={$L$}, title={$c_1 = 2.5$, $T=0.5$, $\Vert (u - u_h) \Vert_{L^\infty(0,T;L^2(\Omega_R))}$}, %cycle list name=, 
            legend pos=south west, %yticklabels={,,},
            ]
           
            %L-infty-L2-u (k = 2)
            %\addplot+[discard if not={order}{2}, discard if not={contrast}{2.5},line width=1.1pt,color=orange,mark=square*] table [x=L, y=L-infty-L2-error-u, col sep=comma] {../data/newexact_1D_jumpingCoefs_k3_WPFalse.csv};
            %\addplot+[discard if not={order}{2}, discard if not={contrast}{2.5},line width=1.1pt,dashed,color=orange,mark=square*] table [x=L, y=bestapprox-L-infty-L2-error-u, col sep=comma] {../data/newexact_1D_jumpingCoefs_k3_WPFalse.csv};
            
            \addplot+[discard if not={order}{2}, discard if not={contrast}{2.5},line width=1.1pt,color=orange,mark=square*] table [x=L, y=L-infty-L2-error-u, col sep=comma] {../data/simpleExact_T0.5_1D_jumpingCoefs_k3.csv};
            \addplot+[discard if not={order}{2}, discard if not={contrast}{2.5},line width=1.1pt,dashed,color=orange,mark=square*] table [x=L, y=bestapprox-L-infty-L2-error-u, col sep=comma] {../data/simpleExact_T0.5_1D_jumpingCoefs_k3.csv};

           
            %L-infty-L2-u (k = 3)
            %\addplot+[discard if not={order}{3}, discard if not={contrast}{2.5},line width=1.1pt,color=cyan!60!black,mark=diamond*] table [x=L, y=L-infty-L2-error-u, col sep=comma] {../data/newexact_1D_jumpingCoefs_k3_WPFalse.csv};
            %\addplot+[discard if not={order}{3}, discard if not={contrast}{2.5},line width=1.1pt,dashed,color=cyan!60!black,mark=diamond*] table [x=L, y=bestapprox-L-infty-L2-error-u, col sep=comma] {../data/newexact_1D_jumpingCoefs_k3_WPFalse.csv};

            \addplot+[discard if not={order}{3}, discard if not={contrast}{2.5},line width=1.1pt,color=cyan!60!black,mark=diamond*] table [x=L, y=L-infty-L2-error-u, col sep=comma] {../data/simpleExact_T0.5_1D_jumpingCoefs_k3.csv};
            \addplot+[discard if not={order}{3}, discard if not={contrast}{2.5},line width=1.1pt,dashed,color=cyan!60!black,mark=diamond*] table [x=L, y=bestapprox-L-infty-L2-error-u, col sep=comma] {../data/simpleExact_T0.5_1D_jumpingCoefs_k3.csv};
        
            %\addplot[gray, dashed, domain=1:4] {15*(1/2^(2))^(x-0.9)};
            \addplot[gray, dashed, domain=1:4] {0.3*(1/2^(2))^(x-0.9)};
            %\addplot[gray, dashed, domain=1:4] {7.5*(1/2^(3))^(x-0.9)};
            \addplot[gray, dashed, domain=1:4] {0.03*(1/2^(3))^(x-0.9)};
            \node [draw=none] at (axis description cs:0.80,0.55) {\color{gray}\footnotesize $\!\!\mathcal{O}(h^{2})$};
            \node [draw=none] at (axis description cs:0.45,0.35) {\color{gray}\footnotesize $\!\!\mathcal{O}(h^{3})$};
            \legend{$k=2 $,, $k = 3$,}

            \nextgroupplot[width=8cm,height=6.2cm,domain=1:4,xmode=linear,ymode=log, ylabel={}, xlabel={$L$}, title={$c_1 = 2.5$, $T = 0.5$, $\Vert \partial_t (u - u_h) \Vert_{L^2(0,T;L^2(\Omega_R))}$}, %cycle list name=, 
            legend pos=south west, %yticklabels={,,},
            ]
            %L2-L2-u_t (k = 2)
            \addplot+[discard if not={order}{2}, discard if not={contrast}{2.5},line width=1.1pt,color=orange,mark=square*] table [x=L, y=L2-L2-error-u_t, col sep=comma] {../data/simpleExact_T0.5_1D_jumpingCoefs_k3.csv};
            \addplot+[discard if not={order}{2}, discard if not={contrast}{2.5},line width=1.1pt,dashed,color=orange,mark=square*] table [x=L, y=bestapprox-L2-L2-error-u_t, col sep=comma] {../data/simpleExact_T0.5_1D_jumpingCoefs_k3.csv};
            

            %L2-L2-u_t (k = 3)
            \addplot+[discard if not={order}{3}, discard if not={contrast}{2.5},line width=1.1pt,color=cyan!60!black,mark=diamond*] table [x=L, y=L2-L2-error-u_t, col sep=comma] {../data/simpleExact_T0.5_1D_jumpingCoefs_k3.csv};
            \addplot+[discard if not={order}{3}, discard if not={contrast}{2.5},line width=1.1pt,dashed,color=cyan!60!black,mark=diamond*] table [x=L, y=bestapprox-L2-L2-error-u_t, col sep=comma] {../data/simpleExact_T0.5_1D_jumpingCoefs_k3.csv};
            
            \addplot[gray, dashed, domain=1:4] {4.5*(1/2^(2))^(x-0.9)};
            %\addplot[gray, dashed, domain=1:4] {1.0*(1/2^(2))^(x-0.9)};
            \addplot[gray, dashed, domain=1:4] {1.0*(1/2^(3))^(x-0.9)};
            %\addplot[gray, dashed, domain=1:4] {0.5*(1/2^(3))^(x-0.9)};
            \node [draw=none] at (axis description cs:0.30,0.87) {\color{gray}\footnotesize $\!\!\mathcal{O}(h^{2})$};
            \node [draw=none] at (axis description cs:0.65,0.2) {\color{gray}\footnotesize $\!\!\mathcal{O}(h^{3})$};
            \legend{$k=2 $,, $k = 3$,}
            \nextgroupplot[width=8cm,height=6.2cm,domain=1:4,xmode=linear,ymode=log, ylabel={}, xlabel={$L$}, title={$c_1 = 2.5$, $T = 0.1$, $\Vert (u - u_h) \Vert_{L^\infty(0,T;L^2(\Omega))}$}, %cycle list name=, 
            legend pos=south west, %yticklabels={,,},
            ]
           
            %L-infty-L2-u (k = 2)
            \addplot+[discard if not={order}{2}, discard if not={contrast}{2.5},line width=1.1pt,color=orange,mark=square*] table [x=L, y=L-infty-L2-error-u, col sep=comma] {../data/simpleExact_T0.1_1D_jumpingCoefs_k3.csv};
            \addplot+[discard if not={order}{2}, discard if not={contrast}{2.5},line width=1.1pt,dashed,color=orange,mark=square*] table [x=L, y=bestapprox-L-infty-L2-error-u, col sep=comma] {../data/simpleExact_T0.1_1D_jumpingCoefs_k3.csv};

           
            %L-infty-L2-u (k = 3)
            \addplot+[discard if not={order}{3}, discard if not={contrast}{2.5},line width=1.1pt,color=cyan!60!black,mark=diamond*] table [x=L, y=L-infty-L2-error-u, col sep=comma] {../data/simpleExact_T0.1_1D_jumpingCoefs_k3.csv};
            \addplot+[discard if not={order}{3}, discard if not={contrast}{2.5},line width=1.1pt,dashed,color=cyan!60!black,mark=diamond*] table [x=L, y=bestapprox-L-infty-L2-error-u, col sep=comma] {../data/simpleExact_T0.1_1D_jumpingCoefs_k3.csv};
        
            %\addplot[gray, dashed, domain=1:4] {15*(1/2^(2))^(x-0.9)};
            \addplot[gray, dashed, domain=1:4] {0.005*(1/2^(2))^(x-0.9)};
            %\addplot[gray, dashed, domain=1:4] {7.5*(1/2^(3))^(x-0.9)};
            \addplot[gray, dashed, domain=1:4] {0.001*(1/2^(3))^(x-0.9)};
            \addplot[gray,dashed] table {
                1 0.009999968541169805
                2 0.006666645694113204
                3 0.004999984270584902
                4 0.003999987416467922

            };
            \node [draw=none] at (axis description cs:0.75,0.45) {\color{gray}\footnotesize $\!\!\mathcal{O}(h^{2})$};
            \node [draw=none] at (axis description cs:0.45,0.35) {\color{gray}\footnotesize $\!\!\mathcal{O}(h^{3})$};
            \node [draw=none] at (axis description cs:0.7,0.75) {\color{gray}\footnotesize $\!\!\mathcal{O}(\vert \log(h) \vert^{-1})$};
            \legend{$k=2 $,, $k = 3$,}

            \nextgroupplot[width=8cm,height=6.2cm,domain=1:4,xmode=linear,ymode=log, ylabel={}, xlabel={$L$}, title={$c_1 = 2.5$, $T = 0.1$, $\Vert \partial_t (u - u_h) \Vert_{L^2(0,T;L^2(\Omega))}$}, %cycle list name=, 
            legend pos=south west, %yticklabels={,,},
            ]
            %L2-L2-u_t (k = 2)
            \addplot+[discard if not={order}{2}, discard if not={contrast}{2.5},line width=1.1pt,color=orange,mark=square*] table [x=L, y=L2-L2-error-u_t, col sep=comma] {../data/simpleExact_T0.1_1D_jumpingCoefs_k3.csv};
            \addplot+[discard if not={order}{2}, discard if not={contrast}{2.5},line width=1.1pt,dashed,color=orange,mark=square*] table [x=L, y=bestapprox-L2-L2-error-u_t, col sep=comma] {../data/simpleExact_T0.1_1D_jumpingCoefs_k3.csv};
            

            %L2-L2-u_t (k = 3)
            \addplot+[discard if not={order}{3}, discard if not={contrast}{2.5},line width=1.1pt,color=cyan!60!black,mark=diamond*] table [x=L, y=L2-L2-error-u_t, col sep=comma] {../data/simpleExact_T0.1_1D_jumpingCoefs_k3.csv};
            \addplot+[discard if not={order}{3}, discard if not={contrast}{2.5},line width=1.1pt,dashed,color=cyan!60!black,mark=diamond*] table [x=L, y=bestapprox-L2-L2-error-u_t, col sep=comma] {../data/simpleExact_T0.1_1D_jumpingCoefs_k3.csv};
            
            \addplot[gray, dashed, domain=1:4] {0.1*(1/2^(2))^(x-0.9)};
            %\addplot[gray, dashed, domain=1:4] {1.0*(1/2^(2))^(x-0.9)};
            \addplot[gray, dashed, domain=1:4] {0.02*(1/2^(3))^(x-0.9)};
            %\addplot[gray, dashed, domain=1:4] {0.5*(1/2^(3))^(x-0.9)};
            \addplot[gray,dashed] table {
                1 0.09999968541169805
                2 0.06666645694113204
                3 0.04999984270584902
                4 0.03999987416467922

            };
            \node [draw=none] at (axis description cs:0.80,0.40) {\color{gray}\footnotesize $\!\!\mathcal{O}(h^{2})$};
            \node [draw=none] at (axis description cs:0.65,0.2) {\color{gray}\footnotesize $\!\!\mathcal{O}(h^{3})$};
            \node [draw=none] at (axis description cs:0.7,0.75) {\color{gray}\footnotesize $\!\!\mathcal{O}(\vert \log(h) \vert^{-1})$};
            \legend{$k=2 $,, $k = 3$,}
            \end{groupplot}
        \end{tikzpicture}
    \end{center}
    \caption{We consider the approximation of \eqref{eq:1D:exact:simple} with polynomial degree $k \in \{2,3\}$ for contrast $c_1 = 2.5$ (left \& middle). Both, the $L^\infty(0,T;L^2(\Omega_R))$-error of $u-u_h$ (left) and the $L^2(0,T;L^2(\Omega_R))$-error of $\dt (u-u_h)$ (middle) converge with order $\mathcal{O}(h^k)$ and are close to the respective error of the $L^2$-bestapproximation (dashed lines). For increasing contrast $c_1$, both errors increase with the same order of magnitude as the respective best approximation error (right). \textbf{left is with $T = 0.5$, the increasing contrast is still with $T = 1.0$ and not on $\Omega_R$..., also adjust the time step...}, rewrite..., plot $\mathcal{O}(\log(h))$ ... }
    \label{fig:jumpCoefs:contrast:2.5}
  \end{figure}


  \begin{figure}[!htbp]
    \begin{center}
        \begin{tikzpicture}[scale=0.85]
            \begin{groupplot}[%
                group style={%
                group size=2 by 1,
                horizontal sep=1.0cm,
                vertical sep=2cm,
                },
            ymajorgrids=true,
            grid style=dashed,
            %ymin = 1e-3, ymax = 0.5e1,
            ]    
            \nextgroupplot[width=8cm,height=6.2cm,domain=1:5,xmode=linear,ymode=log, ylabel={}, xlabel={$c_1$}, title={$\Vert \partial_t (u - u_h) \Vert_{L^2(0,T;L^2(\Omega_R))}$}, %cycle list name=, 
            legend style={legend columns=4, draw=none,nodes={scale=.8}},legend to name=named, %yticklabels={,,},
            %ymin=1e-7,ymax=1e1,
            ]
            %\addplot+[line width=1.1pt,color=teal,mark=*] table [x=contrast, y=L2-L2-error-u_t, col sep=comma]{../data/simpleExact_1D_jumpingCoefs_k3.csv};
            \addplot+[line width=1.1pt,color=teal,mark=*] table [x=contrast, y=L2-L2-error-u_t, col sep=comma]{../data/simpleExact_T0.5_multipleContrasts_1D_jumpingCoefs_k3.csv};
            \addplot+[line width=1.1pt,color=teal,mark=*,dashed] table [x=contrast, y=bestapprox-L2-L2-error-u_t, col sep=comma]{../data/simpleExact_T0.5_multipleContrasts_1D_jumpingCoefs_k3.csv};
            \addplot+[line width=1.1pt,color=orange,mark=diamond*] table [x=contrast, y=L2-L2-error-u_t, col sep=comma]{../data/simpleExact_TAdapted_multipleContrasts_1D_jumpingCoefs_k3.csv};
            \addplot+[line width=1.1pt,color=orange,mark=diamond*,dashed] table [x=contrast, y=bestapprox-L2-L2-error-u_t, col sep=comma]{../data/simpleExact_TAdapted_multipleContrasts_1D_jumpingCoefs_k3.csv};


            \addplot[gray, dashed, domain=1:4.5] {0.001*(1/2^(-2))^(x-0.9)};
            \addplot[gray, dashed, domain=1:4.5] {0.001*(1/2^(-3))^(x-0.9)};

            \node [draw=none] at (axis description cs:0.70,0.37) {\color{gray}\footnotesize $\!\!\mathcal{O}(c_1^{2})$};
            \node [draw=none] at (axis description cs:0.50,0.75) {\color{gray}\footnotesize $\!\!\mathcal{O}(c_1^{3})$};

            \legend{$T = 0.5$, $L^2$-best approx. ($T = 0.5$), $T$ variable, $L^2$-best approx. ($T$ variable)}
            \nextgroupplot[width=8cm,height=6.2cm,domain=1:5,xmode=linear,ymode=log, ylabel={}, xlabel={$c_1$}, title={$\Vert u - u_h \Vert_{L^\infty(0,T;L^2(\Omega_R))}$}, %cycle list name=, 
            legend pos=south east, %yticklabels={,,},
            %ymin=1e-7,ymax=1e1,
            ]
            \addplot+[line width=1.1pt,color=teal,mark=*] table [x=contrast, y=L-infty-L2-error-u, col sep=comma]{../data/simpleExact_T0.5_multipleContrasts_1D_jumpingCoefs_k3.csv};
            \addplot+[line width=1.1pt,color=teal,mark=*,dashed] table [x=contrast, y=bestapprox-L-infty-L2-error-u, col sep=comma]{../data/simpleExact_T0.5_multipleContrasts_1D_jumpingCoefs_k3.csv};

        
            \addplot+[line width=1.1pt,color=orange,mark=diamond*] table [x=contrast, y=L-infty-L2-error-u, col sep=comma]{../data/simpleExact_TAdapted_multipleContrasts_1D_jumpingCoefs_k3.csv};
            \addplot+[line width=1.1pt,color=orange,mark=diamond*,dashed] table [x=contrast, y=bestapprox-L-infty-L2-error-u, col sep=comma]{../data/simpleExact_TAdapted_multipleContrasts_1D_jumpingCoefs_k3.csv};

            
            \addplot[gray, dashed, domain=1:4.5] {0.00001*(1/2^(-2))^(x-0.9)};
            \addplot[gray, dashed, domain=1:4.5] {0.0003*(1/2^(-3))^(x-0.9)};
            \node [draw=none] at (axis description cs:0.70,0.27) {\color{gray}\footnotesize $\!\!\mathcal{O}(c_1^{2})$};
            \node [draw=none] at (axis description cs:0.50,0.75) {\color{gray}\footnotesize $\!\!\mathcal{O}(c_1^{3})$};

            \end{groupplot}
        \end{tikzpicture}
        \pgfplotslegendfromname{named}
    \end{center}
    \caption{For $k = 3$ and $L = 3$, we consider the errors $\Vert \dt (u-u_h) \Vert_{L^2(0,T;L^2(\Omega_R))}$ (left) and $\Vert u - u_h \Vert_{L^\infty(0,T;L^2(\Omega_R))}$ (right) for increasing contrast $c_1 \in \{1.0,1.5,...,4.5\}$. We compare the case where $T = 0.5$ is fixed for all choices of $c_1$ and the case where $T$ decreases with the constrast $c_1$ while keeping the GCC fulfilled.}
    \label{fig:jumpCoefs:increasingcontrast}
  \end{figure}




  \begin{figure}[!htbp]
    \begin{center}
        \begin{tikzpicture}[scale=0.72]
            \begin{groupplot}[%
                group style={%
                group size=2 by 1,
                horizontal sep=1.5cm,
                vertical sep=2cm,
                },
            ymajorgrids=true,
            grid style=dashed,
            %ymin = 1e-3, ymax = 0.5e1,
            ]    
                \nextgroupplot[width=9cm,height=7cm,domain=0:1,xmode=linear,ymode=linear, xlabel={$x$}, ylabel={}, title={$u$ vs. $u_h$ ($c_1 = 2.5$)}, %cycle list name=paulcolors, 
                legend style={legend columns=5, draw=none,nodes={scale=.8}},legend to name=named, %yticklabels={,,},
                ]
                %\draw[fill=gray!40,draw=none,opacity=0.5] (0.25,-3) -- (0.25,3) -- (0,3) -- (0,-3) -- cycle;
                %\draw[fill=gray!40,draw=none,opacity=0.5] (0.75,-3) -- (0.75,3) -- (1.0,3) -- (1.0,-3) -- cycle;
                %\addplot+[discard if not={contrast}{1.0},line width=2.5pt,mark=None,color=black] table [x=x, y=y, col sep=comma] {../../wave_data_assimilation/data/jumpCoefs/exact_plot_data.csv};
                \addplot+[discard if not={contrast}{2.5},line width=3.0pt,mark=None,color=black] table [x=x, y=y, col sep=comma] {../data/exact_plot_data_1D_simple.csv};
                \addplot+[discard if not={L}{1},line width=1.1pt,mark=None] table [x=x, y=y, col sep=comma] {../data/newexact_approx_plot_data_WPFalse_contrast2.5_k3.csv};
                \addplot+[discard if not={L}{2},line width=1.1pt,mark=None] table [x=x, y=y, col sep=comma] {../data/newexact_approx_plot_data_WPFalse_contrast2.5_k3.csv};
                \addplot+[discard if not={L}{3},line width=1.1pt,mark=None] table [x=x, y=y, col sep=comma] {../data/newexact_approx_plot_data_WPFalse_contrast2.5_k3.csv};
                \addplot+[discard if not={L}{4},line width=1.1pt,mark=None] table [x=x, y=y, col sep=comma] {../data/newexact_approx_plot_data_WPFalse_contrast2.5_k3.csv};
                
                

                \draw[dashed,gray,very thick] (0.5,-2) -- (0.5,2);
                \node [draw=none] at (axis description cs:0.45,0.23) {\color{gray} $\Gamma$};

                \legend{exact,$L=1$, $L=2$, $L=3$, $L=4$}
                \nextgroupplot[width=9cm,height=7cm,domain=0:1,xmode=linear,ymode=linear, xlabel={$x$}, ylabel={}, title={$u$ vs. $u_h$ ($c_1 = 5.5$)}, %cycle list name=paulcolors, 
                legend pos=north west, %yticklabels={,,},
                ]
                %\draw[fill=gray!40,draw=none,opacity=0.5] (0.25,-3) -- (0.25,3) -- (0,3) -- (0,-3) -- cycle;
                %\draw[fill=gray!40,draw=none,opacity=0.5] (0.75,-3) -- (0.75,3) -- (1.0,3) -- (1.0,-3) -- cycle;
                %\addplot+[discard if not={contrast}{1.0},line width=2.5pt,mark=None,color=black] table [x=x, y=y, col sep=comma] {../../wave_data_assimilation/data/jumpCoefs/exact_plot_data.csv};
                \addplot+[discard if not={contrast}{5.5},line width=3.0pt,mark=None,color=black] table [x=x, y=y, col sep=comma] {../data/exact_plot_data_1D_simple.csv};
                \addplot+[discard if not={L}{1},line width=1.1pt,mark=None] table [x=x, y=y, col sep=comma] {../data/newexact_approx_plot_data_WPFalse_contrast5.5_k3.csv};
                \addplot+[discard if not={L}{2},line width=1.1pt,mark=None] table [x=x, y=y, col sep=comma] {../data/newexact_approx_plot_data_WPFalse_contrast5.5_k3.csv};
                \addplot+[discard if not={L}{3},line width=1.1pt,mark=None] table [x=x, y=y, col sep=comma] {../data/newexact_approx_plot_data_WPFalse_contrast5.5_k3.csv};
                \addplot+[discard if not={L}{4},line width=1.1pt,mark=None] table [x=x, y=y, col sep=comma] {../data/newexact_approx_plot_data_WPFalse_contrast5.5_k3.csv};
            

                \draw[dashed,gray,very thick] (0.5,-3) -- (0.5,3);
                \node [draw=none] at (axis description cs:0.45,0.23) {\color{gray} $\Gamma$};

                %Colors
                \definecolor{L1Color}{RGB}{55,126,184}
                \definecolor{L2Color}{RGB}{77,175,74}
                \definecolor{L3Color}{RGB}{152,78,163}
                \definecolor{L4Color}{RGB}{255,127,0}

                %L = 1
                \draw[color=L1Color,thick] (0.75,-2) -- (1,-2);
                \draw[color=L1Color,thick] (0.75,-2.05) -- (0.75,-1.95);
                \draw[color=L1Color,thick] (1.0,-2.05) -- (1.0,-1.95);

                %L = 2
                \draw[color=L2Color,thick] (0.75,-1.75) -- (1,-1.75);
                \draw[color=L2Color,thick] (0.75,-1.70) -- (0.75,-1.8);
                \draw[color=L2Color,thick] (0.875,-1.70) -- (0.875,-1.8);
                \draw[color=L2Color,thick] (1.0,-1.70) -- (1.0,-1.8);
                
                %L = 3
                \draw[color=L3Color,thick] (0.75,-1.5) -- (1,-1.5);
                \draw[color=L3Color,thick] (0.75,-1.45) -- (0.75,-1.55);
                \draw[color=L3Color,thick] (0.8125,-1.45) -- (0.8125,-1.55);
                \draw[color=L3Color,thick] (0.875,-1.45) -- (0.875,-1.55);
                \draw[color=L3Color,thick] (0.9375,-1.45) -- (0.9375,-1.55);
                \draw[color=L3Color,thick] (1.0,-1.45) -- (1.0,-1.55);
            
                %L = 4
                \draw[color=L4Color,thick] (0.75,-1.25) -- (1,-1.25);
                \draw[color=L4Color,thick] (0.75,-1.20) -- (0.75,-1.30);
                \draw[color=L4Color,thick] (0.78125,-1.20) -- (0.78125,-1.30);
                \draw[color=L4Color,thick] (0.8125,-1.20) -- (0.8125,-1.30);
                \draw[color=L4Color,thick] (0.84375,-1.20) -- (0.84375,-1.30);
                \draw[color=L4Color,thick] (0.875,-1.20) -- (0.875,-1.30);
                \draw[color=L4Color,thick] (0.90625,-1.20) -- (0.90625,-1.30);
                \draw[color=L4Color,thick] (0.9375,-1.20) -- (0.9375,-1.30);
                \draw[color=L4Color,thick] (0.96875,-1.20) -- (0.96875,-1.30);
                \draw[color=L4Color,thick] (1.0,-1.20) -- (1.0,-1.30);

                %\legend{exact, $L=1$, $L=2$, $L=3$, $L=4$}
                
            \end{groupplot}
        \end{tikzpicture}
        \pgfplotslegendfromname{named}
    \end{center}
    \caption{Exact solution and approximations with $k = 3$ on different refinement levels for $c_1 = 2.5$ (left) and $c_1= 5.5$ (right).}
    \label{fig:jumpCoefs:quality}
  \end{figure}


  \begin{comment}
  \begin{figure}[!htbp]
    \begin{center}
        \begin{tikzpicture}[scale=0.72]
            \begin{groupplot}[%
                group style={%
                group size=3 by 1,
                horizontal sep=1.5cm,
                vertical sep=2cm,
                },
            ymajorgrids=true,
            grid style=dashed,
            %ymin = 1e-3, ymax = 0.5e1,
            ]    
                \nextgroupplot[width=9cm,height=7cm,domain=0:4,xmode=linear,ymode=log, xlabel={}, ylabel={}, title={$\Vert u_0 - u_h \vert_{t = 0} \Vert_{H^1(\Omega)}$}, %cycle list name=paulcolors, 
                legend pos=south east, %yticklabels={,,},
                ]
                %\addplot+[discard if not={contrast}{1.0},line width=2.5pt,mark=None,color=black] table [x=x, y=y, col sep=comma] {../../wave_data_assimilation/data/jumpCoefs/exact_plot_data.csv};
                \addplot+[discard if not={order}{2},discard if not={L}{4},line width=1.1pt] table [x=contrast, y=H1-u1-0, col sep=comma] {../../wave_data_assimilation/data/jumpCoefs/initialH1Test_1D_jumpingCoefs_k2_WPFalse_contrast3.5.csv};
                \addplot+[discard if not={order}{2},discard if not={L}{4},line width=1.5pt] table [x=contrast, y=H1-dt-at-t-0, col sep=comma] {../../wave_data_assimilation/data/jumpCoefs/initialH1TestGCC_1D_jumpingCoefs_k2_WPFalse_contrast3.5.csv};
                \addplot+[discard if not={order}{3},discard if not={L}{4},line width=1.1pt] table [x=contrast, y=H1-u1-0, col sep=comma] {../../wave_data_assimilation/data/jumpCoefs/initialH1Test_1D_jumpingCoefs_k3_WPFalse_contrast3.5.csv};
                \addplot+[line width=1.1pt,dashed] table [x=contrast, y expr=\thisrow{exp_Lambda}*0.00000001, col sep=comma] {../../wave_data_assimilation/data/jumpCoefs/1DsimpleSol_Filippas.csv};

                \addplot[gray, dashed, domain=1:4] {0.03*(1/2^(-2))^(x-0.9)};
                \addplot[gray, dashed, domain=1:4] {0.003*(1/2^(-2))^(x-0.9)};
                
    
                \legend{$k=2 \quad L = 4$, $k = 3 \quad L = 4$, $e^\Lambda$ (rescaled)}
                
                \nextgroupplot[width=9cm,height=7cm,domain=0:4,xmode=linear,ymode=log, xlabel={}, ylabel={}, title={$\Vert u_0 - u_h \vert_{t = 0} \Vert_{H^1(\Omega)}$}, %cycle list name=paulcolors, 
                legend pos=south east, %yticklabels={,,},
                ]
                %\addplot+[discard if not={contrast}{1.0},line width=2.5pt,mark=None,color=black] table [x=x, y=y, col sep=comma] {../../wave_data_assimilation/data/jumpCoefs/exact_plot_data.csv};
                \addplot+[discard if not={order}{2},discard if not={L}{4},line width=1.1pt] table [x=contrast, y=H1-u1-0, col sep=comma] {../../wave_data_assimilation/data/jumpCoefs/initialH1TestNoGCC_1D_jumpingCoefs_k2_WPFalse_contrast3.5.csv};
                \addplot+[discard if not={order}{2},discard if not={L}{4},line width=1.5pt] table [x=contrast, y=H1-dt-at-t-0, col sep=comma] {../../wave_data_assimilation/data/jumpCoefs/initialH1TestnoGCC_1D_jumpingCoefs_k2_WPFalse_contrast3.5.csv};
                %\addplot+[discard if not={order}{3},discard if not={L}{4},line width=1.1pt] table [x=contrast, y=H1-u1-0, col sep=comma] {../../wave_data_assimilation/data/jumpCoefs/initialH1TestNoGCC_1D_jumpingCoefs_k3_WPFalse_contrast3.5.csv};
                \addplot+[line width=1.1pt,dashed] table [x=contrast, y expr=\thisrow{exp_Lambda}*0.0001, col sep=comma] {../../wave_data_assimilation/data/jumpCoefs/1DsimpleSol_Filippas.csv};

                \addplot[gray, dashed, domain=1:4] {0.03*(1/2^(-2))^(x-0.9)};
                \addplot[gray, dashed, domain=1:4] {0.003*(1/2^(-2))^(x-0.9)};
                
    
                \legend{$k=2 \quad L = 4$, $k = 3 \quad L = 4$, $e^\Lambda$ (rescaled)}
                \nextgroupplot[width=9cm,height=7cm,domain=0:4,xmode=linear,ymode=log, xlabel={}, ylabel={}, title={$\Vert u_0 - u_h \vert_{t = 0} \Vert_{H^1(\Omega)}$}, %cycle list name=paulcolors, 
                legend pos=south east, %yticklabels={,,},
                ]
                %\addplot+[discard if not={contrast}{1.0},line width=2.5pt,mark=None,color=black] table [x=x, y=y, col sep=comma] {../../wave_data_assimilation/data/jumpCoefs/exact_plot_data.csv};
                \addplot+[discard if not={order}{2},discard if not={L}{4},line width=1.1pt] table [x=contrast, y=H1-u1-0, col sep=comma] {../../wave_data_assimilation/data/jumpCoefs/initialH1TestRestricted_NoGCC_1D_jumpingCoefs_k2_WPFalse_contrast3.5.csv};
               
                \addplot+[line width=1.1pt,dashed] table [x=contrast, y expr=\thisrow{exp_Lambda}*0.00000001, col sep=comma] {../../wave_data_assimilation/data/jumpCoefs/1DsimpleSol_Filippas.csv};

                \addplot[gray, dashed, domain=1:4] {0.03*(1/2^(-2))^(x-0.9)};
                \addplot[gray, dashed, domain=1:4] {0.003*(1/2^(-2))^(x-0.9)};
                
    
                \legend{$k=2 \quad L = 4$, $k = 3 \quad L = 4$, $e^\Lambda$ (rescaled)}
            \end{groupplot}
        \end{tikzpicture}
    \end{center}
    \caption{Filippas lambda...}
  \end{figure}
\end{comment}



\subsubsection{Multiple jumps}
\noindent The exact solution \eqref{eq:1D:exact:simple} can be extended to the case where we have multiple jumps in the wave speed $c$. For points $p_1,p_2 \in (0,1)$, we decompose the interval $\Omega = [0,1]$ into three subdomains $\Omega_1 = (0,p_1)$, $\Omega_2 = (p_1,p_2)$, and $\Omega_3 = (p_2,1)$ s.t. $\Omega = \cup_{i = 1,2,3} \overline{\Omega}_i$. Then, we consider the following Ansatz for the exact solution: 
\begin{equation}\label{eq:1D:exact:simpleMult}
    u(x,t) := \begin{cases}
        \cos(w_1 c_1 t) \cos(w_1(x-p_1)), & x \in \Omega_1, \\
        \cos(w_2 c_2 t) \cos(w_2(x-p_1)), & x \in \Omega_2, \\
        \cos(w_3 c_3 t) \cos(w_3(x-p_2)), & x \in \Omega_3. 
    \end{cases}
\end{equation}
To ensure that the solution is continuous, we choose $p_2$ in dependence of $p_1$ and $w_2$ in the following way. We know that 
\begin{equation*}
    \cos(w_2(x-p_1)) = 1 \Leftrightarrow w_2(x-p_1) = 2 \pi n, n \in \mathbb{Z} \Leftrightarrow x = \frac{2 \pi n + w_2 p_1}{w_2}, n \in \mathbb{Z}. 
\end{equation*}
Thus, if we choose $p_2 = \frac{2 \pi n + w_2 p_1}{w_2}$ for some $n \in \mathbb{Z}$, the function \eqref{eq:1D:exact:simpleMult} is indeed continuous.  
In the following, we fix $c_2 = 1.0$ and choose $w_1 = 3 \pi$, $w_2 = w_1 c_1/c_2$ as above. Furthermore, we set $c_3 = c_1$, $w_3 = w_1$. 
As a first example, we choose $p_1 = 0.4$ and $n = 3$ such that $p_2 \approx 0.667$ and consider the data domain $\omega = [0.0,0.3] \times [0.7,1.0]$. Figure \ref{fig:jumpCoefs:multipleJumps} shows the exact solution and the approximation with $k = 3$ and $c_1 = 7.5$.

\begin{figure}[!htbp]
    \begin{center}
        \begin{tikzpicture}[scale=0.72]
            \begin{groupplot}[%
                group style={%
                group size=2 by 1,
                horizontal sep=1.5cm,
                vertical sep=2cm,
                },
            ymajorgrids=true,
            grid style=dashed,
            ymin = -0.6e0, ymax = 0.6e0,
            ]    
                \nextgroupplot[width=9cm,height=7cm,domain=0:1,xmode=linear,ymode=linear, xlabel={}, ylabel={}, title={$u$ vs. $u_h$ ($c_1= 7.5$)}, %cycle list name=paulcolors, 
                legend style={legend columns=3, draw=none,nodes={scale=.8}},legend to name=named, %yticklabels={,,},
                ]
                \draw[fill=gray!40,draw=none,opacity=0.5] (0.3,-2) -- (0.3,2) -- (0,2) -- (0,-2) -- cycle;
                \draw[fill=gray!40,draw=none,opacity=0.5] (0.7,-2) -- (0.7,2) -- (1.0,2) -- (1.0,-2) -- cycle;
                %\addplot+[line width=1.5pt,mark=None,color=black] table [x=x, y=y, col sep=comma] {../../wave_data_assimilation/data/jumpCoefs/exact_plot_data_1D_multipleJumps.csv};
                \addplot+[line width=3.0pt,mark=None,color=black] table [x=x, y=y, col sep=comma] {../data/exact_plot_data_1D_multipleJumps_contrast7.5_T0.5.csv};
                %\addplot+[line width=1.0pt,mark=None,color=orange] table [x=x, y=y, col sep=comma] {../data/multipleJumps_approx_plot_data_WPFalse_contrast7.5_k3_uniform.csv};
                \addplot+[line width=1.0pt,mark=None,color=blue!75!white] table [x=x, y=y, col sep=comma] {../data/multipleJumps_approx_plot_data_WPFalse_contrast7.5_k3_uniform_T0.5_L4.csv};
                \addplot+[line width=1.0pt,mark=None,color=orange] table [x=x, y=y, col sep=comma] {../data/multipleJumps_approx_plot_data_WPFalse_contrast7.5_k3_uniform_T0.5_L3.csv};
                

                \draw[dashed,gray,very thick] (0.4,-2) -- (0.4,2);
                \draw[dashed,gray,very thick] (0.6666,-2) -- (0.6666,2);
                %\node [draw=none] at (axis description cs:0.45,0.23) {\color{gray} $\Gamma$};

                \node [draw=none] at (axis description cs:0.38,0.125) {\color{gray} $\Gamma_1$};
                \node [draw=none] at (axis description cs:0.7,0.125) {\color{gray} $\Gamma_2$};
                \node [draw=none] at (axis description cs:0.2,0.125) {\color{gray} $\omega$};
                \node [draw=none] at (axis description cs:0.8,0.125) {\color{gray} $\omega$};


                 %L = 3
                %\draw[color=orange,thick] (0.4,-0.5) -- (2/3,-0.5);
                %\draw[color=orange,thick] (0.4,-0.48) -- (0.4,-0.52);
                %\draw[color=orange,thick] (0.4375,-0.48) -- (0.4375,-0.52);
                %\draw[color=orange,thick] (0.5,-0.48) -- (0.5,-0.52);
                %\draw[color=orange,thick] (0.5625,-0.48) -- (0.5625,-0.52);
                %\draw[color=orange,thick] (0.625,-0.48) -- (0.625,-0.52);
                %\draw[color=orange,thick] (0.6666,-0.48) -- (0.6666,-0.52);
                
                \legend{exact, $L = 4$, $L = 3$}
                \nextgroupplot[width=9cm,height=7cm,domain=0:1,xmode=linear,ymode=linear, xlabel={}, ylabel={}, title={$u$ vs. $u_h$ ($c_1 = 11.5$)}, %cycle list name=paulcolors, 
                legend pos=north west, %yticklabels={,,},
                ]
                \draw[fill=gray!40,draw=none,opacity=0.5] (0.3,-2) -- (0.3,2) -- (0,2) -- (0,-2) -- cycle;
                \draw[fill=gray!40,draw=none,opacity=0.5] (0.7,-2) -- (0.7,2) -- (1.0,2) -- (1.0,-2) -- cycle;
                %\addplot+[line width=1.1pt,mark=None] table [x=x, y=y, col sep=comma] {../data/multipleJumps_approx_plot_data_WPFalse_contrast7.5_k3_uniform.csv};
                %\addplot+[line width=1.1pt,mark=None] table [x=x, y=y, col sep=comma] {../data/multipleJumps_approx_plot_data_WPFalse_contrast7.5_k3_uniform_T0.5.csv};
                

                \addplot+[line width=3.0pt,mark=None,color=black] table [x=x, y=y, col sep=comma] {../data/exact_plot_data_1D_multipleJumps_contrast11.5.csv};
                \addplot+[line width=1.0pt,mark=None,color=blue!75!white] table [x=x, y=y, col sep=comma] {../data/multipleJumps_approx_plot_data_WPFalse_contrast11.5_k3_uniform_T0.5_L4.csv};
                \addplot+[line width=1.0pt,mark=None,color=orange] table [x=x, y=y, col sep=comma] {../data/multipleJumps_approx_plot_data_WPFalse_contrast11.5_k3_uniform_T0.5_L3.csv};
               

                

                \draw[dashed,gray,very thick] (0.4,-2) -- (0.4,2);
                \draw[dashed,gray,very thick] (0.63188,-2) -- (0.63188,2);
                \node [draw=none] at (axis description cs:0.38,0.125) {\color{gray} $\Gamma_1$};
                \node [draw=none] at (axis description cs:0.71,0.125) {\color{gray} $\Gamma_2$};
                \node [draw=none] at (axis description cs:0.2,0.125) {\color{gray} $\omega$};
                \node [draw=none] at (axis description cs:0.8,0.125) {\color{gray} $\omega$};

            
                

            \end{groupplot}
        \end{tikzpicture}
        \pgfplotslegendfromname{named}
    \end{center}
    \caption{Exact solution (left) and approximated solution with $k = 3$ (right) of \eqref{eq:1D:exact:simpleMult} at $t = 0.5$ with $c_1 = c_3 = 7.5$ and $c_2 = 1.0$.}
    \label{fig:jumpCoefs:multipleJumps}
  \end{figure}


\noindent As a second example, we consider the case where the data is only given in one half of the interval, i.e. $\omega = [0,0.3]$. The GCC is fulfilled if 
\begin{equation}
    T > 2 \left[ (p_2 - p_1) + \frac{p_1 - 0.3 + 1 - p_2}{c_1} \right]. 
\end{equation}
Figure \ref{fig:jumpCoefs:multipleJumps:noGCC} shows once more that the approximate solution matches the exact solution well if the GCC is fulfilled, but fails to do so if the GCC is not fulfilled.



  \begin{figure}[!htbp]
    \begin{center}
        \begin{tikzpicture}[scale=0.72]
            \begin{groupplot}[%
                group style={%
                group size=2 by 1,
                horizontal sep=1.5cm,
                vertical sep=2cm,
                },
            ymajorgrids=true,
            grid style=dashed,
            %ymin = -1.1e0, ymax = 1.1e0,
            ]    
                \nextgroupplot[width=9cm,height=7cm,domain=0:1,xmode=linear,ymode=linear, xlabel={}, ylabel={}, title={$u$ vs $u_h$ ($c_1 = 2.5$)}, %cycle list name=paulcolors, 
                legend style={legend columns=3, draw=none,nodes={scale=.8}},legend to name=named,, %yticklabels={,,},
                ]
                \draw[fill=gray!40,draw=none,opacity=0.5] (0.3,-2) -- (0.3,2) -- (0,2) -- (0,-2) -- cycle;
                \addplot+[line width=3.0pt,mark=None,color=black] table [x=x, y=y, col sep=comma] {../data/exact_plot_data_1D_multipleJumps_contrast2.5.csv};
                \addplot+[line width=1.0pt,mark=None,color=orange] table [x=x, y=y, col sep=comma] {../data/multipleJumps_approx_plot_data_WPFalse_contrast2.5_k3_uniform_noGCC.csv};

                \addplot+[line width=1.0pt,mark=None,color=blue!75!white] table [x=x, y=y, col sep=comma] {../data/multipleJumps_approx_plot_data_WPFalse_contrast2.5_k3_dataLeft_T0.5.csv};
                

                \draw[dashed,gray,very thick] (0.4,-2) -- (0.4,2);
                \draw[dashed,gray,very thick] (0.6666,-2) -- (0.6666,2);
                %\node [draw=none] at (axis description cs:0.45,0.23) {\color{gray} $\Gamma$};

                \node [draw=none] at (axis description cs:0.38,0.125) {\color{gray} $\Gamma_1$};
                \node [draw=none] at (axis description cs:0.71,0.125) {\color{gray} $\Gamma_2$};
                \node [draw=none] at (axis description cs:0.2,0.125) {\color{gray} $\omega$};
                \legend{exact,$T=1.0$,$T=0.5$}
                \nextgroupplot[width=9cm,height=7cm,domain=0:1,xmode=linear,ymode=linear, xlabel={}, ylabel={}, title={$u$ vs $u_h$ ($c_1 = 7.5$)}, %cycle list name=paulcolors, 
                legend pos=north west, %yticklabels={,,},
                ]
                \draw[fill=gray!40,draw=none,opacity=0.5] (0.3,-2) -- (0.3,2) -- (0,2) -- (0,-2) -- cycle;
                \addplot+[line width=3.0pt,mark=None,color=black] table [x=x, y=y, col sep=comma] {../data/exact_plot_data_1D_multipleJumps_contrast7.5.csv};
                \addplot+[line width=1.0pt,mark=None,color=orange] table [x=x, y=y, col sep=comma] {../data/multipleJumps_approx_plot_data_WPFalse_contrast7.5_k3_uniform_noGCC.csv};
                
                \addplot+[line width=1.0pt,mark=None,color=blue!75!white] table [x=x, y=y, col sep=comma] {../data/multipleJumps_approx_plot_data_WPFalse_contrast7.5_k3_dataLeft_T0.5.csv};
    

                \draw[dashed,gray,very thick] (0.4,-2) -- (0.4,2);
                \draw[dashed,gray,very thick] (0.6666,-2) -- (0.6666,2);
                \node [draw=none] at (axis description cs:0.38,0.125) {\color{gray} $\Gamma_1$};
                \node [draw=none] at (axis description cs:0.71,0.125) {\color{gray} $\Gamma_2$};
                \node [draw=none] at (axis description cs:0.2,0.125) {\color{gray} $\omega$};

            
            \end{groupplot}
        \end{tikzpicture}
         \pgfplotslegendfromname{named}
    \end{center}
    \caption{Exact solution (left) and approximated solution with $k = 3$ (right) of \eqref{eq:1D:exact:simpleMult} at $t = T/2 = 0.5$ with $c_1 = c_3 = 2.5$, $n = 1$ (upper) and $c_1 = c_3 = 7.5$, $n = 3$ (lower) and $c_2 = 1.0$. We only prescribe data on $\omega = [0.0,0.3]$ and choose $\Delta t = 1/32$ (upper) and $\Delta t = 1/64$ (lower), $h = 1/64$.}
    \label{fig:jumpCoefs:multipleJumps:noGCC}
  \end{figure}

  \begin{comment}
  \begin{figure}[!htbp]
    \begin{center}
        \begin{tikzpicture}[scale=0.72]
            \begin{groupplot}[%
                group style={%
                group size=2 by 1,
                horizontal sep=1.5cm,
                vertical sep=2cm,
                },
            ymajorgrids=true,
            grid style=dashed,
            ymin = -1.1e0, ymax = 1.1e0,
            ]    
                \nextgroupplot[width=9cm,height=7cm,domain=0:1,xmode=linear,ymode=linear, xlabel={}, ylabel={}, title={$u$}, %cycle list name=paulcolors, 
                legend pos=north west, %yticklabels={,,},
                ]
                \addplot+[line width=1.1pt,mark=None] table [x=x, y=y, col sep=comma] {../../wave_data_assimilation/data/jumpCoefs/exact_plot_data_1D_multipleJumps_contrast11.5.csv};
                

                \draw[dashed,gray,very thick] (0.4,-2) -- (0.4,2);
                \draw[dashed,gray,very thick] (0.63188,-2) -- (0.63188,2);
                %\node [draw=none] at (axis description cs:0.45,0.23) {\color{gray} $\Gamma$};

                \node [draw=none] at (axis description cs:0.38,0.65) {\color{gray} $\Gamma_1$};
                \node [draw=none] at (axis description cs:0.71,0.65) {\color{gray} $\Gamma_2$};

                \nextgroupplot[width=9cm,height=7cm,domain=0:1,xmode=linear,ymode=linear, xlabel={}, ylabel={}, title={$u_h$ ($k = 3$)}, %cycle list name=paulcolors, 
                legend pos=north west, %yticklabels={,,},
                ]
                \draw[fill=gray!40,draw=none,opacity=0.5] (0.3,-2) -- (0.3,2) -- (0,2) -- (0,-2) -- cycle;
                \draw[fill=gray!40,draw=none,opacity=0.5] (0.7,-2) -- (0.7,2) -- (1.0,2) -- (1.0,-2) -- cycle;
                \addplot+[line width=1.1pt,mark=None] table [x=x, y=y, col sep=comma] {../../wave_data_assimilation/data/jumpCoefs/multipleJumps_approx_plot_data_WPFalse_contrast11.5_k3.csv};
                

                \draw[dashed,gray,very thick] (0.4,-2) -- (0.4,2);
                \draw[dashed,gray,very thick] (0.63188,-2) -- (0.63188,2);
                \node [draw=none] at (axis description cs:0.38,0.65) {\color{gray} $\Gamma_1$};
                \node [draw=none] at (axis description cs:0.71,0.65) {\color{gray} $\Gamma_2$};
                \node [draw=none] at (axis description cs:0.2,0.15) {\color{gray} $\omega$};
                \node [draw=none] at (axis description cs:0.8,0.15) {\color{gray} $\omega$};
                

            \end{groupplot}
        \end{tikzpicture}
    \end{center}
    \caption{Exact solution (left) and approximated solution with $k = 3$ (right) of \eqref{eq:1D:exact:simpleMult} at $t = 0.5$ with $c_1 = c_3 = 11.5$ and $c_2 = 1.0$ - with bug in the code....}
    \label{fig:jumpCoefs:multipleJumps2}
  \end{figure}

  \begin{figure}[!htbp]
    \begin{center}
        \begin{tikzpicture}[scale=0.72]
            \begin{groupplot}[%
                group style={%
                group size=2 by 1,
                horizontal sep=1.5cm,
                vertical sep=2cm,
                },
            ymajorgrids=true,
            grid style=dashed,
            %ymin = -1.1e0, ymax = 1.1e0,
            ]    
                \nextgroupplot[width=9cm,height=7cm,domain=0:1,xmode=linear,ymode=linear, xlabel={}, ylabel={}, title={$u$}, %cycle list name=paulcolors, 
                legend pos=north west, %yticklabels={,,},
                ]
                \addplot+[line width=1.1pt,mark=None] table [x=x, y=y, col sep=comma] {../../wave_data_assimilation/data/jumpCoefs/exact_plot_data_1D_multipleJumps_contrast11.5.csv};
                

                \draw[dashed,gray,very thick] (0.4,-2) -- (0.4,2);
                \draw[dashed,gray,very thick] (0.63188,-2) -- (0.63188,2);
                %\node [draw=none] at (axis description cs:0.45,0.23) {\color{gray} $\Gamma$};

                \node [draw=none] at (axis description cs:0.38,0.65) {\color{gray} $\Gamma_1$};
                \node [draw=none] at (axis description cs:0.71,0.65) {\color{gray} $\Gamma_2$};

                \nextgroupplot[width=9cm,height=7cm,domain=0:1,xmode=linear,ymode=linear, xlabel={}, ylabel={}, title={$u_h$ ($k = 3$)}, %cycle list name=paulcolors, 
                legend pos=north west, %yticklabels={,,},
                ]
                \draw[fill=gray!40,draw=none,opacity=0.5] (0.3,-2) -- (0.3,2) -- (0,2) -- (0,-2) -- cycle;
                \draw[fill=gray!40,draw=none,opacity=0.5] (0.7,-2) -- (0.7,2) -- (1.0,2) -- (1.0,-2) -- cycle;
                \addplot+[line width=1.1pt,mark=None] table [x=x, y=y, col sep=comma] {../../wave_data_assimilation/data/jumpCoefs/multipleJumps_approx_plot_data_WPFalse_contrast11.5_k3_uniform.csv};
                

                \draw[dashed,gray,very thick] (0.4,-2) -- (0.4,2);
                \draw[dashed,gray,very thick] (0.63188,-2) -- (0.63188,2);
                \node [draw=none] at (axis description cs:0.38,0.65) {\color{gray} $\Gamma_1$};
                \node [draw=none] at (axis description cs:0.71,0.65) {\color{gray} $\Gamma_2$};
                \node [draw=none] at (axis description cs:0.2,0.15) {\color{gray} $\omega$};
                \node [draw=none] at (axis description cs:0.8,0.15) {\color{gray} $\omega$};
                

            \end{groupplot}
        \end{tikzpicture}
    \end{center}
    \caption{Exact solution (left) and approximated solution with $k = 3$ (right) of \eqref{eq:1D:exact:simpleMult} at $t = 0.25$ with $c_1 = c_3 = 11.5$ and $c_2 = 1.0$ (uniform mesh), computed with $64$ points in space, $32$ in time and $T = 0.5$.}
    \label{fig:jumpCoefs:multipleJumps3}
  \end{figure}
\end{comment}

\subsubsection{2nd one dimensional example}
We consider the same geometrical setup as in Sec. \ref{sec:numex:1D:simple} with a more intricate reference solution considered in \cite{BDE22,MEMM19}. With $u_0 (x) := \frac{1}{100} \exp(-(20(x-\frac{1}{5}))^2)$, we consider the exact solution 
\begin{equation}\label{eq:1D:complicatedSol}
    u(x,t) = \begin{cases}
      \sum_{k \ge 0} \left( \frac{c_2-c_1}{c_2+c_1}\right)^k \left( u_0 (k + x - c_1 t) - u_0(k-x-c_1 t)\right), &x \in \Omega_1, \\
      \left( \frac{2c_1}{c_2+c_1}\right) \sum_{k \ge 0} \left( \frac{c_2 - c_1}{c_2 + c_1}\right)^k u_0 \left( \frac{c_1}{c_2} \left( x - \frac{1}{2}\right) +k + \frac{1}{2} - c_1 t \right), & x \in \Omega_2. 
    \end{cases}
\end{equation}
This solution describes a wave traveling towards the interface $\Gamma$. Due to the jump in the wave speed, a part of the wave is reflected from the interface into the domain $\Omega_1$, while the other part passes through the interface into $\Omega_2$ and oscillates stronger. 


\begin{comment}
\begin{figure}[!htbp]
    \begin{center}with 
        \begin{tikzpicture}[scale=0.72]
            \begin{groupplot}[%
                group style={%
                group size=2 by 2,
                horizontal sep=1.5cm,
                vertical sep=2cm,
                },
            ymajorgrids=true,
            grid style=dashed,
            %ymin = 1e-3, ymax = 0.5e1,
            ]    
                \nextgroupplot[width=9cm,height=7cm,domain=0:4,xmode=linear,ymode=log, xlabel={}, ylabel={}, title={$c_1 = 1.0, \Vert (u - u_h) \Vert_{L^\infty(0,T;L^2(\Omega))}$}, %cycle list name=paulcolors, 
                legend pos=south west, %yticklabels={,,},
                ]
                \addplot+[discard if not={order}{2},discard if not={contrast}{1.0},line width=1.1pt] table [x=L, y=L-infty-L2-error-u, col sep=comma] {../../wave_data_assimilation/data/jumpCoefs/travelingWave_1D_jumpingCoefs_k2_WPFalse_contrast1.0.csv};
                \addplot+[discard if not={order}{2},discard if not={contrast}{1.0},line width=1.1pt,dashed] table [x=L, y=bestapprox-L-infty-L2-error-u, col sep=comma] {../../wave_data_assimilation/data/jumpCoefs/travelingWave_1D_jumpingCoefs_k2_WPFalse_contrast1.0.csv};
                %\addplot+[discard if not={order}{3},discard if not={contrast}{1.0},line width=1.1pt] table [x=L, y=L-infty-L2-error-u, col sep=comma] {../../wave_data_assimilation/data/jumpCoefs/travelingWave_1D_jumpingCoefs_k3_WPFalse_contrast1.0.csv};
                \addplot+[discard if not={order}{3},discard if not={contrast}{1.0},line width=1.1pt,dashed] table [x=L, y=bestapprox-L-infty-L2-error-u, col sep=comma] {../../wave_data_assimilation/data/jumpCoefs/travelingWave_1D_jumpingCoefs_k3_WPFalse_contrast1.0.csv};
                


                %\addplot[gray, dashed, domain=1:4] {0.005*(1/2^(2))^(x-0.9)};
                %\addplot[gray, dashed, domain=1:4] {0.0005*(1/2^(3))^(x-0.9)};
                %\node [draw=none] at (axis description cs:0.80,0.45) {\color{gray}\footnotesize $\!\!\mathcal{O}(h^{2})$};
                %\node [draw=none] at (axis description cs:0.65,0.2) {\color{gray}\footnotesize $\!\!\mathcal{O}(h^{3})$};
    
                \legend{$k=2$,best,$k = 3$,best}
                \nextgroupplot[width=9cm,height=7cm,domain=0:1,xmode=linear,ymode=log, xlabel={}, ylabel={}, title={$c_1 = 1.5, \Vert (u - u_h) \Vert_{L^\infty(0,T;L^2(\Omega))}$}, %cycle list name=paulcolors, 
                legend pos=south west, %yticklabels={,,},
                ]
                \addplot+[discard if not={order}{2},discard if not={contrast}{1.5},line width=1.1pt] table [x=L, y=L-infty-L2-error-u, col sep=comma] {../../wave_data_assimilation/data/jumpCoefs/travelingWave_1D_jumpingCoefs_k2_WPFalse_contrast1.5.csv};
                \addplot+[discard if not={order}{2},discard if not={contrast}{1.5},line width=1.1pt,dashed] table [x=L, y=bestapprox-L-infty-L2-error-u, col sep=comma] {../../wave_data_assimilation/data/jumpCoefs/travelingWave_1D_jumpingCoefs_k2_WPFalse_contrast1.5.csv};
                \addplot+[discard if not={order}{3},discard if not={contrast}{1.5},line width=1.1pt] table [x=L, y=L-infty-L2-error-u, col sep=comma] {../../wave_data_assimilation/data/jumpCoefs/travelingWave_1D_jumpingCoefs_k3_WPFalse_contrast1.5.csv};
                \addplot+[discard if not={order}{3},discard if not={contrast}{1.5},line width=1.1pt,dashed] table [x=L, y=bestapprox-L-infty-L2-error-u, col sep=comma] {../../wave_data_assimilation/data/jumpCoefs/travelingWave_1D_jumpingCoefs_k3_WPFalse_contrast1.5.csv};

                %\addplot[gray, dashed, domain=1:4] {6.5*(1/2^(2))^(x-0.9)};
                %\addplot[gray, dashed, domain=1:4] {1.5*(1/2^(3))^(x-0.9)};
                %\node [draw=none] at (axis description cs:0.80,0.40) {\color{gray}\footnotesize $\!\!\mathcal{O}(h^{2})$};
                %\node [draw=none] at (axis description cs:0.65,0.2) {\color{gray}\footnotesize $\!\!\mathcal{O}(h^{3})$};
                \legend{$k=2$,best,$k = 3$,best}

                \nextgroupplot[width=9cm,height=7cm,domain=0:1,xmode=linear,ymode=log, xlabel={}, ylabel={}, title={$c_1 = 2.0, \Vert (u - u_h) \Vert_{L^\infty(0,T;L^2(\Omega))}$}, %cycle list name=paulcolors, 
                legend pos=south west, %yticklabels={,,},
                ]
                \addplot+[discard if not={order}{2},discard if not={contrast}{2.0},line width=1.1pt] table [x=L, y=L-infty-L2-error-u, col sep=comma] {../../wave_data_assimilation/data/jumpCoefs/travelingWave_1D_jumpingCoefs_k2_WPFalse_contrast2.0.csv};
                \addplot+[discard if not={order}{2},discard if not={contrast}{2.0},line width=1.1pt,dashed] table [x=L, y=bestapprox-L-infty-L2-error-u, col sep=comma] {../../wave_data_assimilation/data/jumpCoefs/travelingWave_1D_jumpingCoefs_k2_WPFalse_contrast2.0.csv};
                \addplot+[discard if not={order}{3},discard if not={contrast}{2.0},line width=1.1pt] table [x=L, y=L-infty-L2-error-u, col sep=comma] {../../wave_data_assimilation/data/jumpCoefs/travelingWave_1D_jumpingCoefs_k3_WPFalse_contrast2.0.csv};
                \addplot+[discard if not={order}{3},discard if not={contrast}{2.0},line width=1.1pt,dashed] table [x=L, y=bestapprox-L-infty-L2-error-u, col sep=comma] {../../wave_data_assimilation/data/jumpCoefs/travelingWave_1D_jumpingCoefs_k3_WPFalse_contrast2.0.csv};

                %\addplot[gray, dashed, domain=1:4] {12.5*(1/2^(2))^(x-0.9)};
                %\addplot[gray, dashed, domain=1:4] {5.5*(1/2^(3))^(x-0.9)};
                %\node [draw=none] at (axis description cs:0.80,0.40) {\color{gray}\footnotesize $\!\!\mathcal{O}(h^{2})$};
                %\node [draw=none] at (axis description cs:0.65,0.2) {\color{gray}\footnotesize $\!\!\mathcal{O}(h^{3})$};
                \legend{$k=2$,best,$k = 3$,best}

                \nextgroupplot[width=9cm,height=7cm,domain=0:1,xmode=linear,ymode=log, xlabel={}, ylabel={}, title={$c_1 = 2.5, \Vert (u - u_h) \Vert_{L^\infty(0,T;L^2(\Omega))}$}, %cycle list name=paulcolors, 
                legend pos=south west, %yticklabels={,,},
                ]
                \addplot+[discard if not={order}{2},discard if not={contrast}{2.5},line width=1.1pt] table [x=L, y=L-infty-L2-error-u, col sep=comma] {../../wave_data_assimilation/data/jumpCoefs/travelingWave_1D_jumpingCoefs_k2_WPFalse_contrast2.5.csv};
                \addplot+[discard if not={order}{2},discard if not={contrast}{2.5},line width=1.1pt,dashed] table [x=L, y=bestapprox-L-infty-L2-error-u, col sep=comma] {../../wave_data_assimilation/data/jumpCoefs/travelingWave_1D_jumpingCoefs_k2_WPFalse_contrast2.5.csv};
                \addplot+[discard if not={order}{3},discard if not={contrast}{2.5},line width=1.1pt] table [x=L, y=L-infty-L2-error-u, col sep=comma] {../../wave_data_assimilation/data/jumpCoefs/travelingWave_1D_jumpingCoefs_k3_WPFalse_contrast2.5.csv};
                \addplot+[discard if not={order}{3},discard if not={contrast}{2.5},line width=1.1pt,dashed] table [x=L, y=bestapprox-L-infty-L2-error-u, col sep=comma] {../../wave_data_assimilation/data/jumpCoefs/travelingWave_1D_jumpingCoefs_k3_WPFalse_contrast2.5.csv};

                %\addplot[gray, dashed, domain=1:4] {22.5*(1/2^(2))^(x-0.9)};
                %\addplot[gray, dashed, domain=1:4] {12.5*(1/2^(3))^(x-0.9)};
                %\node [draw=none] at (axis description cs:0.80,0.40) {\color{gray}\footnotesize $\!\!\mathcal{O}(h^{2})$};
                %\node [draw=none] at (axis description cs:0.65,0.2) {\color{gray}\footnotesize $\!\!\mathcal{O}(h^{3})$};
                \legend{$k=2$,best,$k = 3$,best}
            \end{groupplot}
        \end{tikzpicture}
    \end{center}
    \caption{...}
    %\label{fig:2DjumpCoefs}
  \end{figure}
\end{comment}

  \begin{figure}[!htbp]
    \begin{center}
        \begin{tikzpicture}[scale=0.72]
            \begin{groupplot}[%
                group style={%
                group size=2 by 1,
                horizontal sep=1.5cm,
                vertical sep=2cm,
                },
            ymajorgrids=true,
            grid style=dashed,
            %ymin = 1e-3, ymax = 0.5e1,
            ]    
                \nextgroupplot[width=9cm,height=7cm,domain=0:4,xmode=linear,ymode=log, xlabel={}, ylabel={}, title={$L^2$-best approximation ($k = 2$)}, %cycle list name=paulcolors, 
                legend pos=south west, 
                %yticklabels={,,}, 
                xlabel={$L$}, ylabel={$\Vert (u - u_h) \Vert_{L^\infty(0,T;L^2(\Omega))}$},
                ]
                \addplot+[discard if not={order}{2},discard if not={contrast}{1.0},line width=1.1pt] table [x=L, y=bestapprox-L-infty-L2-error-u, col sep=comma] {../data/weakDisc_noRestrict_1D_jumpingCoefs_k2.csv};
                \addplot+[discard if not={order}{2},discard if not={contrast}{1.5},line width=1.1pt] table [x=L, y=bestapprox-L-infty-L2-error-u, col sep=comma] {../data/weakDisc_noRestrict_1D_jumpingCoefs_k2.csv};
                \addplot+[discard if not={order}{2},discard if not={contrast}{2.0},line width=1.1pt] table [x=L, y=bestapprox-L-infty-L2-error-u, col sep=comma] {../data/weakDisc_noRestrict_1D_jumpingCoefs_k2.csv};
                %\addplot+[discard if not={order}{2},discard if not={contrast}{2.0},line width=1.1pt,dashed] table [x=L, y=L-infty-L2-error-u, col sep=comma] {../data/weakDisc_noRestrict_1D_jumpingCoefs_k2.csv};
             
             
                \addplot[gray, dashed, domain=1:4] {0.0025*(1/2^(2))^(x-0.9)};
                \addplot[gray, dashed, domain=1:4] {0.005*(1/2^(3/2))^(x-0.9)};

                \node [draw=none] at (axis description cs:0.50,0.40) {\color{gray}\footnotesize $\!\!\mathcal{O}(h^{2})$};
                \node [draw=none] at (axis description cs:0.90,0.60) {\color{gray}\footnotesize $\!\!\mathcal{O}(h^{3/2})$};


                \legend{$c_1 = 1.0$,$c_1 = 1.5$,$c_1 = 2.0$}
                \nextgroupplot[width=9cm,height=7cm,domain=0:4,xmode=linear,ymode=linear, xlabel={}, ylabel={}, title={$\dt u$ vs. $\dt u_h$ ($k = 3$)}, %cycle list name=paulcolors, 
                legend pos=north west, %yticklabels={,,},
                ]
                \draw[fill=gray!40,draw=none,opacity=0.5] (0.25,-2) -- (0.25,2) -- (0,2) -- (0,-2) -- cycle;
                \draw[fill=gray!40,draw=none,opacity=0.5] (0.75,-2) -- (0.75,2) -- (1.0,2) -- (1.0,-2) -- cycle;


                \addplot+[discard if not={contrast}{2.0},line width=2.5pt,mark=None,color=black] table [x=x, y=y_dt, col sep=comma] {../data/exact_plot_data_1D_complexSol.csv};
                \addplot+[discard if not={L}{3},line width=1.1pt,mark=none] table [x=x, y=y_dt, col sep=comma] {../data/weakDisc_higherDt_noRestrict_approx_plot_data_WPFalse_contrast2.0_k3.csv};
                %\addplot+[discard if not={L}{3},line width=1.1pt,mark=none,dotted] table [x=x, y=y_dt, col sep=comma] {../data/weakDisc_noRestrict_approx_plot_data_WPFalse_contrast2.0_k3.csv};
                \addplot+[discard if not={L}{4},line width=1.1pt,mark=none] table [x=x, y=y_dt, col sep=comma] {../data/weakDisc_higherDt_noRestrict_approx_plot_data_WPFalse_contrast2.0_k3.csv};
                
                \draw[dashed,gray,very thick] (0.5,-2) -- (0.5,2);
                \node [draw=none] at (axis description cs:0.53,0.125) {\color{gray} $\Gamma$};
                \node [draw=none] at (axis description cs:0.175,0.125) {\color{gray} $\omega$};
                \node [draw=none] at (axis description cs:0.83,0.125) {\color{gray} $\omega$};
             
                %\addplot[gray, dashed, domain=1:4] {0.01*(1/2^(2))^(x-0.9)};
                %\addplot[gray, dashed, domain=1:4] {0.025*(1/2^(3/2))^(x-0.9)};

                %L = 2
                %\draw[color=black,thick] (0.75,-0.7) -- (1,-0.7);
                %\draw[color=black,thick] (0.75,-0.68) -- (0.75,-0.72);
                %\draw[color=black,thick] (0.875,-0.68) -- (0.875,-0.72);
                %\draw[color=black,thick] (1.0,-0.68) -- (1.0,-0.72);
                
                %L = 3
                %\draw[color=black,thick] (0.75,-0.6) -- (1,-0.6);
                %\draw[color=black,thick] (0.75,-0.58) -- (0.75,-0.62);
                %\draw[color=black,thick] (0.8125,-0.58) -- (0.8125,-0.62);
                %\draw[color=black,thick] (0.875,-0.58) -- (0.875,-0.62);
                %\draw[color=black,thick] (0.9375,-0.58) -- (0.9375,-0.62);
                %\draw[color=black,thick] (1.0,-0.58) -- (1.0,-0.62);
                %\node[draw=none] at (0.67,-0.6) {\color{black} \footnotesize $L = 3$};

                \legend{exact,$L=4$, $L=5$}
            \end{groupplot}
        \end{tikzpicture}
    \end{center}
    \caption{We consider the approximation of \eqref{eq:1D:complicatedSol} with $h = 1/2^{(L+1)}$ and $\Delta t = 1/2^L$, \textbf{now $\Delta t = 1/2^{L+1}$ on the right}.}
    %\label{fig:2DjumpCoefs}
  \end{figure}




\subsection{Example in two space dimensions}
\noindent In this section, we present numerical examples in two space dimensions. We consider the unit-square $\Omega \coloneqq [0,1]^2 \subset \mathbb{R}^2$ partitioned into subdomains $\Omega_1 = (0.0,0.5)^2$ and $\Omega^2 = (0.5,1.0)^2$. 

\subsubsection{Simple example} 
As a simple test case, we can extend the one-dimensional exact solution defined in \eqref{eq:1D:exact:simple} to the two-dimensional case by setting $u(x,y;t) = u(x;t)$, i.e. we consider the original solution in the $x$-direction. We define the data domain $\omega = \Omega \setminus [0.25,0.75]^2$ and note that $\omega$ fulfills the GCC. To ensure that the data domain and the interface $\Gamma$ are meshed exactly, we consider a quadrilateral mesh with $2^{L+1}$, $L \in \{1,2,3,4\}$, elements in each direction. We consider the final time $T = 0.75$  and set the time step size to $\Delta t = (1/2)^{L+1}$. As before, we fix $c_2 = 1$ and consider different values for $c_1$. The results of the experiment are presented in Fig. \ref{fig:2DjumpCoefs}. 


\begin{comment}
  \begin{figure}[!htbp]
    \begin{center}
        \begin{tikzpicture}[scale=0.72]
            \begin{groupplot}[%
                group style={%
                group size=2 by 2,
                horizontal sep=1.5cm,
                vertical sep=2cm,
                },
            ymajorgrids=true,
            grid style=dashed,
            %ymin = 1e-3, ymax = 0.5e1,
            ]    
                \nextgroupplot[width=9cm,height=7cm,domain=0:4,xmode=linear,ymode=log, xlabel={}, ylabel={}, title={$c_1 = 1.0$}, %cycle list name=paulcolors, 
                legend pos=north east, %yticklabels={,,},
                ]
                %\addplot+[discard if not={contrast}{1.0},line width=2.5pt,mark=None,color=black] table [x=x, y=y, col sep=comma] {../../wave_data_assimilation/data/jumpCoefs/exact_plot_data.csv};
                %\addplot+[discard if not={order}{2},discard if not={contrast}{1.0},line width=1.1pt] table [x=L, y=L-infty-L2-error-ut, col sep=comma] {../../wave_data_assimilation/data/jumpCoefs/2D_errors_simpleRefsol_c1.0.csv};
                \addplot+[discard if not={order}{2},discard if not={contrast}{1.0},line width=1.1pt,color=orange] table [x=L, y=L2-L2-error-u_t, col sep=comma] {../../wave_data_assimilation/data/jumpCoefs/2D_errors_simpleRefsol_c1.0.csv};

            
               
                %\addplot+[discard if not={order}{2},discard if not={contrast}{1.0},line width=1.1pt,dotted] table [x=L, y=bestapprox-L-infty-L2-error-ut, col sep=comma] {../../wave_data_assimilation/data/jumpCoefs/2D_errors_simpleRefsol_c1.0.csv};
                %\addplot+[discard if not={contrast}{1.0},line width=1.1pt] table [x=L, y=L-infty-L2-error-ut, col sep=comma] {../../wave_data_assimilation/data/jumpCoefs/2D_errors_simpleRefsol_k3.csv};
                \addplot+[discard if not={contrast}{1.0},line width=1.1pt,color=teal] table [x=L, y=L2-L2-error-u_t, col sep=comma] {../../wave_data_assimilation/data/jumpCoefs/2D_errors_simpleRefsol_k3.csv};
                %\addplot+[discard if not={contrast}{1.0},line width=1.1pt,dotted] table [x=L, y=bestapprox-L-infty-L2-error-ut, col sep=comma] {../../wave_data_assimilation/data/jumpCoefs/2D_errors_simpleRefsol_k3.csv};

                %L-infty-L2-u error
                \addplot+[discard if not={order}{2},discard if not={contrast}{1.0},line width=1.1pt,dashed,color=orange] table [x=L, y=L-infty-L2-error-u, col sep=comma] {../../wave_data_assimilation/data/jumpCoefs/2D_errors_simpleRefsol_c1.0.csv};
                \addplot+[discard if not={contrast}{1.0},line width=1.1pt,dashed,color=teal] table [x=L, y=L-infty-L2-error-u, col sep=comma] {../../wave_data_assimilation/data/jumpCoefs/2D_errors_simpleRefsol_k3.csv};


                \addplot[gray, dashed, domain=1:4] {0.7*(1/2^(2))^(x-0.9)};
                \addplot[gray, dashed, domain=1:4] {0.1*(1/2^(3))^(x-0.9)};
                \addplot[gray, dashed, domain=1:4] {0.005*(1/2^(3))^(x-0.9)};
                \node [draw=none] at (axis description cs:0.80,0.45) {\color{gray}\footnotesize $\!\!\mathcal{O}(h^{2})$};
                \node [draw=none] at (axis description cs:0.65,0.2) {\color{gray}\footnotesize $\!\!\mathcal{O}(h^{3})$};
    
                \legend{$k=2$,$k=3$}
                \nextgroupplot[width=9cm,height=7cm,domain=0:1,xmode=linear,ymode=log, xlabel={}, ylabel={}, title={$c_1 = 1.5$}, %cycle list name=paulcolors, 
                legend pos=north east, %yticklabels={,,},
                ]
                %\addplot+[discard if not={order}{2},discard if not={contrast}{1.5},line width=1.1pt] table [x=L, y=L-infty-L2-error-ut, col sep=comma] {../../wave_data_assimilation/data/jumpCoefs/2D_errors_simpleRefsol_c1.5.csv};
                \addplot+[discard if not={order}{2},discard if not={contrast}{1.5},line width=1.1pt,color=orange] table [x=L, y=L2-L2-error-u_t, col sep=comma] {../../wave_data_assimilation/data/jumpCoefs/2D_errors_simpleRefsol_c1.5.csv};

                %\addplot+[discard if not={contrast}{1.5},line width=1.1pt] table [x=L, y=L-infty-L2-error-ut, col sep=comma] {../../wave_data_assimilation/data/jumpCoefs/2D_errors_simpleRefsol_k3.csv};
                \addplot+[discard if not={contrast}{1.5},line width=1.1pt,color=teal] table [x=L, y=L2-L2-error-u_t, col sep=comma] {../../wave_data_assimilation/data/jumpCoefs/2D_errors_simpleRefsol_k3.csv};

                %L-infty-L2-u error
                \addplot+[discard if not={order}{2},discard if not={contrast}{1.5},line width=1.1pt,dashed,color=orange] table [x=L, y=L-infty-L2-error-u, col sep=comma] {../../wave_data_assimilation/data/jumpCoefs/2D_errors_simpleRefsol_c1.0.csv};
                \addplot+[discard if not={contrast}{1.5},line width=1.1pt,dashed,color=teal] table [x=L, y=L-infty-L2-error-u, col sep=comma] {../../wave_data_assimilation/data/jumpCoefs/2D_errors_simpleRefsol_k3.csv};


                \addplot[gray, dashed, domain=1:4] {3*(1/2^(2))^(x-0.9)};
                \addplot[gray, dashed, domain=1:4] {0.75*(1/2^(3))^(x-0.9)};
                \addplot[gray, dashed, domain=1:4] {0.025*(1/2^(3))^(x-0.9)};
                \node [draw=none] at (axis description cs:0.80,0.40) {\color{gray}\footnotesize $\!\!\mathcal{O}(h^{2})$};
                \node [draw=none] at (axis description cs:0.65,0.2) {\color{gray}\footnotesize $\!\!\mathcal{O}(h^{3})$};
                \legend{$k=2$,$k = 3$}

                \nextgroupplot[width=9cm,height=7cm,domain=0:1,xmode=linear,ymode=log, xlabel={}, ylabel={}, title={$c_1 = 2.0$}, %cycle list name=paulcolors, 
                legend pos=north east, %yticklabels={,,},
                ]
                %\addplot+[discard if not={order}{2},discard if not={contrast}{2.0},line width=1.1pt] table [x=L, y=L-infty-L2-error-ut, col sep=comma] {../../wave_data_assimilation/data/jumpCoefs/2D_errors_simpleRefsol_c2.0.csv};
                \addplot+[discard if not={order}{2},discard if not={contrast}{2.0},line width=1.1pt,color=orange] table [x=L, y=L2-L2-error-u_t, col sep=comma] {../../wave_data_assimilation/data/jumpCoefs/2D_errors_simpleRefsol_c2.0.csv};
                

                %\addplot+[discard if not={contrast}{2.0},line width=1.1pt] table [x=L, y=L-infty-L2-error-ut, col sep=comma] {../../wave_data_assimilation/data/jumpCoefs/2D_errors_simpleRefsol_k3.csv};
                \addplot+[discard if not={contrast}{2.0},line width=1.1pt,color=teal] table [x=L, y=L2-L2-error-u_t, col sep=comma] {../../wave_data_assimilation/data/jumpCoefs/2D_errors_simpleRefsol_k3.csv};

                %L-infty-L2-u error
                \addplot+[discard if not={order}{2},discard if not={contrast}{2.0},line width=1.1pt,dashed,color=orange] table [x=L, y=L-infty-L2-error-u, col sep=comma] {../../wave_data_assimilation/data/jumpCoefs/2D_errors_simpleRefsol_c1.0.csv};
                \addplot+[discard if not={contrast}{2.0},line width=1.1pt,dashed,color=teal] table [x=L, y=L-infty-L2-error-u, col sep=comma] {../../wave_data_assimilation/data/jumpCoefs/2D_errors_simpleRefsol_k3.csv};


                \addplot[gray, dashed, domain=1:4] {6.5*(1/2^(2))^(x-0.9)};
                \addplot[gray, dashed, domain=1:4] {2*(1/2^(3))^(x-0.9)};
                \addplot[gray, dashed, domain=1:4] {0.06*(1/2^(3))^(x-0.9)};
                \node [draw=none] at (axis description cs:0.80,0.40) {\color{gray}\footnotesize $\!\!\mathcal{O}(h^{2})$};
                \node [draw=none] at (axis description cs:0.65,0.2) {\color{gray}\footnotesize $\!\!\mathcal{O}(h^{3})$};
                \legend{$k=2$,$k = 3$}

                \nextgroupplot[width=9cm,height=7cm,domain=0:1,xmode=linear,ymode=log, xlabel={}, ylabel={}, title={$c_1 = 2.5$}, %cycle list name=paulcolors, 
                legend pos=north east, %yticklabels={,,},
                ]
                %\addplot+[discard if not={order}{2},discard if not={contrast}{2.5},line width=1.1pt] table [x=L, y=L-infty-L2-error-ut, col sep=comma] {../../wave_data_assimilation/data/jumpCoefs/2D_errors_simpleRefsol_c2.5.csv};
                \addplot+[discard if not={order}{2},discard if not={contrast}{2.5},line width=1.1pt,color=orange] table [x=L, y=L2-L2-error-u_t, col sep=comma] {../../wave_data_assimilation/data/jumpCoefs/2D_errors_simpleRefsol_c2.5.csv};
               

                %\addplot+[discard if not={contrast}{2.5},line width=1.1pt] table [x=L, y=L-infty-L2-error-ut, col sep=comma] {../../wave_data_assimilation/data/jumpCoefs/2D_errors_simpleRefsol_k3.csv};
                \addplot+[discard if not={contrast}{2.5},line width=1.1pt,color=teal] table [x=L, y=L2-L2-error-u_t, col sep=comma] {../../wave_data_assimilation/data/jumpCoefs/2D_errors_simpleRefsol_k3.csv};
                
                %L-infty-L2-u error
                \addplot+[discard if not={order}{2},discard if not={contrast}{2.5},line width=1.1pt,dashed,color=orange] table [x=L, y=L-infty-L2-error-u, col sep=comma] {../../wave_data_assimilation/data/jumpCoefs/2D_errors_simpleRefsol_c1.0.csv};
                \addplot+[discard if not={contrast}{2.5},line width=1.1pt,dashed,color=teal] table [x=L, y=L-infty-L2-error-u, col sep=comma] {../../wave_data_assimilation/data/jumpCoefs/2D_errors_simpleRefsol_k3.csv};


                \addplot[gray, dashed, domain=1:4] {11*(1/2^(2))^(x-0.9)};
                \addplot[gray, dashed, domain=1:4] {6*(1/2^(3))^(x-0.9)};
                \addplot[gray, dashed, domain=1:4] {0.1*(1/2^(3))^(x-0.9)};
                \node [draw=none] at (axis description cs:0.80,0.40) {\color{gray}\footnotesize $\!\!\mathcal{O}(h^{2})$};
                \node [draw=none] at (axis description cs:0.65,0.2) {\color{gray}\footnotesize $\!\!\mathcal{O}(h^{3})$};
                \legend{$k=2$,$k = 3$}
            \end{groupplot}
        \end{tikzpicture}
    \end{center}
    \caption{Convergence of $\Vert \dt(u-u_h) \Vert_{L^2(0,T;L^2(\Omega))}$ and $\Vert u - u_h \Vert_{L^\infty(0,T;L^2(\Omega))}$ (dashed) for polynomial degrees $k \in \{2,3\}$ and increasing values of $c_1 \in \{1.0,1.5,2.0,2.5\}$.}
    \label{fig:2DjumpCoefs}
  \end{figure}
\end{comment}

  \begin{figure}[!htbp]
    \begin{center}
        \begin{tikzpicture}[scale=0.72]
            \begin{groupplot}[%
                group style={%
                group size=2 by 1,
                horizontal sep=1.5cm,
                vertical sep=2cm,
                },
            ymajorgrids=true,
            grid style=dashed,
            %ymin = 1e-3, ymax = 0.5e1,
            ]    
                \nextgroupplot[width=9cm,height=7cm,domain=0:4,xmode=linear,ymode=log, xlabel={L}, ylabel={}, title={$c_1 = 2.5$}, %cycle list name=paulcolors, 
                legend pos=north east, %yticklabels={,,},
                ]
                %\addplot+[discard if not={contrast}{1.0},line width=2.5pt,mark=None,color=black] table [x=x, y=y, col sep=comma] {../data/exact_plot_data.csv};
                %\addplot+[discard if not={order}{2},discard if not={contrast}{1.0},line width=1.1pt] table [x=L, y=L-infty-L2-error-ut, col sep=comma] {../data/2D_errors_simpleRefsol_c1.0.csv};
                \addplot+[discard if not={order}{2},discard if not={contrast}{2.5},line width=1.1pt,color=orange] table [x=L, y=L2-L2-error-u_t, col sep=comma] {../data/2D_errors_simpleRefsol_c2.5.csv};

            
               
                %\addplot+[discard if not={order}{2},discard if not={contrast}{1.0},line width=1.1pt,dotted] table [x=L, y=bestapprox-L-infty-L2-error-ut, col sep=comma] {../data/2D_errors_simpleRefsol_c1.0.csv};
                %\addplot+[discard if not={contrast}{1.0},line width=1.1pt] table [x=L, y=L-infty-L2-error-ut, col sep=comma] {../data/2D_errors_simpleRefsol_k3.csv};
                \addplot+[discard if not={contrast}{2.5},line width=1.1pt,color=teal] table [x=L, y=L2-L2-error-u_t, col sep=comma] {../data/2D_errors_simpleRefsol_k3.csv};
                %\addplot+[discard if not={contrast}{1.0},line width=1.1pt,dotted] table [x=L, y=bestapprox-L-infty-L2-error-ut, col sep=comma] {../data/2D_errors_simpleRefsol_k3.csv};

                %L-infty-L2-u errorreduced
                \addplot+[discard if not={order}{2},discard if not={contrast}{2.5},line width=1.1pt,dashed,color=orange] table [x=L, y=L-infty-L2-error-u, col sep=comma] {../data/2D_errors_simpleRefsol_c2.5.csv};
                \addplot+[discard if not={contrast}{2.5},line width=1.1pt,dashed,color=teal] table [x=L, y=L-infty-L2-error-u, col sep=comma] {../data/2D_errors_simpleRefsol_k3.csv};

                %\addplot+[discard if not={order}{2},discard if not={contrast}{1.0},line width=1.1pt,dotted,color=orange] table [x=L, y=bestapprox-L-infty-L2-error-u, col sep=comma] {../data/2D_errors_simpleRefsol_c1.0.csv};
                %\addplot+[discard if not={contrast}{1.0},line width=1.1pt,dotted,color=teal] table [x=L, y=bestapprox-L-infty-L2-error-u, col sep=comma] {../data/2D_errors_simpleRefsol_k3.csv};

                \addplot[gray, dashed, domain=1:4] {9*(1/2^(2))^(x-0.9)};
                \addplot[gray, dashed, domain=1:4] {0.65*(1/2^(2))^(x-0.9)};
                \addplot[gray, dashed, domain=1:4] {0.15*(1/2^(3))^(x-0.9)};
                \addplot[gray, dashed, domain=1:4] {4.5*(1/2^(3))^(x-0.9)};
                \node [draw=none] at (axis description cs:0.50,0.52) {\color{gray}\footnotesize $\!\!\mathcal{O}(h^{2})$};
                \node [draw=none] at (axis description cs:0.45,0.35) {\color{gray}\footnotesize $\!\!\mathcal{O}(h^{3})$};
                \node [draw=none] at (axis description cs:0.60,0.82) {\color{gray}\footnotesize $\!\!\mathcal{O}(h^{2})$};
                \node [draw=none] at (axis description cs:0.6,0.65) {\color{gray}\footnotesize $\!\!\mathcal{O}(h^{3})$};
    

                \legend{$k=2$,$k=3$}
                \nextgroupplot[width=9cm,height=7cm,domain=0:4,xmode=linear,ymode=log, xlabel={$c_1$}, ylabel={}, title={}, %cycle list name=paulcolors, 
                legend pos=north west, %yticklabels={,,},
                ]
                
                \addplot+[discard if not={order}{2},discard if not={L}{4},line width=1.1pt,color=orange] table [x=contrast, y expr= abs(\thisrow{L2-L2-error-u_t}/\thisrow{bestapprox-L2-L2-error-u_t}), col sep=comma] {../data/2D_errors_simpleRefsol_c2.5.csv};

                \addplot+[,discard if not={L}{4},line width=1.1pt,color=teal] table [x=contrast, y expr= abs(\thisrow{L2-L2-error-u_t}/\thisrow{bestapprox-L2-L2-error-u_t}), col sep=comma] {../data/2D_errors_simpleRefsol_k3.csv};
               

               
                %L-infty-L2-u error
                \addplot+[discard if not={order}{2},discard if not={L}{4},line width=1.1pt,dashed,color=orange] table [x=contrast, y expr= abs(\thisrow{L-infty-L2-error-u}/\thisrow{bestapprox-L-infty-L2-error-u}), col sep=comma] {../data/2D_errors_simpleRefsol_c2.5.csv};
                \addplot+[discard if not={L}{4},line width=1.1pt,dashed,color=teal] table [x=contrast, y expr= abs(\thisrow{L-infty-L2-error-u}/\thisrow{bestapprox-L-infty-L2-error-u}), col sep=comma] {../data/2D_errors_simpleRefsol_k3.csv};
               
    
                \addplot[gray, dashed, domain=1:2.5] {2*(1/2^(-3/2))^(x-0.9)};
                \addplot[gray, dashed, domain=1:2.5] {1.35*(1/2^(-1/2))^(x-0.9)};
            
                \node [draw=none] at (axis description cs:0.70,0.7) {\color{gray}\footnotesize $\!\!\mathcal{O}(c_1^{3/2})$};
                \node [draw=none] at (axis description cs:0.7,0.45) {\color{gray}\footnotesize $\!\!\mathcal{O}(c_1^{1/2})$};
    
                \legend{$k=2$,$k=3$}
               
            \end{groupplot}
        \end{tikzpicture}
    \end{center}
    \caption{For $c_1 = 2.5$, $\Vert \dt(u-u_h) \Vert_{L^2(0,T;L^2(\Omega))}$ (solid) and $\Vert u - u_h \Vert_{L^\infty(0,T;L^2(\Omega))}$ (dashed) converge quasi-optimally with order $\mathcal{O}(h^k)$ for polynomial degrees $k \in \{2,3\}$ (left).}
    \label{fig:2DjumpCoefs}
  \end{figure}


\begin{comment}
  \begin{figure}[!htbp]
    \begin{center}
        \begin{tikzpicture}[scale=0.72]
            \begin{groupplot}[%
                group style={%
                group size=1 by 1,
                horizontal sep=1.5cm,
                vertical sep=2cm,
                },0m,domain=1:4,xmode=linear,ymode=log, xlabel={}, ylabel={}, title={$\Vert u_0 - u_h \vert_{t = 0} \Vert_{H^1(\Omega)} + \Vert \partial_t (u - u_h) \vert_{t = 0} \Vert_{L^2(\Omega)}$}, %cycle list name=paulcolors, 
                legend pos=south east, %ymax = 1e2, %yticklabels={,,},
                ]
                %\addplot+[discard if not={contrast}{1.0},line width=2.5pt,mark=None,color=black] table [x=x, y=y, col sep=comma] {../../wave_data_assimilation/data/jumpCoefs/exact_plot_data.csv};
                \addplot+[discard if not={order}{2},discard if not={L}{3},line width=1.1pt] table [x=contrast, y=H1-dt-at-t-0, col sep=comma] {../../wave_data_assimilation/data/jumpCoefs/2D_errors_simpleRefsol_noGCC2_Restricted.csv};
                \addplot+[discard if not={order}{2},discard if not={L}{3},line width=1.1pt,dashed] table [x=contrast, y=bestapprox-H1-dt-at-t-0, col sep=comma] {../../wave_data_assimilation/data/0

                \addplot[gray, dashed, domain=1:4] {1*(1/2^(-2))^(x-0.9)};
                \addplot[gray, dashed, domain=1:4] {1*(1/2^(-1))^(x-0.9)};
                %\addplot[gray, dashed, domain=1:4] {0.003*(1/2^(-2))^(x-0.9)};
                
    
                \legend{$k=2$, best, $e^\Lambda$ (rescaled)}
            
            \end{groupplot}
        \end{tikzpicture}
    \end{center}
    \caption{Filippas lambda... (2D), noGCC}
  \end{figure}
\end{comment}

\subsubsection{Snell's law... }
\noindent Finally, we consider a more elaborate example. We consider the initial conditions 
\begin{align*}
    u_0 (x,y) &= \frac{1}{100} \exp(-(80(x+y-t-1/5))^2) \vert_{t=0}, \\
    u_1(x,y) &= 128(x+y-t-1/5) \exp(-(80(x+y-t-1/5))^2) \vert_{t = 0}
\end{align*}
and compute a reference solution $u_{\text{ref}}$ numerically. We consider the data domain $\omega_T$ as described in Fig. \ref{fig:2d:SnellsLaw} and presribe $u = \Pi_h u_{\text{ref}}$ on $\omega_T$. 


\newcommand{\hpoint}{0.85}
\begin{figure}
    \centering
    \begin{tikzpicture}[scale=4]
        \draw (0,0) -- (1,0) -- (1,1) -- (0,1) -- cycle;
        \draw[fill=gray!40,draw=none,opacity=0.5] (0,1) -- (0,\hpoint) -- (0.65,\hpoint) -- (0.65,1-\hpoint) -- (0,1-\hpoint) -- (0,0) -- (1,0) -- (1,1) -- cycle;
        \node[draw=none] at (0.85,0.25) {\color{gray} $\omega$};
        \draw[stealth-stealth] (0.7,1-\hpoint) -- (0.7,\hpoint); 
        \node[draw=none] at (0.8,0.5) {\color{black} $h_\omega$};
        \draw[dashed] (0.5,0.5) -- (0,0.5); 
        \draw[-stealth,red,very thick] (0,0.2) -- (0.49,0.5) -- (0,0.8);
        \draw[black,very thick] (0.5,0) -- (0.5,1); 
        \node[draw=none] at (0.55,0.5) {\color{black} $\Gamma$};
        \draw (0.25,0.5) to [bend right=45] (0.25,0.35);
        \node[draw=none] at (0.28,0.44) {\color{black} $\theta_c$};
        \node[draw=none] at (0.5,1.1) {GCC not fulfilled};
        \begin{scope}[xshift=1.25cm]
            \draw (0,0) -- (1,0) -- (1,1) -- (0,1) -- cycle;
            \draw[fill=gray!40,draw=none,opacity=0.5] (0,1) -- (0,0.85*\hpoint) -- (0.65,0.85*\hpoint) -- (0.65,1-0.85*\hpoint) -- (0,1-0.85*\hpoint) -- (0,0) -- (1,0) -- (1,1) -- cycle;
            \draw[stealth-stealth] (0.7,1-0.85*\hpoint) -- (0.7,0.85*\hpoint); 
            \node[draw=none] at (0.8,0.5) {\color{black} $h_\omega$};
            \node[draw=none] at (0.85,0.25) {\color{gray} $\omega$};
            \draw[dashed] (0.5,0.5) -- (0,0.5); 
            \draw[-stealth,red,very thick] (0,0.2) -- (0.49,0.5) -- (0,0.8);
            \draw[black,very thick] (0.5,0) -- (0.5,1); 
            \node[draw=none] at (0.55,0.5) {\color{black} $\Gamma$};
            \draw (0.25,0.5) to [bend right=45] (0.25,0.35);
            \node[draw=none] at (0.28,0.44) {\color{black} $\theta_c$};
            \node[draw=none] at (0.5,1.1) {GCC fulfilled};
        \end{scope}
    \end{tikzpicture}
    \caption{According to Snell's law, rays hitting the interface with an angle greater than the critical angle $\theta_c := \arcsin(\frac{c_1}{c_2})$ are reflected back into the domain $\Omega_1$. To ensure that the GCC is fulfilled, the height $h_{\omega}$ has to chosen smaller than $\tan(\theta_c)$.}
    \label{fig:2d:SnellsLaw}
\end{figure}


\begin{comment}
\appendix
\section*{Outtakes}
{\color{gray}
\begin{lem}[Continuity of $A$]
    \begin{enumerate}
        \item For $\Uh \in \ProdFullyDiscrSpace{k}{q}$ and $\mathbf{Y} \in ...$ we have that 
        \begin{equation}
            A[(\Uh,\mathbf{Y})] \le C ... 
        \end{equation}
        \item For $\mathbf{U} \vert_{Q^n} \in ...$ for all $n = 0, \dots, N-1$ and $\Yh \in \ProdFullyDiscrSpace{k^\ast}{q^\ast}$ we have that
        \begin{equation}
            A[(\mathbf{U},\Yh)] \le C ...
        \end{equation}
    \end{enumerate}
\end{lem}

We recall the following interpolation result from \cite{BP24}: 

\begin{lem}[Lem. 2.4 of \cite{BP24}]\label{lem:interpolationOperator}
    Assume that $\Delta t = Ch$ and set $(s,m) = (\min\{k,q\}, \max\{k,q\}+3)$.
    Then, there exists an interpolation operator $\Pi_h$ into $\FullyDiscrSpace{k}{q}$ such that for $n = 0, \dots, N-1$ the following estimates hold
    \begin{enumerate}
        \item $\sum_{n = 0}^{N-1} \{ h^{-1} \Vert u - \Pi_h u \Vert_{L^2(Q^n)} + \Vert \nabla(u - \Pi_h u) \Vert_{L^2(Q^n)} \} \le C \Vert u \Vert_{H^1(Q)}$. 
        \item $\sum_{n = 0}^{N-1} \{ h^{-1} \Vert u - \Pi_h u \Vert_{L^2(Q^n)} + \Vert u - \Pi_h u \Vert_{H^1(Q^n)} \} \le C h^{s} \Vert u \Vert_{H^{m-1}(Q)}$.
        \item $\left( \sum_{n = 0}^{N-1} \int_{I_n} \sum_{K \in \mathcal{T}_h} h^2 \Vert u - \Pi_h u \Vert^2_{H^2(K)} \ \dT \right)^{1/2} \le C h^{s} \Vert u \Vert_{H^{m}(Q)}$ for $u \in H^m(Q_n) \cap C^0(I_n,H^2(\Omega))$.
    \end{enumerate}
\end{lem}



\begin{lem}[Similar to Lem. 2.5 \& Lem. 4.2 of \cite{BP24}]
    In addition to the assumptions of Lem. \ref{lem:interpolationOperator}, let $u \in H^m(Q)$ solve \eqref{eq:waveEquation} and set $\mathbf{U} = (u,\partial_t u)$. Then, the following estimates hold true. 
    \begin{enumerate}
        \item For $u \in H^m(Q) \cap C^0([0,2],H^2(\Omega))$ we have that 
        \begin{equation}
            \vert \mathbf{\Pi}_h \mathbf{U} \vert_{S_h} = \vert \mathbf{\Pi}_h \mathbf{U} - \mathbf{U} \vert_{S_h} \le C h^{s} \Vert u \Vert_{H^m(Q)}.
        \end{equation}
        \item For $u \in H^m(Q)$ we have that
        \begin{equation}
            \vert \mathbf{\Pi}_h \mathbf{U} \vert_{\uparrow \downarrow} \le C h^{s} \Vert u \Vert_{H^{m-1}(Q)}.
        \end{equation}
        \item For $\Uh \in \ProdFullyDiscrSpace{k}{q}$ it holds that 
        \begin{equation}
            \Vert u - \ul_1 \Vert_{\Sigma} + \Vert u - \ul_1 \Vert_{\omega_T} \le C \left( h^{s+1/2} \Vert u \Vert_{H^m(Q)} + \tnorm{(\Uh - \mathbf{\Pi}_h \Uh,0)} \right).
        \end{equation}
        \item For $\mathbf{Y} = (y_1,y_2) \in [H^1(Q)]^2$, we have that
        \begin{equation}
            \Vert \mathbf{\Pi}^\ast_h \mathbf{Y} \Vert_{S_h^\ast} \le C \left( \Vert y_1 \Vert_{H^1(Q)} + \Vert y_2 \Vert_{H^2(Q)} \right). 
        \end{equation}
    \end{enumerate}
\end{lem}

\begin{proof}
    has to be adapted to the current setting.
\end{proof}}

{\color{red} do we need this one? 
\begin{lem}\label{eq:lemma:stability}
    Let $u$ be a solution of \eqref{eq:waveEquation} and $\mathbf{U} = (u,\partial_t u)$. For any $(\Wh,\Yh) \in \ProdFullyDiscrSpace{k}{q} \times \ProdFullyDiscrSpace{k^\ast}{q^\ast}$ it holds that
    \begin{equation*}
        (u - \Pi_h u, \wl_1)_{\omega_T} - \gamma S_h(\mathbf{\Pi}_h \mathbf{U},\Wh) - A[\mathbf{\Pi}_h \mathbf{U},\Yh] - \Sud(\mathbf{\Pi}_h \mathbf{U}, \Wh) \le C h^s \Vert u \Vert_{H^m(Q)} \tnorm{(\Wh,\Yh)}.
    \end{equation*}
\end{lem}

\begin{proof}
    We follow the proof of \cite[Lem. 2.6]{BP24}. 
\end{proof}}
\end{comment}

%\bibliographystyle{alpha}
\bibliography{references}



\end{document}




 
