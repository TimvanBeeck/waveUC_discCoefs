%3 May 2012
\documentclass[10pt,reqno]{amsart}

\setlength{\topmargin}{0cm}
\setlength{\textheight}{21cm}
\setlength{\oddsidemargin}{0in}
\setlength{\evensidemargin}{0in}
\setlength{\textwidth}{6.5in}
\setlength{\parindent}{.25in}

\pagestyle{plain}
\usepackage{stmaryrd}
\usepackage{amssymb, amsmath, amsthm}
\usepackage{mathtools}
\usepackage{xcolor}
\usepackage{hyperref}
%\usepackage{showkeys}
\usepackage{enumerate}
\usepackage{subfig}
\usepackage{enumitem} 
\usepackage{todonotes}
\usepackage{booktabs}
\usepackage[]{algorithm2e}
\usepackage[capitalize]{cleveref}
\usepackage{placeins}
\crefname{equation}{}{}
\usepackage{verbatim}
\usepackage{bbm}

%\usepackage{graphicx}
%% expansion of width
\textwidth=15.7cm
\textheight=22.5cm
\parskip=3pt
\parindent=8mm
\oddsidemargin=2mm
\evensidemargin=0mm
\topmargin=-0.5cm
\marginparwidth=1cm
%% definition of theorem-type environments
\newtheorem{thm}{Theorem}[section]
\newtheorem{lem}[thm]{Lemma}
\newtheorem{cor}[thm]{Corollary}
\newtheorem{prop}[thm]{Proposition}
\newtheorem{defn}[thm]{Definition}
\newtheorem{rmk}{Remark}
\newtheorem{exa}{Example}
\newtheorem{assum}{Assumption}
\numberwithin{equation}{section}
\newcommand{\bel}{\begin{equation} \label}
\newcommand{\ee}{\end{equation}}
\def\beq{\begin{equation}}
\def\eeq{\end{equation}}
\newcommand{\jump}[1]{\llbracket#1\rrbracket}
\newcommand{\bea}{\begin{eqnarray}}
\newcommand{\eea}{\end{eqnarray}}
\newcommand{\beas}{\begin{eqnarray*}}
\newcommand{\eeas}{\end{eqnarray*}}
\newcommand{\pd}{\partial}
\newcommand{\mdiv}[1]{\ensuremath{\mathrm{div} \left( #1 \right)}}
\newcommand{\dd}{\mbox{d}}

\newcommand{\ep}{\varepsilon}
\newcommand{\la}{\lambda}
\newcommand{\va}{\varphi}
\newcommand{\ppp}{\partial}
\newcommand{\chch}{\chi_{\eta}}
\newcommand{\walpha}{\widetilde{\alpha}}
\newcommand{\wbeta}{\widetilde{\beta}}

\newcommand{\re}{\mathfrak R}

\newcommand{\im}{\mathfrak I}
\newcommand{\pdif}[2]{\frac{\partial #1}{\partial #2}}
\newcommand{\ppdif}[2]{\frac{\partial^2 #1}{{\partial #2}^2}}
\newcommand{\R}{\mathbb{R}}
\newcommand{\C}{\mathbb{C}} 
\newcommand{\N}{\mathbb{N}} 
\newcommand{\ooo}{\overline}
\newcommand{\uu}{\mathbf{u}}
\renewcommand{\v}{\mathbf{v}}
\newcommand{\y}{\mathbf{y}}
\newcommand{\RR}{\mathbf{R}}
\newcommand{\Y}{\mathbf{Y}}
\newcommand{\w}{\mathbf{w}}
\newcommand{\z}{\mathbf{z}}
\newcommand{\G}{\mathbf{G}}
\newcommand{\cB}{\mathcal{B}}
\newcommand{\cD}{\mathcal{D}}
\newcommand{\cL}{\mathcal{L}}
\newcommand{\cO}{\mathcal{O}}
\newcommand{\Hin}{\mathcal{H}_{\mathrm{in},T_0}}
\newcommand{\Gi}{S_{\mathrm{in}}}
\newcommand{\Go}{S_{\mathrm{out}}}
\newcommand{\dom}{\mathrm{Dom}}

\newcommand{\sdd}{\mathcal{E}}


\renewcommand{\baselinestretch}{1.5}
%
\renewcommand{\div}{\mathrm{div}\,}  %div
\newcommand{\grad}{\mathrm{grad}\,}  %grad
\newcommand{\rot}{\mathrm{rot}\,}  %rot

\newcommand{\supp}{\mathrm{supp}\,}  %supp
%\newcommand{\span}{\mathrm{span}\,} %span

\allowdisplaybreaks
%%  item
\renewcommand{\theenumi}{\arabic{enumi}}
\renewcommand{\labelenumi}{(\theenumi)}
\renewcommand{\theenumii}{\alph{enumii}}
\renewcommand{\labelenumii}{(\theenumii)}
\def\epsilon{\varepsilon}
%\def\phi {\varphi}
\def \la {{\lambda}}
\def \a {{\alpha}}
\def\t{\theta}
\def\fh{\frac{1}{h}}


\DeclareMathOperator{\dis}{dist}


\newcommand{\wop}{\square_c}

\newcommand{\tnorm}[1]{\vert\hspace{-0.3mm}\Vert#1\Vert\hspace{-0.3mm}\vert}

\providecommand{\abs}[1]{\left\lvert#1\right\rvert}
% pour les normes
\providecommand{\norm}[1]{\left\lVert#1\right\rVert}

\renewcommand{\leq}{\leqslant}
\renewcommand{\geq}{\geqslant}
\providecommand{\abs}[1]{\left\lvert#1\right\rvert}
% pour les normes
\providecommand{\norm}[1]{\left\lVert#1\right\rVert}
\def\thefootnote{{}}



\newcommand{\HOX}[1]{\marginpar{\footnotesize #1}}

\newcommand{\gammaGLS}{\gamma_{\text{GLS}}}
\newcommand{\gammaCIP}{\gamma}

\newcommand{\dT}{\mathrm{d}t}
\newcommand{\dX}{\mathrm{d}x}
\newcommand{\dS}{\mathrm{d}S}

\newcommand{\STdom}{Q}

\newcommand{\STdata}{\omega_T}
\newcommand{\STdataDisc}{\underline{\omega}_T}
\newcommand{\SemiDiscSpace}{\mathcal{W}_h}
\newcommand{\FullyDiscSpace}{W}

%\newcommand{\FullyDiscrSpaceDisc}[2]{ W_{ {#1},{#2}}^{ \text{dc} } }
%\newcommand{\FullyDiscrSpaceCont}[2]{ W_{ {#1},{#2}}^{ \text{c}  } }
%\newcommand{\ProdFullyDiscrSpaceDisc}[2]{  \mathcal{W}_{ {#1},{#2} }^{ \text{dc} } }
%\newcommand{\ProdFullyDiscrSpaceCont}[1]{  \mathcal{W}_{ {#1} }^{ \text{c} } }

\newcommand{\SemiDiscrSpace}[1]{ W^{ {#1}}_{h} }
\newcommand{\ProdSemiDiscrSpace}[1]{ \mathcal{W}^{ {#1} }_{h} }
\newcommand{\FullyDiscrSpace}[2]{ W^{ {#1},{#2}}_{h, \Delta t  } }
\newcommand{\FullyDiscrSpaceHat}[2]{ \hat{W}^{ {#1},{#2}}_{h, \Delta t  } }
\newcommand{\ProdFullyDiscrSpace}[2]{ \mathcal{W}^{ {#1},{#2}}_{h, \Delta t  } }

\DeclarePairedDelimiterX{\inp}[2]{(}{)}{#1, #2}
\newcommand{\tangular}[1]{ \llbracket\kern-0.5ex|#1|\kern-0.5ex\rrbracket} 
%\newcommand{\jump}[1]{\llbracket#1\rrbracket}
\newcommand{\avg}[1]{ \{\!\!\{#1\}\!\!\}}

\newboolean{includeextras}
\ifdefined\withextras
\setboolean{includeextras}{true}
\else
\setboolean{includeextras}{false}
\fi

\newcommand{\putextra}[1]{\ifthenelse{\boolean{includeextras}}{#1}{}}

\newcommand{\Uh}{\underline{\mathbf{U}}_h}
\newcommand{\Yh}{\underline{\mathbf{Y}}_h}
\newcommand{\Zh}{\underline{\mathbf{Z}}_h}
\newcommand{\Wh}{\underline{\mathbf{W}}_h}

\newcommand{\ul}{\underline{u}}
\newcommand{\yl}{\underline{y}}
\newcommand{\zl}{\underline{z}}
\newcommand{\wl}{\underline{w}}

\newcommand{\Sud}{S^{\uparrow \downarrow}_{\Delta t}}


\newcommand{\dt}{\partial_t}
\newcommand{\dtt}{\partial_t^2}

%%% TvB: Added for plots, some might be redundant ##

%------------------------- Added for Plots ----
% Tikz package
\usepackage{tikz}
\usepackage{pgfplots,pgfplotstable}
\usepgfplotslibrary{colorbrewer,groupplots}
\usepackage{caption}
\pgfplotsset{compat=1.18}

\pgfplotsset{
% initialize Set1-5:
cycle list/Set1-5,
% combine it with ’mark list*’:
cycle multiindex* list={
mark list*\nextlist
Set1-5\nextlist
},
}

\pgfplotsset{
    discard if not/.style 2 args={
        x filter/.append code={
            \edef\tempa{\thisrow{#1}}
            \edef\tempb{#2}
            \ifx\tempa\tempb
            \else
                \def\pgfmathresult{inf}
            \fi
        }
    }
}

%%%


\title{Data assimilation for the wave equation with discontinuous coefficients}
\author{Erik Burman, Janosch Preuss and Tim van Beeck}

\date{\today}
\begin{document}
\maketitle

\section{Introduction}
\noindent Let $\Omega \subset \mathbb{R}^d$ be a bounded domain such that there exists subdomains $\Omega_1, \Omega_2 \subseteq \Omega$ such that $\Omega = \overline{\Omega}_1 \cup \overline{\Omega}_2$ and $\Omega_1 \cap \Omega_2 = \emptyset$. We consider a scalar wave speed $c(x) = \mathbbm{1}_{\Omega_1} c_1(x) + \mathbbm{1}_{\Omega_2} c_2(x)$, $c_i \in C^\infty(\Omega_i)$, $i = 1,2$, which may be discontinuous across the interface $\Gamma = \Omega \setminus (\Omega_1 \cup \Omega_2)$, but constant in time.  
\noindent In this article, we consider the following problem: find $u \in Q \rightarrow \mathbb{R}$ such that 
\begin{subequations}
    \begin{alignat}{2}
        (\wop u, u) &= (0,0) \quad &&\text{ in } Q \times \Sigma, \\
        u &= u_{\omega} &&\text{ in } \omega_T,
    \end{alignat}    
\end{subequations}
where $\wop u := \partial_t^2 - \div(c(x)^2 \nabla u)$ is the wave operator and $u_{\omega}$ is the known data. 

\noindent The continuous problem has been studied in \cite{Filippas22}. In particular, the following result has been shown:
\begin{thm}[Thm. 1.1 of \cite{Filippas22}]\label{thm:filippas}
    For any nonempty subset $\omega \subset \Omega$ and $T > \text{dist}(\Omega,\omega)$, there exists $C,\kappa,\mu_0$ such that for any initial data $(u_0,u_1) \in H^1_0(\Omega) \times L^2(\Omega)$ and $u$ solving... one has for any $\mu \ge \mu_0$ that 
    \begin{equation}
        \Vert (u_0,u_1) \Vert_{L^2 \times H^{-1}} \le C e^{\kappa \mu} \Vert u \Vert_{L^2(\omega_T)} + \frac{C}{\mu} \Vert (u_0,u_1) \Vert_{H^1 \times L^2}. 
    \end{equation}
\end{thm}

\section{Discretization (fully discrete \& dG in time)} 
\noindent We introduce a similar scheme as considered in \cite{BP24}, but we incoporate the wave speed $c$. \textbf{spaces, etc..} \\
We introduce the bilinear form 
\begin{equation}
    A[\Uh,\Yh] := \sum_{n = 0}^{N -1} \left\{ (\dt \ul_2, \yl_1)_{Q^n} + a(\ul_1,\yl_1)_{Q^n} + (\dt \ul_1 - \ul_2,\yl_2)_{Q^n} - (c(x)^2 \nabla \ul_1 \cdot \mathbf{n}, \yl_1)_{\Sigma^n} \right\}
\end{equation} 
and the Lagrangian 
\begin{align*}
    \cL_h (\Uh, \Zh) := \frac{1}{2} \Vert \ul_1 - u_{\omega} \Vert^2_{\omega_T} + A[\Uh,\Zh] + \frac{1}{2} \gamma S_h(\Uh,\Uh) - \frac{1}{2} \gamma^\ast S_h^\ast(\Zh,\Zh) + \frac{\gamma}{2} \Sud(\Uh,\Uh).
\end{align*}
To define the stabilization terms $S_h$ and $S_h^\ast$, we introduce the following terms: 
\begin{align*}
    J(\Uh,\Wh) &:= \sum_{n = 0}^{N -1} \int_{I_n} \sum_{F \in \mathcal{F}_i} h (\jump{c(x)^2 \nabla \ul_1}, \jump{c(x)^2 \nabla \wl_1})_F \ \dT, \\
    I_0(\Uh,\Wh) &:= \sum_{n = 0}^{N -1} (\ul_2 - \dt \ul_1, \wl_2-\dt \wl_1)_{Q_n}, \\
    G(\Uh,\Wh) &:= \sum_{n = 0}^{N -1} \int_{I_n} \sum_{K \in \mathcal{T}_h} h^2 (\dt \ul_2 - \div (c(x)^2 \nabla \ul_1),\dt \wl_2 -\div( c(x)^2 \nabla \wl_1))_K \ \dT, \\
    R(\Uh,\Wh) &:= \sum_{n = 0}^{N -1} h^{-1} (\ul_1,\wl_1)_{\Sigma^n}. 
\end{align*}
Then, we define the primal stabilizer as 
\begin{equation}
    S_h(\Uh,\Wh) := J(\Uh,\Wh) + I_0(\Uh,\Wh) + G(\Uh,\Wh) + R(\Uh,\Wh), 
\end{equation}
and the dual stabilizer through 
\begin{equation}
    S_h^\ast(\Yh,\Zh) := \sum_{n = 0}^{N-1} \left\{ (\yl_1,\zl_1)_{Q^n} + a(\yl_1,\zl_1)_{Q^n} + (\yl_2,\zl_2)_{Q^n} + h^{-1} (\yl_1,\zl_1)_{\Sigma^n} \right\}.
\end{equation}
The stabilization term $\Sud$ imposes regularity on the discontinuities in time and is defined as 
\begin{equation}
    \Sud (\Uh,\Wh) := \underline{I}_1^{\uparrow \downarrow}(\Uh,\Wh) + \underline{I}_2^{\uparrow \downarrow}(\Uh,\Wh),
\end{equation}
where 
\begin{align*}
    \underline{I}_1^{\uparrow \downarrow}(\Uh,\Wh) &:= \sum_{n = 0}^{N-1} \left\{ \frac{1}{\Delta t} (\jump{\ul_1^n},\jump{\wl_1^n})_{\Omega} + \Delta t (c(x)^2 \jump{\nabla \ul_1^n},c(x)^2 \jump{\nabla \wl_1^n})_{\Omega}\right\}, \\
    \underline{I}_2^{\uparrow \downarrow}(\Uh,\Wh) &:= \sum_{n = 0}^{N-1} \frac{1}{\Delta t} (\jump{\ul_2^n},\jump{\wl_2^n})_{\Omega}.
\end{align*}

\noindent The first order optimality conditions then take the form: Find $(\Uh,\Zh) \in ...$ such that 
\begin{alignat*}{2}
    (\ul_1,\wl_1)_{\omega_T} + A[\Wh,\Zh] + \gamma S_h(\Uh,\Wh) + \Sud(\Uh,\Wh) &= (u_{\omega},\wl_1)_{\omega_T} \quad &&\forall \Wh \in ... \\
    A[\Uh,\Yh] - \gamma^{\ast} S_h^\ast(\Yh,\Zh) &= 0 \quad &&\forall \Yh \in ... 
\end{alignat*}
For convenience, we introduce
\begin{equation}
    \begin{aligned}
        B[(\Uh,\Zh),(\Wh,\Yh)] := &(\ul_1,\wl_1)_{\omega_T} + A[\Wh,\Zh]+ \gamma S_h(\Uh,\Wh) \\
        &+  \Sud(\Uh,\Wh) + A[\Uh,\Yh] - \gamma^{\ast} S_h^\ast(\Yh,\Zh),
    \end{aligned}
\end{equation}
such that the first order optimality conditions can be written as: Find $(\Uh,\Zh) \in ...$ such that
\begin{equation}
    B[(\Uh,\Zh),(\Wh,\Yh)] = (u_{\omega},\wl_1)_{\omega_T} \quad \forall (\Wh,\Yh) \in ...
\end{equation}


\section{Analysis}
\noindent We define the norms \textbf{has to be shown that they are indeed norms!} 
\begin{equation}
    \tnorm{ (\Uh,\Zh) }^2 := \gamma \vert \Uh \vert^2_{S_h} + \vert \Uh \vert^2_{\uparrow \downarrow} + \Vert \ul_1 \Vert^2_{\omega_T} + \gamma^{\ast} \Vert \Zh \Vert^2_{S_h^\ast},
\end{equation}
where $\vert \cdot \vert_{S_h}, \vert \cdot \vert_{\uparrow \downarrow}$, and $\Vert \cdot \Vert_{S_h^{\ast}}$ are the semi-norms, respectively norms, induced by the stabilizers $S_h$, $\Sud$, and $S_h^{\ast}$:
\begin{equation}
    \vert \Uh \vert^2_{S_h} := S_h(\Uh,\Uh), \quad \vert \Uh \vert^2_{\uparrow \downarrow} := \Sud(\Uh,\Uh), \quad \Vert \Zh \Vert^2_{S_h^{\ast}} := S_h^{\ast}(\Zh,\Zh).
\end{equation}

We modify the steps from \cite{BP24} to show that $\tnorm{(\cdot,\cdot)}$ is indeed a norm. 

\begin{lem}
    The expression $\tnorm{(\cdot,\cdot)}$ is a norm on ... 
\end{lem}

\begin{proof}
    It suffices to show that $\tnorm{(\Uh,\Zh)} = 0$ implies that $\Uh=\Zh= 0$. We first note that $\vert \Uh \vert_{\uparrow \downarrow} = 0$ immediately implies that $(\ul_1,\ul_2) \in [H^1(Q)]^2$. Furthermore, by definition of $\tnorm{(\cdot,\cdot)}$ and definition the stabilization terms $S_h$ and $S_h^{\ast}$, we have that $\Zh = 0$, $\ul_1 \vert_{\Sigma} = 0$, $\ul_1 \vert_{\omega_T} = 0$ and $\dt \ul_1 = \ul_2$. Similar to \cite[Lem. 2.1]{BP24}, we have with partial integration that 
    \begin{align*}
        \Vert \wop \ul_1 \Vert^2_{H^{-1}(Q)} &:= \sup_{\substack{  y \in H^1_0(Q), \\ \norm{y}_{H^1(Q) } = 1  }} \int_{Q} \left\{ -(\dt \ul_1) \dt y + c(x)^2 \nabla \ul_1 \nabla y \right\}  \\
        &= \sup_{\substack{  y \in H^1_0(Q), \\ \norm{y}_{H^1(Q) } = 1  }} \Big\{ \sum_{n = 0}^{N-1} (\ul_2 - \dt \ul_1, \dt y)_{Q_n} + \sum_{n = 0}^{N-1} \int_{I_n} \sum_{K \in \mathcal{T}_h} (\dt \ul_2 - \div(c(x)^2 \nabla \ul_1),y)_{K} \ \dT \\
        &\qquad + \sum_{n = 0}^{N-1} \int_{I_n} \sum_{F \in \mathcal{F}_i} (\jump{c(x)^2 \nabla \ul_1}, y)_F \ \dT + \sum_{n = 0}^{N-1} (\jump{\ul_1^n},y)_{\Sigma^n} \Big\}.
    \end{align*}
    Thus it follows by definition of the stabilizers and $\tnorm{(\Uh,\Zh)} = 0$ that $\Vert \wop \ul_1 \Vert_{H^{-1}(Q)} = 0$. It directly follows from \ref{thm:filippas} that the initial data are zero. {\color{gray} Thats means that $\ul_1$ solves the wave equation with rhs zero, zero initial data, is zero on the boundary and zero on $\omega$... should enough to say that $\ul_1 = 0$ everywhere!?}
\end{proof}

\noindent Since $B[(\Uh,\Zh),(\Uh,-\Zh)] = \tnorm{(\Uh,\Zh)}\tnorm{(\Uh,-\Zh)}$, the inf-sup stability of $B$ on ... is a direct consequence of the previous lemma, i.e. there exists a constant $C>0$ such that 
\begin{equation}
    \sup_{(\Wh,\Yh) \in ...} \frac{B[(\Uh,\Zh),(\Wh,\Yh)]}{\tnorm{(\Wh,\Yh)}} \ge C \tnorm{(\Uh,\Zh)}.
\end{equation}


\section{Numerical experiments}
\subsection{Example in one space dimension}
We consider the example from \cite{BDE22} with domain $\Omega = [0,1]$ that is partitioned into the subdomains $\Omega_1 = (0,0.5)$ and $\Omega_2 = (0.5,1.0)$ such that $\Omega = \overline{\Omega}_1 \cup \overline{\Omega}_2$. Then, we fix the wavespeed $c_2 := c \vert_{\Omega_2} = 1$ and consider different values for $c_1 := c \vert_{\Omega_1}$. With $u_0 (x) := \frac{1}{100} \exp(-(20(x-\frac{1}{5}))^2)$, the exact solution is given by 
\begin{equation}
    u(x,t) = \begin{cases}
      \sum_{k \ge 0} \left( \frac{c_2-c_1}{c_2+c_1}\right)^k \left( u_0 (k + x - c_1 t) 0- u_0(k-x-c_1 t)\right), x \in \Omega_1, \\
      \left( \frac{2c_1}{c_2+c_1}\right) \sum_{k \ge 0} \left( \frac{c_2 - c_1}{c_2 + c_1}\right)^k u_0 \left( \frac{c_1}{c_2} \left( x - \frac{1}{2}\right) +k + \frac{1}{2} - c_1 t \right), x \in \Omega_2. 
    \end{cases}
\end{equation}


\subsection{Example in two space dimensions}

\bibliographystyle{alpha}
\bibliography{references}

\end{document}




 
