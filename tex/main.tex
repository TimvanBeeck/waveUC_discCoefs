%\documentclass[10pt,reqno]{amsart}
\documentclass[sn-mathphys-num]{sn-jnl}

%\setlength{\topmargin}{0cm}
%\setlength{\textheight}{21cm}
%\setlength{\oddsidemargin}{0in}
%\setlength{\evensidemargin}{0in}
%\setlength{\textwidth}{6.5in}
%\setlength{\parindent}{.25in}

%\pagestyle{plain}
\usepackage{stmaryrd}
\usepackage{amssymb, amsmath, amsthm}
\usepackage{mathtools}
\usepackage{xcolor}
\usepackage{hyperref}
%\usepackage{showkeys}
\usepackage{enumerate}
\usepackage{subfig}
\usepackage{enumitem} 
\usepackage{todonotes}
\usepackage{booktabs}
\usepackage[]{algorithm2e}
\usepackage[capitalize]{cleveref}
\usepackage{placeins}
\crefname{equation}{}{}
\usepackage{verbatim}
\usepackage{bbm}

%\usepackage{graphicx}
%% expansion of width
%\textwidth=15.7cm
%\textheight=22.5cm
%\parskip=3pt
%\parindent=8mm
%\oddsidemargin=2mm
%\evensidemargin=0mm
%\topmargin=-0.5cm
%\marginparwidth=1cm


%% definition of theorem-type environments
\newtheorem{thm}{Theorem}[section]
\newtheorem{lem}[thm]{Lemma}
\newtheorem{cor}[thm]{Corollary}
\newtheorem{prop}[thm]{Proposition}
\newtheorem{defn}[thm]{Definition}
\newtheorem{rmk}{Remark}
\newtheorem{exa}{Example}
\newtheorem{assum}{Assumption}
\numberwithin{equation}{section}
\newcommand{\bel}{\begin{equation} \label}
\newcommand{\ee}{\end{equation}}
\def\beq{\begin{equation}}
\def\eeq{\end{equation}}
\newcommand{\jump}[1]{\llbracket#1\rrbracket}
\newcommand{\bea}{\begin{eqnarray}}
\newcommand{\eea}{\end{eqnarray}}
\newcommand{\beas}{\begin{eqnarray*}}
\newcommand{\eeas}{\end{eqnarray*}}
\newcommand{\pd}{\partial}
\newcommand{\mdiv}[1]{\ensuremath{\mathrm{div} \left( #1 \right)}}
\newcommand{\dd}{\mbox{d}}

\newcommand{\ep}{\varepsilon}
\newcommand{\la}{\lambda}
\newcommand{\va}{\varphi}
\newcommand{\ppp}{\partial}
\newcommand{\chch}{\chi_{\eta}}
\newcommand{\walpha}{\widetilde{\alpha}}
\newcommand{\wbeta}{\widetilde{\beta}}

\newcommand{\re}{\mathfrak R}

\newcommand{\im}{\mathfrak I}
\newcommand{\pdif}[2]{\frac{\partial #1}{\partial #2}}
\newcommand{\ppdif}[2]{\frac{\partial^2 #1}{{\partial #2}^2}}
\newcommand{\R}{\mathbb{R}}
\newcommand{\C}{\mathbb{C}} 
\newcommand{\N}{\mathbb{N}} 
\newcommand{\ooo}{\overline}
\newcommand{\uu}{\mathbf{u}}
\renewcommand{\v}{\mathbf{v}}
\newcommand{\y}{\mathbf{y}}
\newcommand{\RR}{\mathbf{R}}
\newcommand{\Y}{\mathbf{Y}}
\newcommand{\w}{\mathbf{w}}
\newcommand{\z}{\mathbf{z}}
\newcommand{\G}{\mathbf{G}}
\newcommand{\cB}{\mathcal{B}}
\newcommand{\cD}{\mathcal{D}}
\newcommand{\cL}{\mathcal{L}}
\newcommand{\cO}{\mathcal{O}}
\newcommand{\Hin}{\mathcal{H}_{\mathrm{in},T_0}}
\newcommand{\Gi}{S_{\mathrm{in}}}
\newcommand{\Go}{S_{\mathrm{out}}}
\newcommand{\dom}{\mathrm{Dom}}

\newcommand{\sdd}{\mathcal{E}}


\renewcommand{\baselinestretch}{1.5}
%
\renewcommand{\div}{\mathrm{div}\,}  %div
\newcommand{\grad}{\mathrm{grad}\,}  %grad
\newcommand{\rot}{\mathrm{rot}\,}  %rot

\newcommand{\supp}{\mathrm{supp}\,}  %supp
%\newcommand{\span}{\mathrm{span}\,} %span

\allowdisplaybreaks
%%  item
\renewcommand{\theenumi}{\arabic{enumi}}
\renewcommand{\labelenumi}{(\theenumi)}
\renewcommand{\theenumii}{\alph{enumii}}
\renewcommand{\labelenumii}{(\theenumii)}
\def\epsilon{\varepsilon}
%\def\phi {\varphi}
\def \la {{\lambda}}
\def \a {{\alpha}}
\def\t{\theta}
\def\fh{\frac{1}{h}}


\DeclareMathOperator{\dis}{dist}


\newcommand{\wop}{\square_c}

\newcommand{\tnorm}[1]{\vert\hspace{-0.3mm}\Vert#1\Vert\hspace{-0.3mm}\vert}

\providecommand{\abs}[1]{\left\lvert#1\right\rvert}
% pour les normes
\providecommand{\norm}[1]{\left\lVert#1\right\rVert}

\renewcommand{\leq}{\leqslant}
\renewcommand{\geq}{\geqslant}
\providecommand{\abs}[1]{\left\lvert#1\right\rvert}
% pour les normes
\providecommand{\norm}[1]{\left\lVert#1\right\rVert}
\def\thefootnote{{}}



\newcommand{\HOX}[1]{\marginpar{\footnotesize #1}}

\newcommand{\gammaGLS}{\gamma_{\text{GLS}}}
\newcommand{\gammaCIP}{\gamma}

\newcommand{\dT}{\mathrm{d}t}
\newcommand{\dX}{\mathrm{d}x}
\newcommand{\dS}{\mathrm{d}S}

\newcommand{\STdom}{Q}

\newcommand{\STdata}{\omega_T}
\newcommand{\STdataDisc}{\underline{\omega}_T}
\newcommand{\SemiDiscSpace}{\mathcal{W}_h}
\newcommand{\FullyDiscSpace}{W}

%\newcommand{\FullyDiscrSpaceDisc}[2]{ W_{ {#1},{#2}}^{ \text{dc} } }
%\newcommand{\FullyDiscrSpaceCont}[2]{ W_{ {#1},{#2}}^{ \text{c}  } }
%\newcommand{\ProdFullyDiscrSpaceDisc}[2]{  \mathcal{W}_{ {#1},{#2} }^{ \text{dc} } }
%\newcommand{\ProdFullyDiscrSpaceCont}[1]{  \mathcal{W}_{ {#1} }^{ \text{c} } }

\newcommand{\SemiDiscrSpace}[1]{ W^{ {#1}}_{h} }
\newcommand{\ProdSemiDiscrSpace}[1]{ \mathcal{W}^{ {#1} }_{h} }
\newcommand{\FullyDiscrSpace}[2]{ W^{ {#1},{#2}}_{h, \Delta t  } }
\newcommand{\FullyDiscrSpaceHat}[2]{ \hat{W}^{ {#1},{#2}}_{h, \Delta t  } }
\newcommand{\ProdFullyDiscrSpace}[2]{ \mathcal{W}^{ {#1},{#2}}_{h, \Delta t  } }

\DeclarePairedDelimiterX{\inp}[2]{(}{)}{#1, #2}
\newcommand{\tangular}[1]{ \llbracket\kern-0.5ex|#1|\kern-0.5ex\rrbracket} 
%\newcommand{\jump}[1]{\llbracket#1\rrbracket}
\newcommand{\avg}[1]{ \{\!\!\{#1\}\!\!\}}

\newboolean{includeextras}
\ifdefined\withextras
\setboolean{includeextras}{true}
\else
\setboolean{includeextras}{false}
\fi

\newcommand{\putextra}[1]{\ifthenelse{\boolean{includeextras}}{#1}{}}

\newcommand{\Uh}{\underline{\mathbf{U}}_h}
\newcommand{\Vh}{\underline{\mathbf{V}}_h}
\newcommand{\Yh}{\underline{\mathbf{Y}}_h}
\newcommand{\Zh}{\underline{\mathbf{Z}}_h}
\newcommand{\Wh}{\underline{\mathbf{W}}_h}

\newcommand{\ul}{\underline{u}}
\newcommand{\yl}{\underline{y}}
\newcommand{\zl}{\underline{z}}
\newcommand{\wl}{\underline{w}}

\newcommand{\Sud}{S^{\uparrow \downarrow}_{\Delta t}}


\newcommand{\dt}{\partial_t}
\newcommand{\dtt}{\partial_t^2}

%%% TvB: Added for plots, some might be redundant ##

%------------------------- Added for Plots ----
% Tikz package
\usepackage{tikz}
\usepackage{pgfplots,pgfplotstable}
\usepgfplotslibrary{colorbrewer,groupplots}
\usepackage{caption}
\pgfplotsset{compat=1.18}


%find colors at https://colorbrewer2.org/#type=qualitative&scheme=Set1&n=5
\pgfplotsset{
% initialize Set1-5:
cycle list/Set1-5,
% combine it with ’mark list*’:
cycle multiindex* list={
mark list*\nextlist
Set1-5\nextlist
},
}

\pgfplotsset{
    discard if not/.style 2 args={
        x filter/.append code={
            \edef\tempa{\thisrow{#1}}
            \edef\tempb{#2}
            \ifx\tempa\tempb
            \else
                \def\pgfmathresult{inf}
            \fi
        }
    }
}

%%%

\providecommand{\jp}[1]{\newline {\color{cyan} \hspace*{-0.5cm} $\rightsquigarrow$ \underline{\textbf{JP:}} \textit{#1} \newline}}

\date{\today}
\begin{document}

\title{A numerical study of variational data assimilation for the wave equation in heterogeneous media}

\author[1]{\fnm{Erik} \sur{Burman}}\email{e.burman@ucl.ac.uk}
\author[2]{\fnm{Janosch} \sur{Preuss}}\email{janosch.preuss@inria.fr}
\author[3]{\fnm{Tim} \sur{van Beeck}}\email{t.beeck@math.uni-goettingen.de}

\affil[1]{\orgdiv{Department of Mathematics}, \orgname{University College London}, \orgaddress{\street{Gower Street}, \city{London}, \postcode{WC1E 6BT}, \country{United Kingdom}}}

\affil[2]{\orgdiv{MAKUTU}, \orgname{Inria}, \orgaddress{\street{...}, \city{...}, \postcode{...}, \country{France}}}

\affil[3]{\orgdiv{Institute for Numerical and Applied Mathematics}, \orgname{University of G\"{o}ttingen}, \orgaddress{\street{Lotzestr. 16-18}, \city{G\"{o}ttingen}, \postcode{37083}, \country{Germany}}}

\abstract{In recent years, several numerical methods for solving the unique continuation problem for the wave equation in a homogeneous medium with given data on the lateral boundary of the space-time cylinder have been proposed. This problem enjoys Lipschitz stability if the geometric control condition is fulfilled, which allows devising optimally convergent numerical methods. In this article, we investigate whether these results carry over to the case in which the medium exhibits a jump discontinuity. Our numerical experiments suggest a positive answer. However, we also observe that the presence of discontinuities in the medium renders the computations far more demanding than in the homogeneous case.
}

\keywords{Unique continuation, data assimilation, wave equation, discontinuous coefficients, geometric control condition}

\maketitle

\section{Introduction}
\todo[inline]{@JP: please update affiliation}
\todo[inline]{@JP: Let's discuss the title again, I feel like we should include 'stability' somewhere, because it's our main objective...}
We consider the discretization of a data assimilation problem for the wave equation in heterogeneous media. In the following, let $\Omega \subset \mathbb{R}^d$ be a bounded domain with smooth boundary $\partial \Omega$. We assume that $\Omega$ is split by a smooth $(d-1)$-dimensional interface $\Gamma$ into two disjoint components such that $\Omega \setminus \Gamma = \Omega_1 \cup 
\Omega_2$ and $\Omega_1 \cap \Omega_2 = \emptyset$.
 
We consider a scalar wave speed
\[
 c(x) := \begin{cases}
       c_1(x),  & x \in \Omega_1, \\
        c_2(x), & x \in \Omega_2,
    \end{cases}
    \quad 
    \text{ with } c_i \in C^\infty(\Omega_i), \; i = 1,2,
\]
 %$c(x) = \mathbbm{1}_{\Omega_1} c_1(x) + \mathbbm{1}_{\Omega_2} c_2(x)$, $c_i \in C^\infty(\Omega_i)$, $i = 1,2$,
which may be discontinuous across the interface $\Gamma$, but is constant in time. For a final time $T > 0$, we consider the space-time domain $Q = (0,T) \times \Omega$ with lateral boundary $\Sigma \coloneqq (0,T) \times \partial \Omega$, and for any nonempty subset $\omega \subset \Omega$, we set $\omega_T \coloneqq (0,T) \times \omega$. 

Let $\wop$ be the wave operator with discontinuous wave-speed defined as 
\begin{equation*}
    \wop u := \partial_t^2 u - \div(c^2 \nabla u).
\end{equation*}
In particular, we write $\square_1$ if $c_i = 1$ for $i = 1,2$, corresponding to the case of a homogeneous medium with unit wave speed.

Then, we consider the problem: find $u : Q \rightarrow \mathbb{R}$ such that 
\begin{equation}\label{eq:waveEquation}
        (\wop u, u) = (0,0) \quad \text{ in } Q \times \Sigma,
\end{equation}
along with the natural interface conditions 
\begin{equation}\label{eq:IF-conditions}
	\jump{u}_{\Gamma} = 0 \text{ on } \Sigma, \qquad 
	\quad \jump{c^2 \nabla u}_{\Gamma} \cdot \mathbf{n}_{\Gamma} = 0 \text{ on } \Sigma, 
\end{equation}
where $\mathbf{n}_{\Gamma}$ is the normal vector on $\Gamma$ pointing into $\Omega_+$ 
and the jump across $\Gamma$ is defined\textsuperscript{1}\footnotemark \footnotetext{\textsuperscript{1}And analogously for vector-valued functions.} as $\jump{u}_{\Gamma} = u_{+}-u_{-}$ with $u_{\pm}$ being the traces 
of $u|_{\Omega_{\pm}}$ on $\Gamma$. \par 
To ensure that the problem is well posed, one usually prescribes initial data
\begin{equation}\label{eq:initialData}
    (u,\dt u) \vert_{t = 0} = (u_0,u_1) \text{ in } \Omega, \tag{IVP} 
\end{equation}
where $u_0 \in H^1(Q)$ and $u_1 \in L^2(Q)$ are given. For a detailed analysis of the well-posedness of the problem \eqref{eq:waveEquation}-\eqref{eq:IF-conditions}+\eqref{eq:initialData} in heterogeneous media, we refer to \cite{StolkPhD}. 

In this work, we consider a \emph{data assimilation} problem instead, where the initial data is unknown, and we assume that there are given measurements $u_{\omega} \in L^2(\omega_T)$ prescribed in a data domain $\omega_T$, with $\omega \subset \Omega$, $\omega \not = \emptyset$:
\begin{equation}\label{eq:dataMatch}
    u = u_{\omega} \text{ in } \omega_T. \tag{DA}
\end{equation}
In a medium without an interface, the problem \eqref{eq:waveEquation}+\eqref{eq:dataMatch} is well-studied; see for example \cite{CM15,BFO20,BFMO21,MM21,DMS23,BDE24}. In particular, if the boundary $\partial \Omega$ is strictly convex, the problem is Lipschitz stable provided that the set $\omega_T \subset Q$ fulfills the geometric control condition (GCC) \cite{BLR92}. 
Roughly speaking, a set $\omega \subset \Omega$ and a final time $T$ fulfill the GCC if all geometric optic rays traveling with unit speed\textsuperscript{2}\footnotemark \footnotetext{\textsuperscript{2}We consider the metric to be adapted to the wave speed $c$.} in $\Omega$, accounting for possible reflections on the boundary $\partial \Omega$, intersect the set $(0,T) \times \omega$. For a more precise definition, we refer to \cite{BLR92}. Assuming that the GCC is satisfied, we have the following result.

\begin{thm}\label{thm:Lipschitz}
    Assume that $\partial \Omega$ is strictly convex and $c_i = 1$ for $i = 1,2$. If $\omega_T \subset Q$ fulfills the GCC, then there exists a constant $C > 0$ such that for any $\phi \in H^1(Q)$, we have the following estimates:
    \begin{align*}
        \Vert \phi \Vert_{L^\infty(0,T;L^2(\Omega))} + \Vert \dt \phi \Vert_{L^2(0,T;H^{-1}(\Omega))} &\le C \left(  \Vert \phi \Vert_{L^2(\omega_T)} + \Vert \phi \Vert_{L^2(\Sigma)} + \Vert \square_1 \phi \Vert_{H^{-1}(Q)} \right), \\
        \Vert \phi \vert_{t = 0} \Vert_{L^2(\Omega)} + \Vert \dt \phi \vert_{t = 0} \Vert_{H^{-1}(\Omega)} &\le C \left(\Vert \phi \Vert_{L^2(\omega_T)} + \Vert \phi \Vert_{L^2(\Sigma)} + \Vert \square_1 \phi \Vert_{H^{-1}(Q)} \right). 
    \end{align*}
\end{thm}

\begin{proof}
    See \cite[Thm. A.4]{BFMO21control} and \cite[Rem. A.5]{BFMO21control}.
\end{proof}

In the heterogeneous case, where $c_1 \not =  c_2$, the problem is less studied. In the absence of the GCC, we have to assume that the final time $T$ is large enough to ensure that problem \eqref{eq:waveEquation}-\eqref{eq:IF-conditions}+\eqref{eq:dataMatch} is well-posed. To be precise, Holmgren's unique continuation theorem guarantees that \eqref{eq:waveEquation}-\eqref{eq:IF-conditions}+\eqref{eq:dataMatch} is uniquely solvable if
\begin{equation}\label{eq:Tcondition}
    T > 2 \sup_{x \in \Omega} \operatorname{dist}_{c}(x,\omega_T),
\end{equation}
where $\operatorname{dist}_c(x,\omega_T) := \inf_{y \in \omega_T} d_c(x,y)$. Here, $d_c$ is the distance function defined as the length of the shortest continuous path in $\Omega$ connecting $x$ and $y$ with respect to the metric adapted to the discontinuous wave speed $c$. While the condition \eqref{eq:Tcondition} guarantees unique solvability, the stability of problem \eqref{eq:waveEquation}-\eqref{eq:IF-conditions}+\eqref{eq:dataMatch} can be poor. However, as shown \cite[Thm. 5.19]{Filippas22}, assumption \eqref{eq:Tcondition} grants the following stability result holds. 

\begin{thm}[Thm. 5.19 of \cite{Filippas22}]\label{thm:filippas}
    Let $\omega_T \subset \Omega$ be nonempty and let $T > 0$ be s.t. \eqref{eq:Tcondition} holds. Then, there exists $C,\kappa$ such that for any initial data $(u_0,u_1) \in H^1_0(\Omega) \times L^2(\Omega)$ and $u$ solving \eqref{eq:IF-conditions}+\eqref{eq:initialData} and  
\[
\wop u \in L^2(Q), \quad u = 0 \text{ on } \Sigma,
\]
one has for any $\mu > 0$ that 
    \begin{equation}\label{eq:FilippasEstimate}
        \Vert (u_0,u_1) \Vert_{L^2 \times H^{-1}} \le C e^{\kappa \mu} \left( \Vert u \Vert_{L^2(\omega_T)} + \Vert \wop u \Vert_{L^2(Q)} \right) + \frac{C}{\mu} \Vert (u_0,u_1) \Vert_{H^1 \times L^2}. 
    \end{equation}
    In particular, if $(u_0,u_1) \not = (0,0)$ we have that 
    \begin{equation}\label{eq:FilippasEstimateLog}
        \Vert (u_0,u_1) \Vert_{L^2 \times H^{-1}} \le C \frac{\Vert (u_0,u_1) \Vert_{H^1 \times L^2}}{\log \left( 1 + \frac{\Vert (u_0,u_1) \Vert_{H^1 \times L^2}}{ \Vert u \Vert_{L^2(\omega_T)} + \Vert \wop u \Vert_{L^2(Q)}} \right)}. 
    \end{equation} 
\end{thm} 

\todo[inline]{@JP: I made some minor changes to the introduction... Here, I felt that we should manage the reader's expectation a bit more and stress that we do not focus on a complete analysis. }

In this article, we consider a discontinuous-in-time finite element method to solve the data assimilation problem \eqref{eq:waveEquation}-\eqref{eq:IF-conditions}+\eqref{eq:dataMatch} in the heterogeneous case. This method was originally introduced for the homogeneous case in \cite{BP24}. The focus of this article is not on the numerical analysis of the proposed discretization, but rather on investigating the numerically observed stability under the assumption of the GCC. Specifically, we want to investigate numerically whether the following assumption -- motivated by \cref{thm:Lipschitz} for the homogeneous case -- is justified. 

\begin{assum}\label{assum:LipschitzStability}
    If $\omega_T \subset Q$ fulfills the GCC, then there exists a constant $C > 0$ such that for any $\phi \in H^1(Q)$ we have the following estimates
    \begin{align*}
        \Vert \phi \Vert_{L^\infty(0,T;L^2(\Omega))} + \Vert \dt \phi \Vert_{L^2(0,T;H^{-1}(\Omega))} &\le C \left(  \Vert \phi \Vert_{L^2(\omega_T)} + \Vert \phi \Vert_{L^2(\Sigma)} + \Vert \square_c \phi \Vert_{H^{-1}(Q)} \right), \\
        \Vert \phi \vert_{t = 0} \Vert_{L^2(\Omega)} + \Vert \dt \phi \vert_{t = 0} \Vert_{H^{-1}(\Omega)} &\le C \left(\Vert \phi \Vert_{L^2(\omega_T)} + \Vert \phi \Vert_{L^2(\Sigma)} + \Vert \square_c \phi \Vert_{H^{-1}(Q)} \right). 
    \end{align*}
\end{assum}

In summary, our goal is to understand whether the Lipschitz stability property is preserved in the presence of strong heterogeneity in the medium, provided that the GCC holds. This question is of high practical importance, as discontinuous media appear in many applications, for example, in seismology. If the corresponding data assimilation problems were indeed subject to the poor logarithmic stability property ensured by \eqref{eq:FilippasEstimateLog}, this would pose an enormous challenge for the design of accurate and reliable numerical methods. 

\vspace*{1cm}
\noindent \textbf{Struture of the article.} In \cref{sec:discretization}, we introduce the method used to discretize \eqref{eq:waveEquation}-\eqref{eq:IF-conditions}+\eqref{eq:dataMatch}. 


\section{Fully discrete discontinuous-in-time discretization}\label{sec:discretization} 
In this section, we present a fully discrete discontinuous-in-time finite element method for discretizing the data assimilation problem \eqref{eq:waveEquation}-\eqref{eq:IF-conditions}+\eqref{eq:dataMatch}, extending the method from \cite{BP24} to the heterogeneous case. \Cref{sec:spaceTimeDiscretization} describes the geometric partition of the space-time domain $Q$ into time slabs, followed by the full discretization in \cref{sec:method}.

\subsection{Discretization of the space-time domain}\label{sec:spaceTimeDiscretization}
Let $\hat{\mathcal{T}}_h$ be a quasi-uniform triangulation with mesh size $h$ such that $\hat{\Omega}:= \cup_{K \in \hat{\mathcal{T}}_h} K $ contains $\Omega$, i.e.\ $\Omega \subset \hat{\Omega}$. Setting $\mathcal{T}_h := \{ K \cap \Omega \mid K \in  \hat{\mathcal{T}}_h \}$ then ensures that $\Omega = \cup_{K \in \mathcal{T}_h} K$ holds true.  
We recall from  \cite[Sec. 4.2]{BFMO21control} that the triangulation $\{ \hat{\mathcal{T}}_h \mid h > 0 \}$ can be constructed such that for $h$ small enough the following continuous trace inequality holds:  For all $v \in [H^1(K)]^d$ and $K \in \mathcal{T}_h$, we have
\begin{equation}\label{eq:traceInequality}
    \Vert v \Vert_{[L^2(\partial K)]^d} \le C \left(h^{-1/2} \Vert v \Vert_{[L^2(K)]^d} + h^{1/2} \Vert \nabla v \Vert_{[L^2(K)]^d} \right).  
\end{equation}
We assume additionally that the triangulation $\mathcal{T}_h$ fits the interface $\Gamma$ and the data domain $\omega$. An extension to allow for unfitted interfaces could be possible by utilizing techniques introduced in \cite{BP25} for a stationary problem, but this is certainly beyond the scope of this article. 

For polynomial degree $k \ge 1$, we define the $H^1$-conforming finite element space 
\begin{equation}
    V_h^k := \{ v \in H^1(\Omega) : v \vert_{K} \in \mathcal{P}^k(K) \ \forall K \in \mathcal{T}_h \},
\end{equation}
where $\mathcal{P}^k(K)$ denotes the space of polynomials of degree at most $k \in \mathbb{N}$ on $K \in \mathcal{T}_h$. Furthermore, we partition the time axis into $N$ subintervals $I_n = (t_n,t_{n+1})$, $n = 0, \dots, N-1$, where $0 = t_0 \le t_1 \le \dots \le t_N = T$. We assume that the intervals are of equal length and denote $\Delta t = \vert t_{n+1} -t_n \vert$. Then, we partition $Q$ and $\Sigma$ into time-slabs 
\begin{equation}
    \begin{aligned}
        Q^n := &I_n \times \Omega, \quad \Sigma^n := I_n \times \Sigma, \quad n = 0, \dots, N-1, \\
        Q = &\bigcup_{n = 0}^{N-1} Q^n, \quad \phantom{:} \Sigma = \bigcup_{n = 0}^{N-1} \Sigma^n.
    \end{aligned}
\end{equation}
In the following, we denote the space-time integrals on the time slabs as 
\begin{equation*}
    (u,v)_{Q^n} := \int_{I_n} \int_{\Omega} uv \ \dX \dT, \quad (u,v)_{\Sigma^n} := \int_{I_n} \int_{\Sigma} uv \ \dS \dT,
\end{equation*}
and define $\Vert v \Vert^2_{Q^n} := (v,v)_{Q^n}$ and $\Vert v \Vert^2_{\Sigma^n} := (v,v)_{\Sigma^n}$. 
Furthermore, we set 
\begin{equation*}
    \omega^n := I_n \times \omega, \quad (u,v)_{\omega^n} := \int_{I_n} \int_{\omega} uv \ \dX \dT, \quad \Vert v \Vert^2_{\omega_T} := \sum_{n = 0}^{N-1} (v,v)_{\omega^n}.
\end{equation*}
Finally, we define the time jump operator 
\begin{equation*}
    v^n_{\pm} (x) := \lim_{s \rightarrow 0^+} v(x,t_n \pm s), \quad \jump{v^n} := v^n_+ - v^n_-. 
\end{equation*}

\subsection{Full discretization}\label{sec:method}
 We define the discontinuous in time finite element spaces 
\begin{equation}
    \FullyDiscrSpace{k}{q} := \otimes_{n = 0}^{N-1} \mathcal{P}^q(I_n) \otimes V_h^k, \quad \ProdFullyDiscrSpace{k}{q} :=  \FullyDiscrSpace{k}{q} \times \FullyDiscrSpace{k}{q}, \quad q \in \mathbb{N}_0, k \in \mathbb{N}. 
\end{equation}
For elements of $\ProdFullyDiscrSpace{k}{q}$, we use the notation $\Uh = (\ul_1,\ul_2) \in \ProdFullyDiscrSpace{k}{q}$. In the following, we denote
\begin{equation*}
    a(u,v)_{Q^n} := \int_{I_n} \int_{\Omega} c^2 \nabla u \cdot \nabla v \ \dX \dT,
\end{equation*}
and introduce a bilinear form $A$ that represents the wave-equation \eqref{eq:waveEquation} in mixed formulation:  
\begin{equation}
    \begin{aligned}
        A[\Uh,\Yh] := \sum_{n = 0}^{N -1} \Big\{ &(\dt \ul_2, \yl_1)_{Q^n} + a(\ul_1,\yl_1)_{Q^n} + (\dt \ul_1 - \ul_2,\yl_2)_{Q^n} \\
        &- (c^2 \nabla \ul_1 \cdot \mathbf{n}, \yl_1)_{\Sigma^n} \Big\},
    \end{aligned}
\end{equation} 
where $\Uh \in \ProdFullyDiscrSpace{k}{q}$ and $\Yh \in \ProdFullyDiscrSpace{k^\ast}{q^\ast}$ with $k, k^\ast, q \in \mathbb{N}$ and $q^\ast \in \mathbb{N}_0$. 
To approximate solutions of \eqref{eq:waveEquation}-\eqref{eq:IF-conditions}+\eqref{eq:dataMatch}, we search for stationary points of the Lagrangian 
\begin{align*}
    \cL_h (\Uh, \Zh) := &\frac{1}{2} \Vert \ul_1 - u_{\omega} \Vert^2_{\omega_T} + A[\Uh,\Zh] + \frac{1}{2}  S_h(\Uh,\Uh) \\
    &- \frac{1}{2}  S_h^\ast(\Zh,\Zh) + \frac{1}{2} \Sud(\Uh,\Uh),
\end{align*}
where $S_h$, $S_h^\ast$, and $\Sud$ are stabilization terms yet to be defined. Note that the first and the second terms of $\cL_h$ incorporate the data and the PDE constraints, respectively. To define the stabilization terms $S_h$ and $S_h^\ast$, we introduce the following terms: 
\begin{align*}
    J(\Uh,\Wh) &:= \sum_{n = 0}^{N -1} \int_{I_n} \sum_{F \in \mathcal{F}_i} h (\jump{c^2 \nabla \ul_1}, \jump{c^2 \nabla \wl_1})_F \ \dT, \\
    G(\Uh,\Wh) &:= \sum_{n = 0}^{N -1} \int_{I_n} \sum_{K \in \mathcal{T}_h} h^2 (\dt \ul_2 - \div (c^2 \nabla \ul_1),\dt \wl_2 -\div( c^2 \nabla \wl_1))_K \ \dT, \\
    I_0(\Uh,\Wh) &:= \sum_{n = 0}^{N -1} (\ul_2 - \dt \ul_1, \wl_2-\dt \wl_1)_{Q_n}, \qquad 
    R(\Uh,\Wh) := \sum_{n = 0}^{N -1} h^{-1} (\ul_1,\wl_1)_{\Sigma^n}. 
\end{align*}
~\jp{I think that in the CIP we have to switch from the full gradient to the normal derivative  $\jump{c^2 \nabla \ul_1} \cdot \mathbf{n}$ only. Indeed, it follows from the first condition in \eqref{eq:IF-conditions} that the tangential part of the gradient on $\Gamma$ must be continuous $\jump{\nabla_{\Gamma} u} = 0$. So stabbilizing the full gradient is not consistent, which becomes an issue in \cref{thm:bestapprox} as $S_h(\mathbf{U},\Wh) = 0$ would not hold.
}
$J(\cdot,\cdot)$ is a continuous interior penalty term in space, $G(\cdot,\cdot)$ is a Galerkin least squares term enforcing the PDE locally on each element, $I_0(\cdot,\cdot)$ enforces that $\ul_2 = \dt \ul_1$, and $R(\cdot,\cdot)$ ensures boundary stability. 
Then we define the primal stabilizer $S_h$ as 
\begin{equation}\label{eq:primalStab}
    S_h(\Uh,\Wh) := J(\Uh,\Wh) + I_0(\Uh,\Wh) + G(\Uh,\Wh) + R(\Uh,\Wh), 
\end{equation}
and the dual stabilizer $S_h^\ast$ through 
\begin{equation}
    S_h^\ast(\Yh,\Zh) := \sum_{n = 0}^{N-1} \left\{ (\yl_1,\zl_1)_{Q^n} + a(\yl_1,\zl_1)_{Q^n} + (\yl_2,\zl_2)_{Q^n} + h^{-1} (\yl_1,\zl_1)_{\Sigma^n} \right\}.
\end{equation}
The remaining stabilization term $\Sud$ imposes regularity on the discontinuities in time and is defined as 
\begin{equation}
    \Sud (\Uh,\Wh) := \underline{I}_1^{\uparrow \downarrow}(\Uh,\Wh) + \underline{I}_2^{\uparrow \downarrow}(\Uh,\Wh),
\end{equation}
where 
\begin{align*}
    \underline{I}_1^{\uparrow \downarrow}(\Uh,\Wh) &:= \sum_{n = 0}^{N-1} \left\{ \frac{1}{\Delta t} (\jump{\ul_1^n},\jump{\wl_1^n})_{\Omega} + \Delta t (c^2 \jump{\nabla \ul_1^n},c^2 \jump{\nabla \wl_1^n})_{\Omega}\right\}, \\
    \underline{I}_2^{\uparrow \downarrow}(\Uh,\Wh) &:= \sum_{n = 0}^{N-1} \frac{1}{\Delta t} (\jump{\ul_2^n},\jump{\wl_2^n})_{\Omega}.
\end{align*}

 With the definition of $\cL_h$, the first order optimality conditions take the form: Find $(\Uh,\Zh) \in \ProdFullyDiscrSpace{k}{q} \times \ProdFullyDiscrSpace{k^\ast}{q^\ast}$ such that 
\begin{alignat*}{2}
    (\ul_1,\wl_1)_{\omega_T} \! + \! A[\Wh,\Zh] + S_h(\Uh,\Wh) + \Sud(\Uh,\Wh) &= (u_{\omega},\wl_1)_{\omega_T} \ &&\forall \Wh \in \ProdFullyDiscrSpace{k}{q} \\
    A[\Uh,\Yh] -  S_h^\ast(\Yh,\Zh) &= 0 \ &&\forall \Yh \in \ProdFullyDiscrSpace{k^\ast}{q^\ast} 
\end{alignat*}
In a more compact form, we can write these conditions as: Find $(\Uh,\Zh) \in \ProdFullyDiscrSpace{k}{q} \times \ProdFullyDiscrSpace{k^\ast}{q^\ast}$ such that
\begin{equation}\label{eq:discreteProblem}
    B[(\Uh,\Zh),(\Wh,\Yh)] = (u_{\omega},\wl_1)_{\omega_T} \quad \forall (\Wh,\Yh) \in \ProdFullyDiscrSpace{k}{q} \times \ProdFullyDiscrSpace{k^\ast}{q^\ast},
\end{equation}
where 
\begin{equation}
    \begin{aligned}
        B[(\Uh,\Zh),(\Wh,\Yh)] \coloneqq \ &(\ul_1,\wl_1)_{\omega_T} + A[\Wh,\Zh]+ S_h(\Uh,\Wh) \\
        &+  \Sud(\Uh,\Wh) + A[\Uh,\Yh] - S_h^\ast(\Yh,\Zh).
    \end{aligned}
\end{equation}

\section{Error analysis}
This section outlines how the steps from \cite{BP24} can be adapted to the heterogeneous setting to analyze the approximation of \eqref{eq:waveEquation}-\eqref{eq:IF-conditions}+\eqref{eq:dataMatch} through \eqref{eq:discreteProblem}. In \cref{sec:analysis:infsup}, we establish inf-sup stability and continuity of the discrete problem with respect to suitable discrete norms.
In \cref{sec:analysis:approx}, we show that the error between the first component $\ul_1$ of the discrete solution and the continuous solution $u$ is bounded by the best approximation error in these norms. Finally, we argue that \cref{assum:LipschitzStability} allows recovering convergence rates in a physically relevant norm.


\subsection{Inf-sup stability and continuity}\label{sec:analysis:infsup}
Let $\vert \cdot \vert_{S_h}, \vert \cdot \vert_{\uparrow \downarrow}$, and $\Vert \cdot \Vert_{S_h^{\ast}}$ be the (semi-)norms induced by the stabilizers $S_h$, $\Sud$, and $S_h^{\ast}$:
\begin{equation}
    \vert \Uh \vert^2_{S_h} := S_h(\Uh,\Uh), \quad \vert \Uh \vert^2_{\uparrow \downarrow} := \Sud(\Uh,\Uh), \quad \Vert \Zh \Vert^2_{S_h^{\ast}} := S_h^{\ast}(\Zh,\Zh).
\end{equation}
Then, we define the discrete norm 
\begin{equation}
    \tnorm{ (\Uh,\Zh) }^2 :=  \vert \Uh \vert^2_{S_h} + \vert \Uh \vert^2_{\uparrow \downarrow} + \Vert \ul_1 \Vert^2_{\omega_T} + \Vert \Zh \Vert^2_{S_h^\ast},
\end{equation}
and its strengthened version
\begin{equation}\label{eq:tnormwop}
    \begin{aligned}
        \tnorm{(\Uh, \Zh)}^2_{\wop} := \tnorm{(\Uh, \Zh)}^2 &+ \sum_{n = 0}^{N -1} \Big\{ \Vert \dt \ul_2 \Vert^2_{Q^n} + \Vert c^2 \nabla \ul_1 \Vert^2_{Q^n} + \Vert \dt \ul_1 \Vert^2_{Q^n} \\
        &+ \Vert \ul_2 \Vert^2_{Q^n} + \int_{I_n} \sum_{K \in \mathcal{T}_h} h^2 \Vert c^2 \ul_1 \Vert^2_{H^2(K)} \dT \Big\}. 
    \end{aligned}
\end{equation}
%Note it suffices to show that $\tnorm{(\cdot,\cdot)}$ is a norm. We modify the steps from \cite{BP24} to show the following result. 

\begin{lem}
    The expressions $\tnorm{(\cdot,\cdot)}$ and $\tnorm{(\cdot,\cdot)}_{\wop}$ are norms on $\ProdFullyDiscrSpace{k}{q} \times \ProdFullyDiscrSpace{k^\ast}{q^\ast}$.  
\end{lem}

\begin{proof}
    It suffices to show that $\tnorm{(\cdot,\cdot)}$ is a norm. Since $\tnorm{(\cdot,\cdot)}$ is a semi-norm by definition, we only require that $\tnorm{(\Uh,\Zh)} = 0$ implies $\Uh=\Zh= 0$. Assume that $\tnorm{(\Uh,\Zh)} = 0$. Then, in particular, $\vert \Uh \vert_{\uparrow \downarrow} = 0$ and therefore $(\ul_1,\ul_2) \in [H^1(Q)]^2$. Furthermore, by definition of the stabilization terms $S_h$ and $S_h^{\ast}$, we have that $\Zh = 0$, $\ul_1 \vert_{\Sigma} = 0$, $\ul_1 \vert_{\omega_T} = 0$ and $\dt \ul_1 = \ul_2$. Similar to \cite[Lem. 2.1]{BP24}, partial integration yields 
    \begin{align*}
	   & \Vert \wop \ul_1 \Vert^2_{H^{-1}(Q)} := \sup_{\substack{  y \in H^1_0(Q), \\ \norm{y}_{H^1(Q) } = 1  }} \int_{Q} \left\{ -(\dt \ul_1) \dt y + c^2 \nabla \ul_1 \nabla y \right\}  \\
        &= \sup_{\substack{  y \in H^1_0(Q), \\ \norm{y}_{H^1(Q) } = 1  }} \Big\{ \sum_{n = 0}^{N-1} (\ul_2 - \dt \ul_1, \dt y)_{Q_n} 
        + \sum_{n = 0}^{N-1} \int_{I_n} \sum_{K \in \mathcal{T}_h} (\dt \ul_2 - \div(c^2 \nabla \ul_1),y)_{K} \ \dT \\
	    & \hspace{6em} + \sum_{n = 0}^{N-1} \int_{I_n} \sum_{F \in \mathcal{F}_i} (\jump{c^2 \nabla \ul_1} \cdot \mathbf{n}, y)_F \ \dT + \sum_{n = 0}^{N-1} (\jump{\ul_1^n},y)_{\Sigma^n} \Big\},
    \end{align*}
where $\jump{c^2 \nabla \ul_1} \cdot \mathbf{n}$  denotes the jump of the normal derivative over interior facets. It follows by definition of the stabilizers and $\tnorm{(\Uh,\Zh)} = 0$ that $\Vert \wop \ul_1 \Vert_{H^{-1}(Q)} = 0$ and that the interface conditions \eqref{eq:IF-conditions} are fulfilled. Thus, by taking $\mu \rightarrow \infty$ in \cref{thm:filippas}, we conclude that $u_0 = u_1 = 0$. As solutions to \eqref{eq:waveEquation}-\eqref{eq:IF-conditions}+\eqref{eq:initialData} depend continuously on the data \cite{StolkPhD}, it follows that $\Uh = 0$. 
\end{proof}

 From this result and the identity
\[
B[(\Uh,\Zh),(\Uh,-\Zh)] = \tnorm{(\Uh,\Zh)}^2 = \tnorm{(\Uh,\Zh)}\tnorm{(\Uh,-\Zh)},
\]
it directly follows that the bilinear form $B$ enjoys inf-sup stability on $\ProdFullyDiscrSpace{k}{q} \times \ProdFullyDiscrSpace{k^\ast}{q^\ast}$ with respect to the $\tnorm{(\cdot,\cdot)}$-norm. 

\begin{cor}\label{cor:infsup}
    There exists a constant $C_B>0$ such that 
    \begin{equation}
        \sup_{(\Wh,\Yh) \in \ProdFullyDiscrSpace{k}{q} \times \ProdFullyDiscrSpace{k^\ast}{q^\ast}} \frac{B[(\Uh,\Zh),(\Wh,\Yh)]}{\tnorm{(\Wh,\Yh)}} \ge C_B \tnorm{(\Uh,\Zh)}
    \end{equation}
    for all $(\Uh,\Zh) \in \ProdFullyDiscrSpace{k}{q} \times \ProdFullyDiscrSpace{k^\ast}{q^\ast}$.
\end{cor}

%\begin{proof}
%    Since $B[(\Uh,\Zh),(\Uh,-\Zh)] = \tnorm{(\Uh,\Zh)}^2 = \tnoas .. does not holdm{(\Uh,\Zh)}\tnorm{(\Uh,-\Zh)}$, we conclude that  
%    \begin{align*}
%        \sup_{(\Wh,\Yh) \in \ProdFullyDiscrSpace{k}{q} \times \ProdFullyDiscrSpace{k^\ast}{q^\ast}} &\frac{B[(\Uh,\Zh),(\Wh,\Yh)]}{\tnorm{(\Wh,\Yh)}} \\
%        &\quad \ge \frac{B[(\Uh,\Zh),(\Uh,-\Zh)]}{\tnorm{(\Uh,-\Zh)}} = \tnorm{(\Uh,\Zh)}.
%    \end{align*}
%\end{proof}


The trace inequality \eqref{eq:traceInequality} and the Cauchy-Schwarz inequality immediately yield the continuity of the bilinear form $A$.

\begin{lem}[Continuity of $A$]\label{lem:continuityA}
If $\mathbf{U}|_{Q^n} \in [H^1(Q^n) \cap L^2(0,T;H^2(\mathcal{T}_h))] \times H^1(Q^n)$ for $n=0,\ldots,N-1$ and for all $\Yh \in \ProdFullyDiscrSpace{k^\ast}{q^\ast}$, we have that 
    \begin{equation*}
        A[\mathbf{U},\Yh] \le C \tnorm{(\mathbf{U}, 0)}_{\wop} \tnorm{(0,\Yh)}. 
    \end{equation*} 
\end{lem}
\begin{comment}
\begin{proof}
    First, note that $\tnorm{(0,\Yh)} = \Vert \Yh \Vert_{S_h^\ast}$. Using the Cauchy-Schwarz inequality, we obtain
    \begin{equation*}
        \sum_{n = 0}^{N-1} \left\{ (\dt u_2, \underline{y}_1)_{Q^n} + a(u_1,\underline{y}_1)_{Q^n} \right\} \le C \left( \sum_{n = 0}^{N-1} \Vert \dt u_2 \Vert^2_{Q^n} + \Vert c^2 \nabla u_1 \Vert^2_{Q^n} \right)^{1/2} \! \! \! \Vert \Yh \Vert_{S_h^{\ast}},
    \end{equation*}
    and 
    \begin{equation*}
        \sum_{n = 0}^{N-1} (\dt u_1 - u_2,\underline{y}_2)_{Q^n} \le C \left( \sum_{n = 0}^{N-1} \Vert \dt u_1 \Vert^2_{Q^n} + \Vert u_2 \Vert^2_{Q^n} \right)^{1/2} \Vert \Yh \Vert_{S_h^{\ast}}.
    \end{equation*}
    Since the interface $\Gamma$ is fitted by the triangulation, we have that $c^2 \nabla u_1 \in H^1(K)$ for each $K \in \mathcal{T}_h$. Thus, we can apply the trace inequality \eqref{eq:traceInequality} to obtain
    \begin{align*}
        \sum_{n = 0}^{N-1} (c^2 \nabla &u_1 \cdot n, \underline{y}_1)_{\Sigma^n} \\
        &\le \left( \sum_{n = 0}^{N-1} \int_{I_n} \sum_{K \in \mathcal{T}_h} h \Vert c^2 \nabla v \Vert_{\partial \Omega \cap \partial K}^2 \dT \right)^{1/2} \left( \sum_{n = 0}^{N-1} \int_{I_n} \sum_{K \in \mathcal{T}_h} h^{-1} \Vert \underline{y}_1 \Vert_{\partial \Omega \cap \partial K}^2 \dT \right)^{1/2} \\
        &\le C \left( \sum_{n = 0}^{N-1} \left\{ \Vert c^2 \nabla u_1 \Vert^2_{Q^n} + \int_{I_n} \sum_{K \in \mathcal{T}_h} h^2 \Vert c^2 u_1 \Vert_{H^2(K)} \dT \right\} \right)^{1/2} \Vert \Yh \Vert_{S_h^{\ast}}.
    \end{align*}
    Putting all estimates together, the claim follows. 
\end{proof}
\end{comment}

\subsection{Approximation results}\label{sec:analysis:approx}
With \cref{cor:infsup} and \cref{lem:continuityA}, we can show that the approximation error in the $\tnorm{(\cdot,\cdot)}$-norm is bounded by the best approximation error in the $\tnorm{(\cdot,\cdot)}_{\wop}$-norm.

\begin{thm}\label{thm:bestapprox}
    Let $u$ be a sufficient regular solution of \eqref{eq:waveEquation}-\eqref{eq:IF-conditions}+\eqref{eq:dataMatch} and $(\Uh,\Zh) \in \ProdFullyDiscrSpace{k}{q} \times \ProdFullyDiscrSpace{k^\ast}{q^\ast}$ be the solution to \eqref{eq:discreteProblem}. Set $\mathbf{U} := (u,\partial_t u)$. Then, there exists a constant $C>0$ such that
    \begin{equation}
   \tnorm{(\mathbf{U} - \Uh,\Zh)} \le \left( 1 + \frac{C}{C_B} \right) \inf_{\Vh \in \ProdFullyDiscrSpace{k}{q}} \tnorm{(\mathbf{U} - \Vh, 0)}_{\wop}.  
    \end{equation}
\end{thm}

\begin{proof}
    Let $\Vh \in \ProdFullyDiscrSpace{k}{q}$  and $(\Wh,\Yh) \in \ProdFullyDiscrSpace{k}{q} \times \ProdFullyDiscrSpace{k^\ast}{q^\ast}$ be arbitrary. The triangle inequality yields 
    \begin{equation*}
        \tnorm{(\mathbf{U} - \Uh,\Zh)} \le \tnorm{(\mathbf{U} - \Vh,0)} + \tnorm{(\Vh - \Uh,\Zh)}.
    \end{equation*}
    For the second term, we consider
    \begin{align*}
        B[&(\Uh - \Vh, \Zh),( \Wh, \Yh)] \\
        &= B[(\Uh,\Zh),( \Wh, \Yh)] - (\underline{v}_1, \underline{w}_1)_{\omega_T} - S_h(\Vh,\Wh) - \Sud(\Vh,\Wh) - A[\Vh,\Yh] \\
        &= (u_{\omega}, \underline{w}_1)_{\omega_T} - (\underline{v}_1, \underline{w}_1)_{\omega_T} - S_h(\Vh,\Wh) - \Sud(\Vh,\Wh) - A[\Vh,\Yh] \\
        &= (u - \underline{v}_1, \underline{w}_1)_{\omega_T}  +  S_h(\mathbf{U} - \Vh,\Wh) + \Sud(\mathbf{U} - \Vh,\Wh) + A[\mathbf{U} - \Vh,\Yh],
    \end{align*} 
    where we use the fact that $\mathbf{U}$ is sufficiently smooth and $u = u_{\omega}$ on $\omega_T$. Then, we can apply \cref{lem:continuityA} to obtain 
    \begin{align*}
        (u - &\underline{v}_1, \underline{w}_1)_{\omega_T}  +  S_h(\mathbf{U} - \Vh,\Wh) + \Sud(\mathbf{U} - \Vh,\Wh) + A[\mathbf{U} - \Vh,\Yh] \\
        &\le C(\tnorm{(\mathbf{U} - \Vh,0)}_{\wop} \tnorm{(0,\Yh)}  + \tnorm{(\mathbf{U} - \Vh,0)} \tnorm{(\Wh,0)} ) \\
        &\le C \tnorm{(\mathbf{U} - \Vh,0)}_{\wop} \tnorm{(\Wh,\Yh)}.
    \end{align*}
    Thus, the inf-sup condition on $B$ yields that 
    \begin{equation*}
    \tnorm{(\mathbf{U} - \Uh,\Zh)} \le \frac{C}{C_B} \tnorm{(\mathbf{U} - \Vh, 0)}_{\wop},
    \end{equation*}
    which gives the claim. 
\end{proof}


\begin{cor}\label{cor:tnormConvRates}
    Under the assumption of \cref{thm:bestapprox}, we have that
    \begin{equation*}
        \tnorm{(\mathbf{U} - \Uh,\Zh)} \le C h^s \Vert u \Vert_{H^m(Q)},
    \end{equation*}
    where $(s,m) := (\min\{ k,q \}, \max\{k,q\}+3)$. 
\end{cor}

\begin{proof}
    Due to the assumption that $c_i \in C^\infty(\Omega_i)$, $i = 1,2$, there exists $C > 0$ s.t.  
    \begin{equation*}
        \tnorm{(\Uh - \Vh, 0)}_{\wop} \le C \tnorm{(\Uh - \Vh, 0)}_{\square_1},
    \end{equation*}
    where $\tnorm{(\cdot, \cdot)}_{\square_1}$ is defined by \eqref{eq:tnormwop} with $c_i = 1$, $i = 1,2$. Thus, we can apply the interpolation results derived in \cite[Lemma 5]{BP24} to obtain the claim. 
\end{proof}


If the problem is Lipschitz stable, for instance in the homogeneous case (see \cref{thm:Lipschitz}), we can prove convergence in a physically relevant norm. To deal with the discontinuity in time of the discrete solution, a suitable lifting operator$L_{\Delta t} : \FullyDiscrSpace{k}{q} \rightarrow C^0(0,T;V_h^k)$ that ensures that $L_{\Delta t} \ul_1 \in H^1(Q)$ has been introduced in \cite[Sec. 4]{BP24}. Then, we state the following result.

\begin{thm}\label{thm:convergence}
    Let \cref{assum:LipschitzStability} be satisfied. For $u$ being a sufficiently regular solution of \eqref{eq:waveEquation} and $(\Uh,\Zh) \in \ProdFullyDiscrSpace{k}{q} \times \ProdFullyDiscrSpace{k^\ast}{q^\ast}$ being the solution to \eqref{eq:discreteProblem}, there exists a constant $C>0$ such that
    \begin{equation}
        \Vert u - L_{\Delta t} \ul_1 \Vert_{L^\infty(0,T;L^2(\Omega))} + \Vert \dt (u - L_{\Delta t} \ul_1) \Vert_{L^2(0,T;H^{-1}(\Omega))} \le C h^s \Vert u \Vert_{H^m(Q)}. 
    \end{equation}
    In particular, we have that 
    \begin{equation}
        \Vert u_0 - \ul_1 \vert_{t = 0} \Vert_{L^2(\Omega)} + \Vert u_1 - (\dt \ul_1) \vert_{t = 0} \Vert_{H^{-1}(\Omega)} \le C h^s \Vert u \Vert_{H^m(Q)}.
    \end{equation}
\end{thm}

\begin{proof}
    The proof follows with the same arguments as the proof of \cite[Theorem 10]{BP24}.
\end{proof}


\begin{rmk}[Homogenous media]\label{rem:homMedia}
    If $c_i = 1$ for $i = 1,2$, \cref{assum:LipschitzStability} holds (see \cref{thm:Lipschitz}), and we recover the result from \cite[Theorem 10]{BP24}.
\end{rmk}

\begin{rmk}[Logarithmic stability]
    If \cref{assum:LipschitzStability} is not satisfied, we could still hope to utilize the results from \cref{thm:filippas}, provided that $T$ is large enough such that \eqref{eq:Tcondition} holds. To this end, we would have to deal with the following technicalities:  
    \begin{enumerate}
        \item The results from \cref{thm:filippas} are formulated in terms of $\Vert \wop u \Vert_{L^2(Q)}$, but we can only control $\Vert \wop \tilde{e}_h \Vert_{H^{-1}(Q)}$ for the residual $\tilde{e}_h = u - L_{\Delta t} \ul_1$. Thus, we would have to rewrite the estimates from \cref{thm:filippas} with $\Vert \wop u \Vert_{H^{-1}(Q)}$ instead of $\Vert \wop u \Vert_{L^2(Q)}$. To this end, one might be able to shift the regularity suitably, cf. \cite[Sec. 2.6]{StolkPhD}, or use similar arguments as in \cite[Thm. A.1]{BFMO21}.
        \item We have to ensure that the constants in the estimates from \cref{thm:filippas} remain bounded when we consider the residual $\tilde{e}_h$. In particular, we have to control $C_{0,1} := \Vert (\tilde{e}_h \vert_{t = 0}, \dt \tilde{e}_h \vert_{t = 0}) \Vert_{H^1 \times L^2}$. We note that we can always ensure that $C_{0,1}$ is bounded by adding a suitable Tikhonov regularization term to \eqref{eq:primalStab}. 
    \end{enumerate}
    Then, we can use the arguments of \cite[Theorem 10]{BP24} to show that 
    \begin{equation*}
        \Vert (u - \ul_1) \vert_{t = 0} \Vert_{L^2(\Omega)}  + \Vert \dt (u - \ul_1) \vert_{t = 0} \Vert_{H^{-1}(\Omega)} \le C C_{0,1} \log(1 + C_{0,1} h^{-s} \Vert u \Vert_{H^m(Q)}^{-1})^{-1}. 
    \end{equation*}
\end{rmk}


\section{Numerical experiments}
 In this section, we present numerical experiments carried out with the proposed method. The examples are implemented using the \texttt{FEniCSx} library \cite{BarattaEtal2023,BasixJoss,ScroggsEtal2022,AlnaesEtal2014}. 
~\jp{We need to prepare some reproduction material ...}
\subsection{Examples in one space dimension}
 We consider the unit interval $\Omega \coloneqq (0,1) \subset \mathbb{R}$ partitioned into $2^{L+1}$, $L \in \mathbb{N}$, uniform elements of size $h = 1/2^{L+1}$. Unless specified otherwise, we choose the time step as $\Delta t = h$. 

\subsubsection{Simple Example}\label{sec:numex:1D:simple}
 We define two subdomains $\Omega_1 = (0,0.5)$ and $\Omega_2 = (0.5,1.0)$ such that $\bar{\Omega} = \overline{\Omega}_1 \cup \overline{\Omega}_2$ and fix the wavespeed $c_2 := c \vert_{\Omega_2} = 1$ in $\Omega_2$. For variable $c_1 := c \vert_{\Omega_1}$, we consider the exact solution \cite{MHI08}
\begin{equation}\label{eq:1D:exact:simple}
    u(x,t) := \begin{cases}
        \cos(w_1 c_1 t) \cos(w_1(x-0.5)), & x \in \Omega_1, \\
        \cos(w_2 c_2 t) \cos(w_2(x-0.5)), & x \in \Omega_2,
    \end{cases}
\end{equation}
where we set $w_1 = 3 \pi$ and $w_2 = w_1 c_1 / c_2$. Furthermore, we define
\begin{equation}
	\omega \coloneqq [0.0,0.25] \cup [0.75,1.0], \quad \text{and} \quad \Omega_R \coloneqq \bar{\Omega} \setminus \omega = [0.25,0.75].
\end{equation}


In one space dimension, the requirement \eqref{eq:Tcondition} on $T$ already implies the GCC. Let $d(x,y) \coloneqq \vert x - y \vert$, $x, y \in \Omega$, be the Euclidean metric. Then, we set
\begin{equation}
    d_c(x,y) \coloneqq \mathbbm{1}_{\Omega_1} c_1(x)^{-1} d(x,y) + \mathbbm{1}_{\Omega_2} d(x,y), \quad x,y \in \Omega.
\end{equation}
~\jp{I am confused by this formula. What is the argument of the characteristic functions $\mathbbm{1}_{\Omega_1}$? Also, a metric is required to be symmetric, so $d_c(x,y) = d_c(y,x) $ 
needs to be ensured?
}
Thus, for $\omega$ as above, $c_2 = 1$ fixed and $c_1 > 1.0$, the condition \eqref{eq:Tcondition} translates to $T > 0.25(1+c_1^{-1})$. 


\Cref{fig:jumpCoefs:contrast:2.5} shows that the errors $\Vert u-L_{\Delta t} \ul_1 \Vert_{L^\infty(0,T;L^2(\Omega_R))}$ and $\Vert \partial_t (u-L_{\Delta t} \ul_1) \Vert_{L^2(0,T;L^2(\Omega_R))}$ converge with order $\mathcal{O}(h^k)$ for $T$ large enough, but if $T$ is chosen too small, the order of convergence degrades.

In \cref{fig:jumpCoefs:increasingcontrast} the dependence of the errors for increasing $c_1$ and 
leaving $c_2 = 1$ fixed (i.e.\ increasing the contrast value) is investigated. When $T$ is chosen such that the GCC is barely fulfilled we observe that the errors merely grow polynomially in the contrast, more precisely like $\mathcal{O}(c_1^3)$.
This is an important observation since an exponential growth in the contrast would have rendered the corresponding problems nearly impossible to resolve numerically.
Still, compared to the best approximation error which grows quadratically we loose a factor of $c_1$. This entails that a strong refinement of the discretization is required to maintain the accuracy of the numerical solution as the contrast increases. We illustrate this problem in \cref{fig:jumpCoefs:quality}. For $c_1 = 2.5$ the exact solution is already well-captured by the numerical solution on refinement level $L=2$. However, for $c_1 = 5.5$ this refinement level is clearly insufficient. 

%should come from: 1D-jump-convergence_Ex2.py
\begin{comment}
\begin{figure}[!htbp]
    \begin{center}
        \begin{tikzpicture}[scale=0.72]
            \begin{groupplot}[%
                group style={%
                group size=2 by 2,
                horizontal sep=1.25cm,
                vertical sep=2cm,
                },
            ymajorgrids=true,
            grid style=dashed,
            %ymin = 1e-3, ymax = 0.5e1,
            ]    
            \nextgroupplot[width=8cm,height=6.2cm,domain=1:4,xmode=linear,ymode=log, ylabel={}, xlabel={$L$}, title={$T=0.5$, $\Vert (u-L_{\Delta t} \ul_1) \Vert_{L^\infty(0,T;L^2(\Omega_R))}$}, %cycle list name=, 
            legend pos=south west, %yticklabels={,,},
            ]
           
            %L-infty-L2-u (k = 2)
            %\addplot+[discard if not={order}{2}, discard if not={contrast}{2.5},line width=1.1pt,color=orange,mark=square*] table [x=L, y=L-infty-L2-error-u, col sep=comma] {../data/newexact_1D_jumpingCoefs_k3_WPFalse.csv};
            %\addplot+[discard if not={order}{2}, discard if not={contrast}{2.5},line width=1.1pt,dashed,color=orange,mark=square*] table [x=L, y=bestapprox-L-infty-L2-error-u, col sep=comma] {../data/newexact_1D_jumpingCoefs_k3_WPFalse.csv};
            
            \addplot+[discard if not={order}{2}, discard if not={contrast}{2.5},line width=1.1pt,color=orange,mark=square*] table [x=L, y=L-infty-L2-error-u, col sep=comma] {../data/simpleExact_T0.5_1D_jumpingCoefs_k3.csv};
            \addplot+[discard if not={order}{2}, discard if not={contrast}{2.5},line width=1.1pt,dashed,color=orange,mark=square*] table [x=L, y=bestapprox-L-infty-L2-error-u, col sep=comma] {../data/simpleExact_T0.5_1D_jumpingCoefs_k3.csv};
            %\addplot+[discard if not={order}{2}, discard if not={contrast}{2.5},line width=2.1pt,dashed,color=orange,mark=square*] table [x=L, y=bestapprox-L-infty-L2-error-u, col sep=comma] {../dataNew/simpleExact_T0.5_multipleContrasts_1D_jumpingCoefs_k2.csv};
            
 
           
            %L-infty-L2-u (k = 3)
            %\addplot+[discard if not={order}{3}, discard if not={contrast}{2.5},line width=1.1pt,color=cyan!60!black,mark=diamond*] table [x=L, y=L-infty-L2-error-u, col sep=comma] {../data/newexact_1D_jumpingCoefs_k3_WPFalse.csv};
            %\addplot+[discard if not={order}{3}, discard if not={contrast}{2.5},line width=1.1pt,dashed,color=cyan!60!black,mark=diamond*] table [x=L, y=bestapprox-L-infty-L2-error-u, col sep=comma] {../data/newexact_1D_jumpingCoefs_k3_WPFalse.csv};

            \addplot+[discard if not={order}{3}, discard if not={contrast}{2.5},line width=1.1pt,color=cyan!60!black,mark=diamond*] table [x=L, y=L-infty-L2-error-u, col sep=comma] {../data/simpleExact_T0.5_1D_jumpingCoefs_k3.csv};
            \addplot+[discard if not={order}{3}, discard if not={contrast}{2.5},line width=1.1pt,dashed,color=cyan!60!black,mark=diamond*] table [x=L, y=bestapprox-L-infty-L2-error-u, col sep=comma] {../data/simpleExact_T0.5_1D_jumpingCoefs_k3.csv};
        
            %\addplot[gray, dashed, domain=1:4] {15*(1/2^(2))^(x-0.9)};
            \addplot[gray, dashed, domain=1:4] {0.3*(1/2^(2))^(x-0.9)};
            %\addplot[gray, dashed, domain=1:4] {7.5*(1/2^(3))^(x-0.9)};
            \addplot[gray, dashed, domain=1:4] {0.03*(1/2^(3))^(x-0.9)};
            \node [draw=none] at (axis description cs:0.80,0.55) {\color{gray}\footnotesize $\!\!\mathcal{O}(h^{2})$};
            \node [draw=none] at (axis description cs:0.45,0.35) {\color{gray}\footnotesize $\!\!\mathcal{O}(h^{3})$};
            \legend{$k=2 $,, $k = 3$,}

            \nextgroupplot[width=8cm,height=6.2cm,domain=1:4,xmode=linear,ymode=log, ylabel={}, xlabel={$L$}, title={$T = 0.5$, $\Vert \partial_t (u-L_{\Delta t} \ul_1) \Vert_{L^2(0,T;L^2(\Omega_R))}$}, %cycle list name=, 
            legend pos=south west, %yticklabels={,,},
            ]

            %L2-L2-u_t (k = 2)
            \addplot+[discard if not={order}{2}, discard if not={contrast}{2.5},line width=1.1pt,color=orange,mark=square*] table [x=L, y=L2-L2-error-u_t, col sep=comma] {../data/simpleExact_T0.5_1D_jumpingCoefs_k3.csv};
            \addplot+[discard if not={order}{2}, discard if not={contrast}{2.5},line width=1.1pt,dashed,color=orange,mark=square*] table [x=L, y=bestapprox-L2-L2-error-u_t, col sep=comma] {../data/simpleExact_T0.5_1D_jumpingCoefs_k3.csv};


            %L2-L2-u_t (k = 3)
            \addplot+[discard if not={order}{3}, discard if not={contrast}{2.5},line width=1.1pt,color=cyan!60!black,mark=diamond*] table [x=L, y=L2-L2-error-u_t, col sep=comma] {../data/simpleExact_T0.5_1D_jumpingCoefs_k3.csv};
            \addplot+[discard if not={order}{3}, discard if not={contrast}{2.5},line width=1.1pt,dashed,color=cyan!60!black,mark=diamond*] table [x=L, y=bestapprox-L2-L2-error-u_t, col sep=comma] {../data/simpleExact_T0.5_1D_jumpingCoefs_k3.csv};
            
            \addplot[gray, dashed, domain=1:4] {4.5*(1/2^(2))^(x-0.9)};
            %\addplot[gray, dashed, domain=1:4] {1.0*(1/2^(2))^(x-0.9)};
            \addplot[gray, dashed, domain=1:4] {1.0*(1/2^(3))^(x-0.9)};
            %\addplot[gray, dashed, domain=1:4] {0.5*(1/2^(3))^(x-0.9)};
            \node [draw=none] at (axis description cs:0.30,0.87) {\color{gray}\footnotesize $\!\!\mathcal{O}(h^{2})$};
            \node [draw=none] at (axis description cs:0.65,0.2) {\color{gray}\footnotesize $\!\!\mathcal{O}(h^{3})$};
            \legend{$k=2 $,, $k = 3$,}
            \nextgroupplot[width=8cm,height=6.2cm,domain=1:4,xmode=linear,ymode=log, ylabel={}, xlabel={$L$}, title={$T = 0.1$, $\Vert (u-L_{\Delta t} \ul_1) \Vert_{L^\infty(0,T;L^2(\Omega))}$}, %cycle list name=, 
            legend pos=south west, %yticklabels={,,},
            ]
           
            %L-infty-L2-u (k = 2)
            \addplot+[discard if not={order}{2}, discard if not={contrast}{2.5},line width=1.1pt,color=orange,mark=square*] table [x=L, y=L-infty-L2-error-u, col sep=comma] {../data/simpleExact_T0.1_1D_jumpingCoefs_k3.csv};
            \addplot+[discard if not={order}{2}, discard if not={contrast}{2.5},line width=1.1pt,dashed,color=orange,mark=square*] table [x=L, y=bestapprox-L-infty-L2-error-u, col sep=comma] {../data/simpleExact_T0.1_1D_jumpingCoefs_k3.csv};

           
            %L-infty-L2-u (k = 3)
            \addplot+[discard if not={order}{3}, discard if not={contrast}{2.5},line width=1.1pt,color=cyan!60!black,mark=diamond*] table [x=L, y=L-infty-L2-error-u, col sep=comma] {../data/simpleExact_T0.1_1D_jumpingCoefs_k3.csv};
            \addplot+[discard if not={order}{3}, discard if not={contrast}{2.5},line width=1.1pt,dashed,color=cyan!60!black,mark=diamond*] table [x=L, y=bestapprox-L-infty-L2-error-u, col sep=comma] {../data/simpleExact_T0.1_1D_jumpingCoefs_k3.csv};
        
            %\addplot[gray, dashed, domain=1:4] {15*(1/2^(2))^(x-0.9)};
            \addplot[gray, dashed, domain=1:4] {0.005*(1/2^(2))^(x-0.9)};
            %\addplot[gray, dashed, domain=1:4] {7.5*(1/2^(3))^(x-0.9)};
            \addplot[gray, dashed, domain=1:4] {0.001*(1/2^(3))^(x-0.9)};
            \addplot[gray,dashed] table {
                1 0.009999968541169805
                2 0.006666645694113204
                3 0.004999984270584902
                4 0.003999987416467922

            };
            \node [draw=none] at (axis description cs:0.75,0.45) {\color{gray}\footnotesize $\!\!\mathcal{O}(h^{2})$};
            \node [draw=none] at (axis description cs:0.45,0.35) {\color{gray}\footnotesize $\!\!\mathcal{O}(h^{3})$};
            \node [draw=none] at (axis description cs:0.7,0.75) {\color{gray}\footnotesize $\!\!\mathcal{O}(\vert \log(h) \vert^{-1})$};
            \legend{$k=2 $,, $k = 3$,}

            \nextgroupplot[width=8cm,height=6.2cm,domain=1:4,xmode=linear,ymode=log, ylabel={}, xlabel={$L$}, title={$T = 0.1$, $\Vert \partial_t (u-L_{\Delta t} \ul_1) \Vert_{L^2(0,T;L^2(\Omega))}$}, %cycle list name=, 
            legend pos=south west, %yticklabels={,,},
            ]
            %L2-L2-u_t (k = 2)
            \addplot+[discard if not={order}{2}, discard if not={contrast}{2.5},line width=1.1pt,color=orange,mark=square*] table [x=L, y=L2-L2-error-u_t, col sep=comma] {../data/simpleExact_T0.1_1D_jumpingCoefs_k3.csv};
            \addplot+[discard if not={order}{2}, discard if not={contrast}{2.5},line width=1.1pt,dashed,color=orange,mark=square*] table [x=L, y=bestapprox-L2-L2-error-u_t, col sep=comma] {../data/simpleExact_T0.1_1D_jumpingCoefs_k3.csv};
            

            %L2-L2-u_t (k = 3)
            \addplot+[discard if not={order}{3}, discard if not={contrast}{2.5},line width=1.1pt,color=cyan!60!black,mark=diamond*] table [x=L, y=L2-L2-error-u_t, col sep=comma] {../data/simpleExact_T0.1_1D_jumpingCoefs_k3.csv};
            \addplot+[discard if not={order}{3}, discard if not={contrast}{2.5},line width=1.1pt,dashed,color=cyan!60!black,mark=diamond*] table [x=L, y=bestapprox-L2-L2-error-u_t, col sep=comma] {../data/simpleExact_T0.1_1D_jumpingCoefs_k3.csv};
            
            \addplot[gray, dashed, domain=1:4] {0.1*(1/2^(2))^(x-0.9)};
            %\addplot[gray, dashed, domain=1:4] {1.0*(1/2^(2))^(x-0.9)};
            \addplot[gray, dashed, domain=1:4] {0.02*(1/2^(3))^(x-0.9)};
            %\addplot[gray, dashed, domain=1:4] {0.5*(1/2^(3))^(x-0.9)};
            \addplot[gray,dashed] table {
                1 0.09999968541169805
                2 0.06666645694113204
                3 0.04999984270584902
                4 0.03999987416467922

            };
            \node [draw=none] at (axis description cs:0.80,0.40) {\color{gray}\footnotesize $\!\!\mathcal{O}(h^{2})$};
            \node [draw=none] at (axis description cs:0.65,0.2) {\color{gray}\footnotesize $\!\!\mathcal{O}(h^{3})$};
            \node [draw=none] at (axis description cs:0.7,0.75) {\color{gray}\footnotesize $\!\!\mathcal{O}(\vert \log(h) \vert^{-1})$};
            \legend{$k=2 $,, $k = 3$,}
            \end{groupplot}
        \end{tikzpicture}
    \end{center}
    \caption{For the approximation of \eqref{eq:1D:exact:simple} with polynomial degree $k \in \{2,3\}$ and contrast $c_1 = 2.5$ we measure the errors $\Vert u-L_{\Delta t} \ul_1 \Vert_{L^\infty(0,T;L^2(\Omega_R))}$ (left) and $\Vert \partial_t (u-L_{\Delta t} \ul_1) \Vert_{L^2(0,T;L^2(\Omega_R))}$ (right) and compare to the respective error of the $L^2$-bestapproximation (dashed). We consider the final time $T = 0.5$ (top) and $T = 0.1$ (bottom). The gray dashed lines indicate the observed convergence rates.}
    \label{fig:jumpCoefs:contrast:2.5}
  \end{figure}
\end{comment}

\begin{figure}[!htbp]
    \begin{center}
        \begin{tikzpicture}[scale=0.8]
            \begin{groupplot}[%
                group style={%
                group size=2 by 2,
                horizontal sep=1.25cm,
                vertical sep=2cm,
                },
            ymajorgrids=true,
            grid style=dashed,
            %ymin = 1e-3, ymax = 0.5e1,
            ]    
            \nextgroupplot[width=8cm,height=6.2cm,domain=1:4,xmode=linear,ymode=log, ylabel={}, xlabel={$L$}, title={$T=0.5$, $\Vert (u-L_{\Delta t} \ul_1) \Vert_{L^\infty(0,T;L^2(\Omega_R))}$}, %cycle list name=,
            xtick={1,2,3,4}, 
            legend style={legend columns=4, draw=none,nodes={scale=.8}},legend to name=named,
            ]
           
            %L-infty-L2-u (k = 2)
            %\addplot+[discard if not={order}{2}, discard if not={contrast}{2.5},line width=1.1pt,color=orange,mark=square*] table [x=L, y=L-infty-L2-error-u, col sep=comma] {../data/newexact_1D_jumpingCoefs_k3_WPFalse.csv};
            %\addplot+[discard if not={order}{2}, discard if not={contrast}{2.5},line width=1.1pt,dashed,color=orange,mark=square*] table [x=L, y=bestapprox-L-infty-L2-error-u, col sep=comma] {../data/newexact_1D_jumpingCoefs_k3_WPFalse.csv};
            
            \addplot+[discard if not={order}{2}, discard if not={contrast}{2.5},line width=1.1pt,color=orange,mark=square*] table [x=L, y=L-infty-L2-error-u, col sep=comma] {../dataNew/simplexExact_singleContrast_T0.5.csv};
            \addplot+[discard if not={order}{2}, discard if not={contrast}{2.5},line width=1.1pt,dashed,color=orange,mark=square*] table [x=L, y=bestapprox-L-infty-L2-error-u, col sep=comma] {../dataNew/simplexExact_singleContrast_T0.5.csv};
            %\addplot+[discard if not={order}{2}, discard if not={contrast}{2.5},line width=2.1pt,dashed,color=orange,mark=square*] table [x=L, y=bestapprox-L-infty-L2-error-u, col sep=comma] {../dataNew/simpleExact_T0.5_multipleContrasts_1D_jumpingCoefs_k2.csv};
            
 
           
            %L-infty-L2-u (k = 3)
            %\addplot+[discard if not={order}{3}, discard if not={contrast}{2.5},line width=1.1pt,color=cyan!60!black,mark=diamond*] table [x=L, y=L-infty-L2-error-u, col sep=comma] {../data/newexact_1D_jumpingCoefs_k3_WPFalse.csv};
            %\addplot+[discard if not={order}{3}, discard if not={contrast}{2.5},line width=1.1pt,dashed,color=cyan!60!black,mark=diamond*] table [x=L, y=bestapprox-L-infty-L2-error-u, col sep=comma] {../data/newexact_1D_jumpingCoefs_k3_WPFalse.csv};

            \addplot+[discard if not={order}{3}, discard if not={contrast}{2.5},line width=1.1pt,color=cyan!60!black,mark=diamond*] table [x=L, y=L-infty-L2-error-u, col sep=comma] {../dataNew/simplexExact_singleContrast_T0.5.csv};
            \addplot+[discard if not={order}{3}, discard if not={contrast}{2.5},line width=1.1pt,dashed,color=cyan!60!black,mark=diamond*] table [x=L, y=bestapprox-L-infty-L2-error-u, col sep=comma] {../dataNew/simplexExact_singleContrast_T0.5.csv};
        
            \addplot[gray, dotted, domain=1:4] {0.1*(1/2^(3))^(x-0.9)};
            \addplot[gray, dashed, domain=1:4] {1.45*(1/2^(2))^(x-0.9)};
            %\node [draw=none] at (axis description cs:0.80,0.55) {\color{gray}\footnotesize $\!\!\mathcal{O}(h^{2})$};
            %\node [draw=none] at (axis description cs:0.45,0.35) {\color{gray}\footnotesize $\!\!\mathcal{O}(h^{3})$};
            
            \legend{$k=2 $,, $k = 3$,,$\mathcal{O}(h^{2})$,$\mathcal{O}(h^{3})$}

            \nextgroupplot[width=8cm,height=6.2cm,domain=1:4,xmode=linear,ymode=log, ylabel={}, xlabel={$L$}, title={$T = 0.5$, $\Vert \partial_t (u-L_{\Delta t} \ul_1) \Vert_{L^2(0,T;L^2(\Omega_R))}$}, %cycle list name=, 
            xtick={1,2,3,4},
            legend pos=south west, %yticklabels={,,},
            ]

            %L2-L2-u_t (k = 2)
            \addplot+[discard if not={order}{2}, discard if not={contrast}{2.5},line width=1.1pt,color=orange,mark=square*] table [x=L, y=L2-L2-error-u_t, col sep=comma] {../dataNew/simplexExact_singleContrast_T0.5.csv};
            \addplot+[discard if not={order}{2}, discard if not={contrast}{2.5},line width=1.1pt,dashed,color=orange,mark=square*] table [x=L, y=bestapprox-L2-L2-error-u_t, col sep=comma] {../dataNew/simplexExact_singleContrast_T0.5.csv};


            %L2-L2-u_t (k = 3)
            \addplot+[discard if not={order}{3}, discard if not={contrast}{2.5},line width=1.1pt,color=cyan!60!black,mark=diamond*] table [x=L, y=L2-L2-error-u_t, col sep=comma] {../dataNew/simplexExact_singleContrast_T0.5.csv};
            \addplot+[discard if not={order}{3}, discard if not={contrast}{2.5},line width=1.1pt,dashed,color=cyan!60!black,mark=diamond*] table [x=L, y=bestapprox-L2-L2-error-u_t, col sep=comma] {../dataNew/simplexExact_singleContrast_T0.5.csv};
            
            \addplot[gray, dashed, domain=1:4] {4.5*(1/2^(2))^(x-0.9)};
            \addplot[gray, dotted, domain=1:4] {1.0*(1/2^(3))^(x-0.9)};
            %\node [draw=none] at (axis description cs:0.30,0.87) {\color{gray}\footnotesize $\!\!\mathcal{O}(h^{2})$};
            %\node [draw=none] at (axis description cs:0.65,0.2) {\color{gray}\footnotesize $\!\!\mathcal{O}(h^{3})$};
            %\legend{$k=2 $,, $k = 3$,}
            \nextgroupplot[width=8cm,height=6.2cm,domain=1:4,xmode=linear,ymode=log, ylabel={}, xlabel={$L$}, title={$T = 0.1$, $\Vert (u-L_{\Delta t} \ul_1) \Vert_{L^\infty(0,T;L^2(\Omega))}$}, %cycle list name=, 
            xtick={1,2,3,4},
            legend pos=south west, %yticklabels={,,},
            ]
           
            %L-infty-L2-u (k = 2)
            \addplot+[discard if not={order}{2}, discard if not={contrast}{2.5},line width=1.1pt,color=orange,mark=square*] table [x=L, y=L-infty-L2-error-u, col sep=comma] {../dataNew/simplexExact_singleContrast_T0.1.csv};
            \addplot+[discard if not={order}{2}, discard if not={contrast}{2.5},line width=1.1pt,dashed,color=orange,mark=square*] table [x=L, y=bestapprox-L-infty-L2-error-u, col sep=comma] {../dataNew/simplexExact_singleContrast_T0.1.csv};

           
            %L-infty-L2-u (k = 3)
            \addplot+[discard if not={order}{3}, discard if not={contrast}{2.5},line width=1.1pt,color=cyan!60!black,mark=diamond*] table [x=L, y=L-infty-L2-error-u, col sep=comma] {../dataNew/simplexExact_singleContrast_T0.1.csv};
            \addplot+[discard if not={order}{3}, discard if not={contrast}{2.5},line width=1.1pt,dashed,color=cyan!60!black,mark=diamond*] table [x=L, y=bestapprox-L-infty-L2-error-u, col sep=comma] {../dataNew/simplexExact_singleContrast_T0.1.csv};
        
            %\addplot[gray, dashed, domain=1:4] {15*(1/2^(2))^(x-0.9)};
            \addplot[gray, dashed, domain=1:4] {0.09*(1/2^(2))^(x-0.9)};
            %\addplot[gray, dashed, domain=1:4] {7.5*(1/2^(3))^(x-0.9)};
            \addplot[gray, dotted, domain=1:4] {0.05*(1/2^(3))^(x-0.9)};
            %\addplot[gray,dashed] table {
            %    1 0.009999968541169805
            %    2 0.006666645694113204
            %    3 0.004999984270584902
            %    4 0.003999987416467922
            %};
            \addplot[gray,dashdotted] table {
                1 1.609999968541169805
                2 1.606666645694113204
                3 1.604999984270584902
                4 1.603999987416467922
            };
            %\node [draw=none] at (axis description cs:0.75,0.45) {\color{gray}\footnotesize $\!\!\mathcal{O}(h^{2})$};
            %\node [draw=none] at (axis description cs:0.45,0.35) {\color{gray}\footnotesize $\!\!\mathcal{O}(h^{3})$};
            \node [draw=none] at (axis description cs:0.7,0.94) {\color{gray}\footnotesize $\!\!\mathcal{O}(\vert \log(h) \vert^{-1})$};
            
            
            %\legend{$k=2 $,, $k = 3$,}

            \nextgroupplot[width=8cm,height=6.2cm,domain=1:4,xmode=linear,ymode=log, ylabel={}, xlabel={$L$}, title={$T = 0.1$, $\Vert \partial_t (u-L_{\Delta t} \ul_1) \Vert_{L^2(0,T;L^2(\Omega))}$}, %cycle list name=, 
            xtick={1,2,3,4},
            legend pos=south west, %yticklabels={,,},
            ]
            %L2-L2-u_t (k = 2)
            \addplot+[discard if not={order}{2}, discard if not={contrast}{2.5},line width=1.1pt,color=orange,mark=square*] table [x=L, y=L2-L2-error-u_t, col sep=comma] {../dataNew/simplexExact_singleContrast_T0.1.csv};
            \addplot+[discard if not={order}{2}, discard if not={contrast}{2.5},line width=1.1pt,dashed,color=orange,mark=square*] table [x=L, y=bestapprox-L2-L2-error-u_t, col sep=comma] {../dataNew/simplexExact_singleContrast_T0.1.csv};
            

            %L2-L2-u_t (k = 3)
            \addplot+[discard if not={order}{3}, discard if not={contrast}{2.5},line width=1.1pt,color=cyan!60!black,mark=diamond*] table [x=L, y=L2-L2-error-u_t, col sep=comma] {../dataNew/simplexExact_singleContrast_T0.1.csv};
            \addplot+[discard if not={order}{3}, discard if not={contrast}{2.5},line width=1.1pt,dashed,color=cyan!60!black,mark=diamond*] table [x=L, y=bestapprox-L2-L2-error-u_t, col sep=comma] {../dataNew/simplexExact_singleContrast_T0.1.csv};
            
            \addplot[gray, dashed, domain=1:4] {0.5*(1/2^(2))^(x-0.9)};
            %\addplot[gray, dashed, domain=1:4] {1.0*(1/2^(2))^(x-0.9)};
            \addplot[gray, dotted, domain=1:4] {0.2*(1/2^(3))^(x-0.9)};
            %\addplot[gray, dashed, domain=1:4] {0.5*(1/2^(3))^(x-0.9)};
            %\addplot[gray,dashed] table {
            %    1 0.09999968541169805
            %    2 0.06666645694113204
            %    3 0.04999984270584902
            %    4 0.03999987416467922
            %};
            \addplot[gray,dashdotted] table {
                1 7.59999968541169805
                2 7.56666645694113204
                3 7.54999984270584902
                4 7.53999987416467922

            };
            %\node [draw=none] at (axis description cs:0.80,0.40) {\color{gray}\footnotesize $\!\!\mathcal{O}(h^{2})$};
            %\node [draw=none] at (axis description cs:0.65,0.2) {\color{gray}\footnotesize $\!\!\mathcal{O}(h^{3})$};
            \node [draw=none] at (axis description cs:0.7,0.94) {\color{gray}\footnotesize $\!\!\mathcal{O}(\vert \log(h) \vert^{-1})$};
            %\legend{$k=2 $,, $k = 3$,}
            \end{groupplot}
        \end{tikzpicture}
        \pgfplotslegendfromname{named}
    \end{center}
    \caption{For the approximation of \eqref{eq:1D:exact:simple} with polynomial degree $k \in \{2,3\}$ and contrast $c_1 = 2.5$ we measure the errors $\Vert u-L_{\Delta t} \ul_1 \Vert_{L^\infty(0,T;L^2(\Omega_R))}$ (left) and $\Vert \partial_t (u-L_{\Delta t} \ul_1) \Vert_{L^2(0,T;L^2(\Omega_R))}$ (right) and compare to the respective error of the $L^2$-best approximation (dashed). We consider the final time $T = 0.5$ (top) and $T = 0.1$ (bottom). The gray dashed lines indicate the observed convergence rates.}
    \label{fig:jumpCoefs:contrast:2.5}
  \end{figure}



  \begin{figure}[!htbp]
    \begin{center}
        \begin{tikzpicture}[scale=0.85]
            \begin{groupplot}[%
                group style={%
                group size=2 by 1,
                horizontal sep=1.0cm,
                vertical sep=2cm,
                },
            ymajorgrids=true,
            grid style=dashed,
            %ymin = 1e-3, ymax = 0.5e1,
            ]    
            \nextgroupplot[width=8cm,height=6.2cm,domain=1:5,xmode=linear,ymode=log, ylabel={}, xlabel={$c_1$}, title={$\Vert \partial_t (u-L_{\Delta t} \ul_1) \Vert_{L^2(0,T;L^2(\Omega_R))}$}, %cycle list name=, 
            legend style={legend columns=4, draw=none,nodes={scale=.8}},legend to name=named, %yticklabels={,,},
            xtick={1,1.5,2,2.5,3,3.5,4,4.5},
            %ymin=1e-7,ymax=1e1,
            ]
            %\addplot+[line width=1.1pt,color=teal,mark=*] table [x=contrast, y=L2-L2-error-u_t, col sep=comma]{../data/simpleExact_1D_jumpingCoefs_k3.csv};
            \addplot+[line width=1.1pt,color=teal,mark=*] table [x=contrast, y=L2-L2-error-u_t, col sep=comma]{../dataNew/simplexExact_MultContrast_T0.5.csv};
            \addplot+[line width=1.1pt,color=teal,mark=*,dashed] table [x=contrast, y=bestapprox-L2-L2-error-u_t, col sep=comma]{../dataNew/simplexExact_MultContrast_T0.5.csv};
            \addplot+[line width=1.1pt,color=orange,mark=diamond*] table [x=contrast, y=L2-L2-error-u_t, col sep=comma]{../dataNew/simplexExact_MultContrast_TAdap.csv};
            \addplot+[line width=1.1pt,color=orange,mark=diamond*,dashed] table [x=contrast, y=bestapprox-L2-L2-error-u_t, col sep=comma]{../dataNew/simplexExact_MultContrast_TAdap.csv};


            \addplot[gray, dashed, domain=1:4.5] {0.001*(1/2^(-2))^(x-0.9)};
            \addplot[gray, dashed, domain=1:4.5] {0.001*(1/2^(-3))^(x-0.9)};

            \node [draw=none] at (axis description cs:0.70,0.37) {\color{gray}\footnotesize $\!\!\mathcal{O}(c_1^{2})$};
            \node [draw=none] at (axis description cs:0.50,0.75) {\color{gray}\footnotesize $\!\!\mathcal{O}(c_1^{3})$};

            \legend{$T = 0.5$, $L^2$-best approx. ($T = 0.5$), $T$ variable, $L^2$-best approx. ($T$ variable)}
            \nextgroupplot[width=8cm,height=6.2cm,domain=1:5,xmode=linear,ymode=log, ylabel={}, xlabel={$c_1$}, title={$\Vert u-L_{\Delta t} \ul_1 \Vert_{L^\infty(0,T;L^2(\Omega_R))}$}, %cycle list name=, 
            legend pos=south east, %yticklabels={,,},
            %ymin=1e-7,ymax=1e1,
            xtick={1,1.5,2,2.5,3,3.5,4,4.5},
            ]
            \addplot+[line width=1.1pt,color=teal,mark=*] table [x=contrast, y=L-infty-L2-error-u, col sep=comma]{../dataNew/simplexExact_MultContrast_T0.5.csv};
            \addplot+[line width=1.1pt,color=teal,mark=*,dashed] table [x=contrast, y=bestapprox-L-infty-L2-error-u, col sep=comma]{../dataNew/simplexExact_MultContrast_T0.5.csv};

        
            \addplot+[line width=1.1pt,color=orange,mark=diamond*] table [x=contrast, y=L-infty-L2-error-u, col sep=comma]{../dataNew/simplexExact_MultContrast_TAdap.csv};
            \addplot+[line width=1.1pt,color=orange,mark=diamond*,dashed] table [x=contrast, y=bestapprox-L-infty-L2-error-u, col sep=comma]{../dataNew/simplexExact_MultContrast_TAdap.csv};

            
            \addplot[gray, dashed, domain=1:4.5] {0.00001*(1/2^(-2))^(x-0.9)};
            \addplot[gray, dashed, domain=1:4.5] {0.0003*(1/2^(-3))^(x-0.9)};
            \node [draw=none] at (axis description cs:0.70,0.27) {\color{gray}\footnotesize $\!\!\mathcal{O}(c_1^{2})$};
            \node [draw=none] at (axis description cs:0.50,0.75) {\color{gray}\footnotesize $\!\!\mathcal{O}(c_1^{3})$};

            \end{groupplot}
        \end{tikzpicture}
        \pgfplotslegendfromname{named}
    \end{center}
    \caption{For $k = 3$ and $L = 3$, we consider the errors $\Vert \dt (u-L_{\Delta t} \ul_1) \Vert_{L^2(0,T;L^2(\Omega_R))}$ (left) and $\Vert u-L_{\Delta t} \ul_1 \Vert_{L^\infty(0,T;L^2(\Omega_R))}$ (right) for increasing contrast $c_1 \in \{1.0,1.5,...,4.5\}$. We compare the case where $T = 0.5$ is fixed for all choices of $c_1$ and the case where $T$ barely fulfills $T > 0.25(1+c_1^{-1})$.}
    \label{fig:jumpCoefs:increasingcontrast}
  \end{figure}





  \begin{figure}[!htbp]
    \begin{center}
        \begin{tikzpicture}[scale=0.72]
            \begin{groupplot}[%
                group style={%
                group size=2 by 1,
                horizontal sep=1.5cm,
                vertical sep=2cm,
                },
            ymajorgrids=true,
            grid style=dashed,
            %ymin = 1e-3, ymax = 0.5e1,
            ]    
                \nextgroupplot[width=9cm,height=7cm,domain=0:1,xmode=linear,ymode=linear, xlabel={$x$}, ylabel={}, title={$u$ vs. $u_h$ ($c_1 = 2.5$)}, %cycle list name=paulcolors, 
                legend style={legend columns=5, draw=none,nodes={scale=.8}},legend to name=named, %yticklabels={,,},
                ]
  

                \addplot[black,line width=3pt,mark=None] table [x=x, y=y, col sep=comma] {../dataNew/plots/exact_plotSJ_contrast2.5_time0.25.csv};

                \addplot+[discard if not={L}{1},discard if not={contrast}{2.5},line width=1.1pt,mark=None] table [x=x, y=y, col sep=comma] {../dataNew/plots/simpleExact_ApproxPlot_T0.5.csv};
                %
                \addplot+[discard if not={L}{2},discard if not={contrast}{2.5},line width=1.1pt,mark=None] table [x=x, y=y, col sep=comma] {../dataNew/plots/simpleExact_ApproxPlot_T0.5.csv};
                %
                \addplot+[discard if not={L}{3},discard if not={contrast}{2.5},line width=1.1pt,mark=None] table [x=x, y=y, col sep=comma] {../dataNew/plots/simpleExact_ApproxPlot_T0.5.csv};
                %
                \addplot+[discard if not={L}{4},discard if not={contrast}{2.5},line width=1.1pt,mark=None] table [x=x, y=y, col sep=comma] {../dataNew/plots/simpleExact_ApproxPlot_T0.5.csv};
                

                \draw[dashed,gray,very thick] (0.5,-2) -- (0.5,2);
                \node [draw=none] at (axis description cs:0.45,0.23) {\color{gray} $\Gamma$};

                \legend{exact,$L=1$, $L=2$, $L=3$, $L=4$}
                \nextgroupplot[width=9cm,height=7cm,domain=0:1,xmode=linear,ymode=linear, xlabel={$x$}, ylabel={}, title={$u$ vs. $u_h$ ($c_1 = 5.5$)}, %cycle list name=paulcolors, 
                legend pos=north west, %yticklabels={,,},
                ymin = -2.25, ymax = 2.25,
                ]
               
                \addplot[black,line width=3pt,mark=None] table [x=x, y=y, col sep=comma] {../dataNew/plots/exact_plotSJ_contrast5.5_time0.25.csv};

                \addplot+[discard if not={L}{1},discard if not={contrast}{5.5},line width=1.1pt,mark=None] table [x=x, y=y, col sep=comma] {../dataNew/plots/simpleExact_ApproxPlot_T0.5.csv};
                %
                \addplot+[discard if not={L}{2},discard if not={contrast}{5.5},line width=1.1pt,mark=None] table [x=x, y=y, col sep=comma] {../dataNew/plots/simpleExact_ApproxPlot_T0.5.csv};
                %
                \addplot+[discard if not={L}{3},discard if not={contrast}{5.5},line width=1.1pt,mark=None] table [x=x, y=y, col sep=comma] {../dataNew/plots/simpleExact_ApproxPlot_T0.5.csv};
                %
                \addplot+[discard if not={L}{4},discard if not={contrast}{5.5},line width=1.1pt,mark=None] table [x=x, y=y, col sep=comma] {../dataNew/plots/simpleExact_ApproxPlot_T0.5.csv};
            

                \draw[dashed,gray,very thick] (0.5,-3) -- (0.5,3);
                \node [draw=none] at (axis description cs:0.45,0.23) {\color{gray} $\Gamma$};

                %Colors
                \definecolor{L1Color}{RGB}{55,126,184}
                \definecolor{L2Color}{RGB}{77,175,74}
                \definecolor{L3Color}{RGB}{152,78,163}
                \definecolor{L4Color}{RGB}{255,127,0}

                %L = 1
                \draw[color=L1Color,thick] (0.75,-2) -- (1,-2);
                \draw[color=L1Color,thick] (0.75,-2.05) -- (0.75,-1.95);
                \draw[color=L1Color,thick] (1.0,-2.05) -- (1.0,-1.95);

                %L = 2
                \draw[color=L2Color,thick] (0.75,-1.75) -- (1,-1.75);
                \draw[color=L2Color,thick] (0.75,-1.70) -- (0.75,-1.8);
                \draw[color=L2Color,thick] (0.875,-1.70) -- (0.875,-1.8);
                \draw[color=L2Color,thick] (1.0,-1.70) -- (1.0,-1.8);
                
                %L = 3
                \draw[color=L3Color,thick] (0.75,-1.5) -- (1,-1.5);
                \draw[color=L3Color,thick] (0.75,-1.45) -- (0.75,-1.55);
                \draw[color=L3Color,thick] (0.8125,-1.45) -- (0.8125,-1.55);
                \draw[color=L3Color,thick] (0.875,-1.45) -- (0.875,-1.55);
                \draw[color=L3Color,thick] (0.9375,-1.45) -- (0.9375,-1.55);
                \draw[color=L3Color,thick] (1.0,-1.45) -- (1.0,-1.55);
            
                %L = 4
                \draw[color=L4Color,thick] (0.75,-1.25) -- (1,-1.25);
                \draw[color=L4Color,thick] (0.75,-1.20) -- (0.75,-1.30);
                \draw[color=L4Color,thick] (0.78125,-1.20) -- (0.78125,-1.30);
                \draw[color=L4Color,thick] (0.8125,-1.20) -- (0.8125,-1.30);
                \draw[color=L4Color,thick] (0.84375,-1.20) -- (0.84375,-1.30);
                \draw[color=L4Color,thick] (0.875,-1.20) -- (0.875,-1.30);
                \draw[color=L4Color,thick] (0.90625,-1.20) -- (0.90625,-1.30);
                \draw[color=L4Color,thick] (0.9375,-1.20) -- (0.9375,-1.30);
                \draw[color=L4Color,thick] (0.96875,-1.20) -- (0.96875,-1.30);
                \draw[color=L4Color,thick] (1.0,-1.20) -- (1.0,-1.30);

                %\legend{exact, $L=1$, $L=2$, $L=3$, $L=4$}
                
            \end{groupplot}
        \end{tikzpicture}
        \pgfplotslegendfromname{named}
    \end{center}
    \caption{At $t = 0.25$, we compare the exact solution \eqref{eq:1D:exact:simple} and approximations with polynomial degree $k = 3$ on refinement levels $L \in \{1,2,3,4 \}$ for contrasts $c_1 = 2.5$ (left) and $c_1= 5.5$ (right). The scale in the right plot indicates the mesh size $h = 1/2^{L+1}$ for the different refinement levels.}
    \label{fig:jumpCoefs:quality}
  \end{figure}


\subsubsection{Multiple jumps}
 The exact solution \eqref{eq:1D:exact:simple} can be extended to the case where we have multiple jumps in the wave speed $c$. For two points $p_1,p_2 \in (0,1)$, we decompose $\Omega$ into three subdomains $\Omega_1 = (0,p_1)$, $\Omega_2 = (p_1,p_2)$, and $\Omega_3 = (p_2,1)$ s.t. $\bar{\Omega} = \cup_{i = 1,2,3} \overline{\Omega}_i$. Then, we consider the following ansatz for the exact solution: 
\begin{equation}\label{eq:1D:exact:simpleMult}
    u(x,t) := \begin{cases}
        \cos(w_1 c_1 t) \cos(w_1(x-p_1)), & x \in \Omega_1, \\
        \cos(w_2 c_2 t) \cos(w_2(x-p_1)), & x \in \Omega_2, \\
        \cos(w_3 c_3 t) \cos(w_3(x-p_2)), & x \in \Omega_3. 
    \end{cases}
\end{equation}
To ensure the continuity of \eqref{eq:1D:exact:simpleMult}, we choose $p_2$ in dependence of $p_1$ and $w_2$. For $n \in \mathbb{Z}$, we set
\begin{align*}
    p_2 \coloneqq \frac{2 \pi n + w_2 p_1}{w_2} \quad \Longrightarrow \quad \cos(w_2(p_2-p_1)) = 1.
\end{align*}


In the following, we fix $c_2 = 1.0$ and choose $w_1 = 3 \pi$, $w_2 = w_1 c_1/c_2$ as above. Furthermore, we set $c_3 = c_1$, $w_3 = w_1$. 
As a first example, we choose $p_1 = 0.4$ and $n = 3$ such that $p_2 \approx 0.667$ and consider the data domain $\omega = [0.0,0.3] \times [0.7,1.0]$. For contrast $c_1 \in \{ 7.5,11.5 \}$, \cref{fig:jumpCoefs:multipleJumps} shows the exact solution and the approximation with $k = 3$ for fixed time step $\Delta t = 1/32$. 

\begin{figure}[!htbp]
    \begin{center}
        \begin{tikzpicture}[scale=0.72]
            \begin{groupplot}[%
                group style={%
                group size=2 by 1,
                horizontal sep=1.5cm,
                vertical sep=2cm,
                },
            ymajorgrids=true,
            grid style=dashed,
            ymin = -0.6e0, ymax = 0.6e0,
            ]    
                \nextgroupplot[width=9cm,height=7cm,domain=0:1,xmode=linear,ymode=linear, xlabel={}, ylabel={}, title={$u$ vs. $u_h$ ($c_1= 7.5$)}, %cycle list name=paulcolors, 
                legend style={legend columns=3, draw=none,nodes={scale=.8}},legend to name=named, %yticklabels={,,},
                ]
                \draw[fill=gray!40,draw=none,opacity=0.5] (0.3,-2) -- (0.3,2) -- (0,2) -- (0,-2) -- cycle;
                \draw[fill=gray!40,draw=none,opacity=0.5] (0.7,-2) -- (0.7,2) -- (1.0,2) -- (1.0,-2) -- cycle;
   

                \addplot[black,line width=3pt,mark=None] table [x=x, y=y, col sep=comma] {../dataNew/plots/exact_plotMJ_contrast7.5_time0.25.csv};
                %
                \addplot+[discard if not={L}{3},discard if not={contrast}{7.5},line width=1.0pt,mark=None,color=orange] table [x=x, y=y, col sep=comma] {../dataNew/plots/MultipleJumpsExact_ApproxPlot_T0.5_L3_contrast7.5.csv};
                %
                \addplot+[discard if not={L}{4},discard if not={contrast}{7.5},line width=1.0pt,mark=None,color=blue!75!white] table [x=x, y=y, col sep=comma] {../dataNew/plots/MultipleJumpsExact_ApproxPlot_T0.5_L4_contrast7.5.csv};

                
                \draw[dashed,gray,very thick] (0.4,-2) -- (0.4,2);
                \draw[dashed,gray,very thick] (0.6666,-2) -- (0.6666,2);
   

                \node [draw=none] at (axis description cs:0.38,0.125) {\color{gray} $\Gamma_1$};
                \node [draw=none] at (axis description cs:0.7,0.125) {\color{gray} $\Gamma_2$};
                \node [draw=none] at (axis description cs:0.2,0.125) {\color{gray} $\omega$};
                \node [draw=none] at (axis description cs:0.8,0.125) {\color{gray} $\omega$};

                
                \legend{exact, $L = 3$, $L = 4$}
                \nextgroupplot[width=9cm,height=7cm,domain=0:1,xmode=linear,ymode=linear, xlabel={}, ylabel={}, title={$u$ vs. $u_h$ ($c_1 = 11.5$)}, %cycle list name=paulcolors, 
                legend pos=north west, %yticklabels={,,},
                ]
                \draw[fill=gray!40,draw=none,opacity=0.5] (0.3,-2) -- (0.3,2) -- (0,2) -- (0,-2) -- cycle;
                \draw[fill=gray!40,draw=none,opacity=0.5] (0.7,-2) -- (0.7,2) -- (1.0,2) -- (1.0,-2) -- cycle;

                \addplot[black,line width=3pt,mark=None] table [x=x, y=y, col sep=comma] {../dataNew/plots/exact_plotMJ_contrast11.5_time0.25.csv};
                %
                \addplot+[discard if not={L}{3},discard if not={contrast}{11.5},line width=1.0pt,mark=None,color=orange] table [x=x, y=y, col sep=comma] {../dataNew/plots/MultipleJumpsExact_ApproxPlot_T0.5_L3_contrast11.5.csv};
                %
                \addplot+[discard if not={L}{4},discard if not={contrast}{11.5},line width=1.0pt,mark=None,color=blue!75!white] table [x=x, y=y, col sep=comma] {../dataNew/plots/MultipleJumpsExact_ApproxPlot_T0.5_L4_contrast11.5.csv};
               

                

                \draw[dashed,gray,very thick] (0.4,-2) -- (0.4,2);
                \draw[dashed,gray,very thick] (0.63188,-2) -- (0.63188,2);
                \node [draw=none] at (axis description cs:0.38,0.125) {\color{gray} $\Gamma_1$};
                \node [draw=none] at (axis description cs:0.71,0.125) {\color{gray} $\Gamma_2$};
                \node [draw=none] at (axis description cs:0.2,0.125) {\color{gray} $\omega$};
                \node [draw=none] at (axis description cs:0.8,0.125) {\color{gray} $\omega$};

            
                

            \end{groupplot}
        \end{tikzpicture}
        \pgfplotslegendfromname{named}
    \end{center}
    \caption{Exact solution \eqref{eq:1D:exact:simpleMult} and approximated solution with $k = 3$ of \eqref{eq:1D:exact:simpleMult} at $t = 0.25$ with contrast $c_1 = 7.5$ (left) and $c_1 = 11.5$ (right). We compare the approximations on the refinement levels $L = 3$ and $L = 4$, where we set $\Delta t = 1/32$ for both cases.}
    \label{fig:jumpCoefs:multipleJumps}
  \end{figure}


 As a second example, we consider the case where the data is only given in one half of the interval, i.e. $\omega = [0,0.3]$. The GCC is fulfilled if 
\begin{equation}
    T > 2 \left[ (p_2 - p_1) + \frac{p_1 - 0.3 + 1 - p_2}{c_1} \right]. 
\end{equation}
\Cref{fig:jumpCoefs:multipleJumps:noGCC} shows once more that the approximate solution matches the exact solution well if the GCC is fulfilled, but fails to do so if the GCC is not fulfilled.



  \begin{figure}[!htbp]
    \begin{center}
        \begin{tikzpicture}[scale=0.72]
            \begin{groupplot}[%
                group style={%
                group size=2 by 1,
                horizontal sep=1.5cm,
                vertical sep=2cm,
                },
            ymajorgrids=true,
            grid style=dashed,
            %ymin = -1.1e0, ymax = 1.1e0,
            ]    
                \nextgroupplot[width=9cm,height=7cm,domain=0:1,xmode=linear,ymode=linear, xlabel={}, ylabel={}, title={$u$ vs $u_h$ ($c_1 = 2.5$)}, %cycle list name=paulcolors, 
                legend style={legend columns=3, draw=none,nodes={scale=.8}},legend to name=named,, %yticklabels={,,},
                ]
                \draw[fill=gray!40,draw=none,opacity=0.5] (0.3,-2) -- (0.3,2) -- (0,2) -- (0,-2) -- cycle;

                \addplot[black,line width=3.0pt,mark=None] table [x=x, y=y, col sep=comma] {../dataNew/plots/exact_plotMJ_contrast2.5_time0.5.csv};

                \addplot+[line width=1.0pt,mark=None,color=orange] table [x=x, y=y, col sep=comma] {../data/multipleJumps_approx_plot_data_WPFalse_contrast2.5_k3_uniform_noGCC.csv};

                \addplot+[line width=1.0pt,mark=None,color=blue!75!white] table [x=x, y=y, col sep=comma] {../data/multipleJumps_approx_plot_data_WPFalse_contrast2.5_k3_dataLeft_T0.5.csv};
                

                \draw[dashed,gray,very thick] (0.4,-2) -- (0.4,2);
                \draw[dashed,gray,very thick] (0.6666,-2) -- (0.6666,2);
                %\node [draw=none] at (axis description cs:0.45,0.23) {\color{gray} $\Gamma$};

                \node [draw=none] at (axis description cs:0.38,0.125) {\color{gray} $\Gamma_1$};
                \node [draw=none] at (axis description cs:0.71,0.125) {\color{gray} $\Gamma_2$};
                \node [draw=none] at (axis description cs:0.2,0.125) {\color{gray} $\omega$};
                \legend{exact,$T=1.0$,$T=0.5$}
                \nextgroupplot[width=9cm,height=7cm,domain=0:1,xmode=linear,ymode=linear, xlabel={}, ylabel={}, title={$u$ vs $u_h$ ($c_1 = 7.5$)}, %cycle list name=paulcolors, 
                legend pos=north west, %yticklabels={,,},
                ]
                \draw[fill=gray!40,draw=none,opacity=0.5] (0.3,-2) -- (0.3,2) -- (0,2) -- (0,-2) -- cycle;

                \addplot[black,line width=3.0pt,mark=None] table [x=x, y=y, col sep=comma] {../dataNew/plots/exact_plotMJ_contrast7.5_time0.5.csv};



                \addplot+[line width=3.0pt,mark=None,color=blue!75!white] table [x=x, y=y, col sep=comma] {../dataNew/plots/MultipleJumpsExact_ApproxPlot_CompareT0.5_contrast7.5Test.csv};

     
                
                \addplot+[line width=1.0pt,mark=None,color=orange] table [x=x, y=y, col sep=comma] {../dataNew/plots/MultipleJumpsExact_ApproxPlot_CompareT1.0_contrast7.5.csv};
    

                \draw[dashed,gray,very thick] (0.4,-2) -- (0.4,2);
                \draw[dashed,gray,very thick] (0.6666,-2) -- (0.6666,2);
                \node [draw=none] at (axis description cs:0.38,0.125) {\color{gray} $\Gamma_1$};
                \node [draw=none] at (axis description cs:0.71,0.125) {\color{gray} $\Gamma_2$};
                \node [draw=none] at (axis description cs:0.2,0.125) {\color{gray} $\omega$};

            
            \end{groupplot}
        \end{tikzpicture}
         \pgfplotslegendfromname{named}
    \end{center}
    \caption{Exact solution \eqref{eq:1D:exact:simpleMult} and approximated solution with $k = 3$ at $t = 0.5$ with $c_1 = c_3 = 2.5$, $n = 1$ (left) and $c_1 = c_3 = 7.5$, $n = 3$ (right) and $c_2 = 1.0$. We compare the approximations on refinement level $L = 4$ with $T = 1.0$ and $T = 0.5$ while keeping the time step fixed at $\Delta t = 1/32$.}
    \label{fig:jumpCoefs:multipleJumps:noGCC}
  \end{figure}


\subsubsection{2nd one dimensional example}

Let us consider a final one-dimensional example to illustrate the high resolution 
required to simulate wave propagation in heterogeneous media. 
We consider the same geometrical setup as in \cref{sec:numex:1D:simple} with a more intricate reference solution considered in \cite{BDE22,MEMM19}. With $u_0 (x) := \frac{1}{100} \exp(-(20(x-\frac{1}{5}))^2)$, we consider the exact solution 
\begin{equation}\label{eq:1D:complicatedSol}
    u(x,t) = \begin{cases}
      \sum_{k \ge 0} \left( \frac{c_2-c_1}{c_2+c_1}\right)^k \left( u_0 (k + x - c_1 t) - u_0(k-x-c_1 t)\right), &x \in \Omega_1, \\
      \left( \frac{2c_1}{c_2+c_1}\right) \sum_{k \ge 0} \left( \frac{c_2 - c_1}{c_2 + c_1}\right)^k u_0 \left( \frac{c_1}{c_2} \left( x - \frac{1}{2}\right) +k + \frac{1}{2} - c_1 t \right), & x \in \Omega_2. 
    \end{cases}
\end{equation}
This solution describes a wave traveling towards the interface $\Gamma$. Due to the jump in the wave speed, a part of the wave is reflected from the interface into the domain $\Omega_1$, while the other part passes through the interface into $\Omega_2$ and oscillates stronger. 

In \cref{fig:1D:complicatedSol} we consider the approximation of this function by the $L^2$-interpolation. The plot on the right shows that for $c_1 = 2.5$ the velocity is not captured well yet even for $k=3$ on the finest refinement level we considered. In terms of convergence rates, we merely observe $\mathcal{O}(h^2)$ for $k=2$. In view of the previous experiments it is clear that this poses a serious obstruction to obtaining accurate solutions to the corresponding unique continuation problem.


  \begin{figure}[!htbp]
    \begin{center}
        \begin{tikzpicture}[scale=0.72]
            \begin{groupplot}[%
                group style={%
                group size=2 by 1,
                horizontal sep=1.5cm,
                vertical sep=2cm,
                },
            ymajorgrids=true,
            grid style=dashed,
            %ymin = 1e-3, ymax = 0.5e1,
            ]    
                \nextgroupplot[width=9cm,height=7cm,domain=0:4,xmode=linear,ymode=log, xlabel={}, ylabel={}, title={$L^2$-best approximation ($k = 2$)}, %cycle list name=paulcolors, 
                legend pos=south west, 
                %yticklabels={,,}, 
                xlabel={$L$}, 
		%ylabel={$\Vert (u-L_{\Delta t} \ul_1) \Vert_{L^\infty(0,T;L^2(\Omega))}$},
                ]
                %\addplot+[discard if not={order}{2},discard if not={contrast}{1.0},line width=1.1pt] table [x=L, y=bestapprox-L-infty-L2-error-u, col sep=comma] {../data/weakDisc_noRestrict_1D_jumpingCoefs_k2.csv};

                \addplot+[discard if not={order}{2},discard if not={contrast}{1.0},line width=1.1pt] table [x=L, y=bestapprox-L-infty-L2-error-u, col sep=comma] {../dataNew/weakDisc_evenhigherDt_noRestrict_1D_jumpingCoefs_k2.csv};

                \addplot+[discard if not={order}{2},discard if not={contrast}{1.5},line width=1.1pt] table [x=L, y=bestapprox-L-infty-L2-error-u, col sep=comma] {../dataNew/weakDisc_evenhigherDt_noRestrict_1D_jumpingCoefs_k2.csv};

                \addplot+[discard if not={order}{2},discard if not={contrast}{2.0},line width=1.1pt] table [x=L, y=bestapprox-L-infty-L2-error-u, col sep=comma] {../dataNew/weakDisc_evenhigherDt_noRestrict_1D_jumpingCoefs_k2.csv};

                %\addplot+[discard if not={order}{2},discard if not={contrast}{1.5},line width=1.1pt] table [x=L, y=bestapprox-L-infty-L2-error-u, col sep=comma] {../data/weakDisc_noRestrict_1D_jumpingCoefs_k2.csv};
                %\addplot+[discard if not={order}{2},discard if not={contrast}{2.0},line width=1.1pt] table [x=L, y=bestapprox-L-infty-L2-error-u, col sep=comma] {../data/weakDisc_noRestrict_1D_jumpingCoefs_k2.csv};
                %\addplot+[discard if not={order}{2},discard if not={contrast}{2.0},line width=1.1pt,dashed] table [x=L, y=L-infty-L2-error-u, col sep=comma] {../data/weakDisc_noRestrict_1D_jumpingCoefs_k2.csv};
             
             
                \addplot[gray, dashed, domain=1:4] {0.0005*(1/2^(2))^(x-0.9)};
                \addplot[gray, dotted, domain=1:4] {0.0008*(1/2^(3))^(x-0.9)};
                \addplot[gray, dashed, domain=1:4] {0.0025*(1/2^(3/2))^(x-0.9)};
                \addplot[gray, dotted, domain=1:4] {0.0008*(1/2^(5/2))^(x-0.9)};

                \node [draw=none] at (axis description cs:0.50,0.40) {\color{gray}\footnotesize $\!\!\mathcal{O}(h^{2})$};
                \node [draw=none] at (axis description cs:0.90,0.60) {\color{gray}\footnotesize $\!\!\mathcal{O}(h^{3/2})$};


                \legend{$c_1 = 1.0$,$c_1 = 1.5$,$c_1 = 2.0$}
                \nextgroupplot[width=9cm,height=7cm,domain=0:4,xmode=linear,ymode=linear, xlabel={}, ylabel={}, title={$\dt u$ vs. $\dt u_h$ ($k = 3$)}, %cycle list name=paulcolors, 
                legend pos=north west, %yticklabels={,,},
                ]
                \draw[fill=gray!40,draw=none,opacity=0.5] (0.25,-2) -- (0.25,2) -- (0,2) -- (0,-2) -- cycle;
                \draw[fill=gray!40,draw=none,opacity=0.5] (0.75,-2) -- (0.75,2) -- (1.0,2) -- (1.0,-2) -- cycle;


                \addplot+[discard if not={contrast}{2.0},line width=2.5pt,mark=None,color=black] table [x=x, y=y_dt, col sep=comma] {../data/exact_plot_data_1D_complexSol.csv};
                %\addplot+[discard if not={L}{3},line width=1.1pt,mark=none] table [x=x, y=y_dt, col sep=comma] {../data/weakDisc_higherDt_noRestrict_approx_plot_data_WPFalse_contrast2.0_k3.csv};
                %\addplot+[discard if not={L}{3},line width=1.1pt,mark=none,dotted] table [x=x, y=y_dt, col sep=comma] {../data/weakDisc_noRestrict_approx_plot_data_WPFalse_contrast2.0_k3.csv};
                %\addplot+[discard if not={L}{4},line width=1.1pt,mark=none] table [x=x, y=y_dt, col sep=comma] {../data/weakDisc_higherDt_noRestrict_approx_plot_data_WPFalse_contrast2.0_k3.csv};


                \addplot+[discard if not={L}{3},line width=1.1pt,mark=none] table [x=x, y=y_dt, col sep=comma] {../dataNew/plots/weakDisc_evenhigherDt_noRestrict_approx_plot_data_WPFalse_contrast2.0_k3.csv};

                \addplot+[discard if not={L}{4},line width=1.1pt,mark=none] table [x=x, y=y_dt, col sep=comma] {../dataNew/plots/weakDisc_evenhigherDt_noRestrict_approx_plot_data_WPFalse_contrast2.0_k3.csv};
                
                \draw[dashed,gray,very thick] (0.5,-2) -- (0.5,2);
                \node [draw=none] at (axis description cs:0.53,0.125) {\color{gray} $\Gamma$};
                \node [draw=none] at (axis description cs:0.175,0.125) {\color{gray} $\omega$};
                \node [draw=none] at (axis description cs:0.83,0.125) {\color{gray} $\omega$};
             
                %\addplot[gray, dashed, domain=1:4] {0.01*(1/2^(2))^(x-0.9)};
                %\addplot[gray, dashed, domain=1:4] {0.025*(1/2^(3/2))^(x-0.9)};

                %L = 2
                %\draw[color=black,thick] (0.75,-0.7) -- (1,-0.7);
                %\draw[color=black,thick] (0.75,-0.68) -- (0.75,-0.72);
                %\draw[color=black,thick] (0.875,-0.68) -- (0.875,-0.72);
                %\draw[color=black,thick] (1.0,-0.68) -- (1.0,-0.72);
                
                %L = 3
                %\draw[color=black,thick] (0.75,-0.6) -- (1,-0.6);
                %\draw[color=black,thick] (0.75,-0.58) -- (0.75,-0.62);
                %\draw[color=black,thick] (0.8125,-0.58) -- (0.8125,-0.62);
                %\draw[color=black,thick] (0.875,-0.58) -- (0.875,-0.62);
                %\draw[color=black,thick] (0.9375,-0.58) -- (0.9375,-0.62);
                %\draw[color=black,thick] (1.0,-0.58) -- (1.0,-0.62);
                %\node[draw=none] at (0.67,-0.6) {\color{black} \footnotesize $L = 3$};

                \legend{exact,$L=3$, $L=4$}
            \end{groupplot}
        \end{tikzpicture}
        \caption{The $L^2$-bestapproximation error of \eqref{eq:1D:complicatedSol} with $k = 2$ for $c_2 = 1$ and $c_1 \in \{1.0,1.5,2.0\}$ (left) and the approximation of the time derivative of \eqref{eq:1D:complicatedSol} with $k = 3$ for $c_1 = 2.5$ for $L = 4,5$, $h = 1/2^{L+1} = \Delta t$ (right). \textbf{what is happening here now? before it was $\mathcal{O}(h^2)$ and $\mathcal{O}(h^{3/2})$..., right plot looks fine, also $c_1 = 2.0$ for the right plot...}}
        \label{fig:1D:complicatedSol}
    \end{center}
  \end{figure}



\subsection{Example in two space dimensions}
To conclude the numerical examples, we consider another example in two space dimensions. We partition the unit-square $\Omega \coloneqq [0,1]^2 \subset \mathbb{R}^2$ into subdomains $\Omega_1 = \{ (x,y) \in \Omega: x < 0.5 \}$ and $\Omega_2 = \{ (x,y) \in \Omega: x > 0.5 \}$, and 
extend the exact solution defined in \eqref{eq:1D:exact:simple} to the two-dimensional case by setting $u(x,y;t) \coloneqq u(x;t)$. 

We define the data domain $\omega \coloneqq \Omega \setminus [0.25,0.75]^2$ and note that $\omega$ fulfills the GCC. To ensure that the data domain and the interface $\Gamma$ are meshed exactly, we consider a quadrilateral mesh with $2^{L+1}$, $L \in \{1,2,3,4\}$, elements in each direction. We consider the final time $T = 0.75$  and set the time step size to $\Delta t = (1/2)^{L+1}$. As before, we fix the wavespeed $c_2 = 1$ and consider different values for $c_1$ to control the contrast. 

The results of the experiment are presented in \cref{fig:2DjumpCoefs}. 
For $c_1 = 2.5$, $\Vert \dt(u-L_{\Delta t} \ul_1) \Vert_{L^2(0,T;L^2(\Omega))}$ and $\Vert u-L_{\Delta t} \ul_1 \Vert_{L^\infty(0,T;L^2(\Omega))}$ converge quasi-optimally with order $\mathcal{O}(h^k)$ for polynomial degrees $k \in \{2,3\}$. Similar as already observed in the one-dimensional case the right plot in \cref{fig:2DjumpCoefs} shows that these errors are more sensitive to increasing $c_1$ than the corresponding best approximation errors. 
In particular, the ratio for $\Vert u-L_{\Delta t} \ul_1 \Vert_{L^\infty(0,T;L^2(\Omega))}$ with $k=2$ appears to grow as $\mathcal{O}(c_1^{3/2})$ which renders it very demanding to maintain a reasonable accuracy as the contrast increases. 

  \begin{figure}[!htbp]
    \begin{center}
        \begin{tikzpicture}[scale=0.76]
            \begin{groupplot}[%
                group style={%
                group size=2 by 1,
                horizontal sep=1.25cm,
                vertical sep=2cm,
                },
            ymajorgrids=true,
            grid style=dashed,
            %ymin = 1e-3, ymax = 0.5e1,
            ]    
                \nextgroupplot[width=9cm,height=7cm,domain=0:4,xmode=linear,ymode=log, xlabel={refinement level $L$}, ylabel={}, title={Contrast $c_1 = 2.5$}, %cycle list name=paulcolors, 
                %legend pos=north east, %yticklabels={,,},
                xtick={1,2,3,4},
                legend style={legend columns=3, draw=none,nodes={scale=.8}},
                legend to name=named,
                ]
                %
                \addplot+[discard if not={order}{2},discard if not={contrast}{2.5},line width=1.1pt,color=orange] table [x=L, y=L2-L2-error-u_t, col sep=comma] {../dataNew/2D_errors_simpleRefsolFull.csv};
                %
                \addplot+[discard if not={order}{3},discard if not={contrast}{2.5},line width=1.1pt,color=teal] table [x=L, y=L2-L2-error-u_t, col sep=comma] {../dataNew/2D_errors_simpleRefsolFull.csv};
                %
                \addplot+[discard if not={order}{2},discard if not={contrast}{2.5},line width=1.1pt,dashed,color=orange] table [x=L, y=L-infty-L2-error-u, col sep=comma] {../dataNew/2D_errors_simpleRefsolFull.csv};
                %
                \addplot+[discard if not={order}{3},discard if not={contrast}{2.5},line width=1.1pt,dashed,color=teal] table [x=L, y=L-infty-L2-error-u, col sep=comma] {../dataNew/2D_errors_simpleRefsolFull.csv};

                \addplot[gray, dashed, domain=1:4] {9*(1/2^(2))^(x-0.9)};
                \addplot[gray, dashed, domain=1:4] {0.65*(1/2^(2))^(x-0.9)};
                \addplot[gray, dashed, domain=1:4] {0.15*(1/2^(3))^(x-0.9)};
                \addplot[gray, dashed, domain=1:4] {4.5*(1/2^(3))^(x-0.9)};
                \node [draw=none] at (axis description cs:0.50,0.52) {\color{gray}\footnotesize $\!\!\mathcal{O}(h^{2})$};
                \node [draw=none] at (axis description cs:0.45,0.35) {\color{gray}\footnotesize $\!\!\mathcal{O}(h^{3})$};
                \node [draw=none] at (axis description cs:0.60,0.82) {\color{gray}\footnotesize $\!\!\mathcal{O}(h^{2})$};
                \node [draw=none] at (axis description cs:0.6,0.65) {\color{gray}\footnotesize $\!\!\mathcal{O}(h^{3})$};
    

                \legend{$k=2$,$k=3$}
                \nextgroupplot[width=9cm,height=7cm,domain=0:4,xmode=linear,ymode=log, xlabel={contrast $c_1$}, ylabel={}, title={}, %cycle list name=paulcolors, 
                legend pos=north west, %yticklabels={,,},
                xtick={1.0,1.5,2.0,2.5},
                ]
                

               

                \addplot+[discard if not={order}{2},discard if not={L}{4},line width=1.1pt,color=orange] table [x=contrast, y expr= abs(\thisrow{L2-L2-error-u_t}/\thisrow{bestapprox-L2-L2-error-u_t}), col sep=comma]  {../dataNew/2D_errors_simpleRefsolFull.csv};
                %
                \addplot+[discard if not={order}{3},discard if not={L}{4},line width=1.1pt,color=teal] table [x=contrast, y expr= abs(\thisrow{L2-L2-error-u_t}/\thisrow{bestapprox-L2-L2-error-u_t}), col sep=comma] {../dataNew/2D_errors_simpleRefsolFull.csv};

               
                %L-infty-L2-u error
                \addplot+[discard if not={order}{2},discard if not={L}{4},line width=1.1pt,dashed,color=orange] table [x=contrast, y expr= abs(\thisrow{L-infty-L2-error-u}/\thisrow{bestapprox-L-infty-L2-error-u}), col sep=comma] {../dataNew/2D_errors_simpleRefsolFull.csv};
                \addplot+[discard if not={order}{3},discard if not={L}{4},line width=1.1pt,dashed,color=teal] table [x=contrast, y expr= abs(\thisrow{L-infty-L2-error-u}/\thisrow{bestapprox-L-infty-L2-error-u}), col sep=comma] {../dataNew/2D_errors_simpleRefsolFull.csv};
               
    
                \addplot[gray, dashed, domain=1:2.5] {2*(1/2^(-3/2))^(x-0.9)};
                \addplot[gray, dashed, domain=1:2.5] {1.35*(1/2^(-1/2))^(x-0.9)};
            
                \node [draw=none] at (axis description cs:0.70,0.7) {\color{gray}\footnotesize $\!\!\mathcal{O}(c_1^{3/2})$};
                \node [draw=none] at (axis description cs:0.7,0.45) {\color{gray}\footnotesize $\!\!\mathcal{O}(c_1^{1/2})$};
    
                %\legend{$k=2$,$k=3$}
               
            \end{groupplot}
        \end{tikzpicture}
        \pgfplotslegendfromname{named}
    \end{center}
    \caption{Left: Convergence of $\Vert \dt(u-L_{\Delta t} \ul_1) \Vert_{L^2(0,T;L^2(\Omega))}$ (solid) and $\Vert u-L_{\Delta t} \ul_1 \Vert_{L^\infty(0,T;L^2(\Omega))}$ (dashed) for $c_1 = 2.5$. Right: Ratio between these errors and the $L^2$-bestapproximation error for increasing $c_1$.}
    \label{fig:2DjumpCoefs}
  \end{figure}


\section{Conclusion}
In this article we have investigated a numerical method to solve the unique continuation problem 
for the wave equation in the presence of a jump discontinuity in the wavespeed. The main conclusion from our numerical experiments is that obtaining accurate numerical solutions in the presence of material interfaces involving large contrasts is computationally demanding. Even though the problem appears to be Lipschitz stable if the geometric control condition is fulfilled, the resolution required to capture the complex behavior of the solution is a limiting factor in numerical simulations. Note here that unique continuation problems are globally coupled in space and time which poses an obstacle for their efficient numerical solution. 


\bibliography{references}

\end{document}
