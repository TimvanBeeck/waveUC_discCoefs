%\documentclass[10pt,reqno]{amsart}
\documentclass[tikz,border=2mm]{standalone}

%\setlength{\topmargin}{0cm}
%\setlength{\textheight}{21cm}
%\setlength{\oddsidemargin}{0in}
%\setlength{\evensidemargin}{0in}
%\setlength{\textwidth}{6.5in}
%\setlength{\parindent}{.25in}

%\pagestyle{plain}
\usepackage{stmaryrd}
\usepackage{amssymb, amsmath, amsthm}
\usepackage{mathtools}
\usepackage{xcolor}
\usepackage{hyperref}
%\usepackage{showkeys}
\usepackage{enumerate}
\usepackage{subfig}
\usepackage{enumitem} 
\usepackage{todonotes}
\usepackage{booktabs}
\usepackage[]{algorithm2e}
\usepackage[capitalize]{cleveref}
\usepackage{placeins}
\crefname{equation}{}{}
\usepackage{verbatim}
\usepackage{bbm}

%\usepackage{graphicx}
%% expansion of width
%\textwidth=15.7cm
%\textheight=22.5cm
%\parskip=3pt
%\parindent=8mm
%\oddsidemargin=2mm
%\evensidemargin=0mm
%\topmargin=-0.5cm
%\marginparwidth=1cm


%% definition of theorem-type environments
\newtheorem{thm}{Theorem}[section]
\newtheorem{lem}[thm]{Lemma}
\newtheorem{cor}[thm]{Corollary}
\newtheorem{prop}[thm]{Proposition}
\newtheorem{defn}[thm]{Definition}
\newtheorem{rmk}{Remark}
\newtheorem{exa}{Example}
\newtheorem{assum}{Assumption}
\numberwithin{equation}{section}
\newcommand{\bel}{\begin{equation} \label}
\newcommand{\ee}{\end{equation}}
\def\beq{\begin{equation}}
\def\eeq{\end{equation}}
\newcommand{\jump}[1]{\llbracket#1\rrbracket}
\newcommand{\bea}{\begin{eqnarray}}
\newcommand{\eea}{\end{eqnarray}}
\newcommand{\beas}{\begin{eqnarray*}}
\newcommand{\eeas}{\end{eqnarray*}}
\newcommand{\pd}{\partial}
\newcommand{\mdiv}[1]{\ensuremath{\mathrm{div} \left( #1 \right)}}
\newcommand{\dd}{\mbox{d}}

\newcommand{\ep}{\varepsilon}
\newcommand{\la}{\lambda}
\newcommand{\va}{\varphi}
\newcommand{\ppp}{\partial}
\newcommand{\chch}{\chi_{\eta}}
\newcommand{\walpha}{\widetilde{\alpha}}
\newcommand{\wbeta}{\widetilde{\beta}}

\newcommand{\re}{\mathfrak R}

\newcommand{\im}{\mathfrak I}
\newcommand{\pdif}[2]{\frac{\partial #1}{\partial #2}}
\newcommand{\ppdif}[2]{\frac{\partial^2 #1}{{\partial #2}^2}}
\newcommand{\R}{\mathbb{R}}
\newcommand{\C}{\mathbb{C}} 
\newcommand{\N}{\mathbb{N}} 
\newcommand{\ooo}{\overline}
\newcommand{\uu}{\mathbf{u}}
\renewcommand{\v}{\mathbf{v}}
\newcommand{\y}{\mathbf{y}}
\newcommand{\RR}{\mathbf{R}}
\newcommand{\Y}{\mathbf{Y}}
\newcommand{\w}{\mathbf{w}}
\newcommand{\z}{\mathbf{z}}
\newcommand{\G}{\mathbf{G}}
\newcommand{\cB}{\mathcal{B}}
\newcommand{\cD}{\mathcal{D}}
\newcommand{\cL}{\mathcal{L}}
\newcommand{\cO}{\mathcal{O}}
\newcommand{\Hin}{\mathcal{H}_{\mathrm{in},T_0}}
\newcommand{\Gi}{S_{\mathrm{in}}}
\newcommand{\Go}{S_{\mathrm{out}}}
\newcommand{\dom}{\mathrm{Dom}}

\newcommand{\sdd}{\mathcal{E}}


\renewcommand{\baselinestretch}{1.5}
%
\renewcommand{\div}{\mathrm{div}\,}  %div
\newcommand{\grad}{\mathrm{grad}\,}  %grad
\newcommand{\rot}{\mathrm{rot}\,}  %rot

\newcommand{\supp}{\mathrm{supp}\,}  %supp
%\newcommand{\span}{\mathrm{span}\,} %span

\allowdisplaybreaks
%%  item
\renewcommand{\theenumi}{\arabic{enumi}}
\renewcommand{\labelenumi}{(\theenumi)}
\renewcommand{\theenumii}{\alph{enumii}}
\renewcommand{\labelenumii}{(\theenumii)}
\def\epsilon{\varepsilon}
%\def\phi {\varphi}
\def \la {{\lambda}}
\def \a {{\alpha}}
\def\t{\theta}
\def\fh{\frac{1}{h}}


\DeclareMathOperator{\dis}{dist}


\newcommand{\wop}{\square_c}

\newcommand{\tnorm}[1]{\vert\hspace{-0.3mm}\Vert#1\Vert\hspace{-0.3mm}\vert}

\providecommand{\abs}[1]{\left\lvert#1\right\rvert}
% pour les normes
\providecommand{\norm}[1]{\left\lVert#1\right\rVert}

\renewcommand{\leq}{\leqslant}
\renewcommand{\geq}{\geqslant}
\providecommand{\abs}[1]{\left\lvert#1\right\rvert}
% pour les normes
\providecommand{\norm}[1]{\left\lVert#1\right\rVert}
\def\thefootnote{{}}



\newcommand{\HOX}[1]{\marginpar{\footnotesize #1}}

\newcommand{\gammaGLS}{\gamma_{\text{GLS}}}
\newcommand{\gammaCIP}{\gamma}

\newcommand{\dT}{\mathrm{d}t}
\newcommand{\dX}{\mathrm{d}x}
\newcommand{\dS}{\mathrm{d}S}

\newcommand{\STdom}{Q}

\newcommand{\STdata}{\omega_T}
\newcommand{\STdataDisc}{\underline{\omega}_T}
\newcommand{\SemiDiscSpace}{\mathcal{W}_h}
\newcommand{\FullyDiscSpace}{W}

%\newcommand{\FullyDiscrSpaceDisc}[2]{ W_{ {#1},{#2}}^{ \text{dc} } }
%\newcommand{\FullyDiscrSpaceCont}[2]{ W_{ {#1},{#2}}^{ \text{c}  } }
%\newcommand{\ProdFullyDiscrSpaceDisc}[2]{  \mathcal{W}_{ {#1},{#2} }^{ \text{dc} } }
%\newcommand{\ProdFullyDiscrSpaceCont}[1]{  \mathcal{W}_{ {#1} }^{ \text{c} } }

\newcommand{\SemiDiscrSpace}[1]{ W^{ {#1}}_{h} }
\newcommand{\ProdSemiDiscrSpace}[1]{ \mathcal{W}^{ {#1} }_{h} }
\newcommand{\FullyDiscrSpace}[2]{ W^{ {#1},{#2}}_{h, \Delta t  } }
\newcommand{\FullyDiscrSpaceHat}[2]{ \hat{W}^{ {#1},{#2}}_{h, \Delta t  } }
\newcommand{\ProdFullyDiscrSpace}[2]{ \mathcal{W}^{ {#1},{#2}}_{h, \Delta t  } }

\DeclarePairedDelimiterX{\inp}[2]{(}{)}{#1, #2}
\newcommand{\tangular}[1]{ \llbracket\kern-0.5ex|#1|\kern-0.5ex\rrbracket} 
%\newcommand{\jump}[1]{\llbracket#1\rrbracket}
\newcommand{\avg}[1]{ \{\!\!\{#1\}\!\!\}}

\newboolean{includeextras}
\ifdefined\withextras
\setboolean{includeextras}{true}
\else
\setboolean{includeextras}{false}
\fi

\newcommand{\putextra}[1]{\ifthenelse{\boolean{includeextras}}{#1}{}}

\newcommand{\Uh}{\underline{\mathbf{U}}_h}
\newcommand{\Vh}{\underline{\mathbf{V}}_h}
\newcommand{\Yh}{\underline{\mathbf{Y}}_h}
\newcommand{\Zh}{\underline{\mathbf{Z}}_h}
\newcommand{\Wh}{\underline{\mathbf{W}}_h}

\newcommand{\ul}{\underline{u}}
\newcommand{\yl}{\underline{y}}
\newcommand{\zl}{\underline{z}}
\newcommand{\wl}{\underline{w}}

\newcommand{\Sud}{S^{\uparrow \downarrow}_{\Delta t}}


\newcommand{\dt}{\partial_t}
\newcommand{\dtt}{\partial_t^2}



%%% TvB: Added for plots, some might be redundant ##

%------------------------- Added for Plots ----
% Tikz package
\usepackage{tikz}
\usepackage{pgfplots,pgfplotstable}
\usepgfplotslibrary{colorbrewer,groupplots}
\usepackage{caption}
\pgfplotsset{compat=1.18}

%\usetikzlibrary{external}
%\tikzexternalize[prefix=figures/]

%find colors at https://colorbrewer2.org/#type=qualitative&scheme=Set1&n=5
\pgfplotsset{
% initialize Set1-5:
cycle list/Set1-5,
% combine it with ’mark list*’:
cycle multiindex* list={
mark list*\nextlist
Set1-5\nextlist
},
}

\pgfplotsset{
    discard if not/.style 2 args={
        x filter/.append code={
            \edef\tempa{\thisrow{#1}}
            \edef\tempb{#2}
            \ifx\tempa\tempb
            \else
                \def\pgfmathresult{inf}
            \fi
        }
    }
}


\begin{document}
        \begin{tikzpicture}[scale=0.72]
            \begin{groupplot}[%
                group style={%
                group size=2 by 1,
                horizontal sep=1.5cm,
                vertical sep=2cm,
                },
            ymajorgrids=true,
            grid style=dashed,
            %ymin = 1e-3, ymax = 0.5e1,
            ]    
                \nextgroupplot[width=9cm,height=7cm,domain=0:1,xmode=linear,ymode=linear, xlabel={$x$}, ylabel={}, title={$u$ vs. $u_h$ ($c_1 = 2.5$)}, %cycle list name=paulcolors, 
                legend style={legend columns=5, draw=none,nodes={scale=.8}},legend to name=named, %yticklabels={,,},
                ]
  

                \addplot[black,line width=3pt,mark=None] table [x=x, y=y, col sep=comma] {../data/plots/exact_plotSJ_contrast2.5_time0.25.csv};

                \addplot+[discard if not={L}{1},discard if not={contrast}{2.5},line width=1.1pt,mark=None] table [x=x, y=y, col sep=comma] {../data/plots/simpleExact_ApproxPlot_T0.5.csv};
                %
                \addplot+[discard if not={L}{2},discard if not={contrast}{2.5},line width=1.1pt,mark=None] table [x=x, y=y, col sep=comma] {../data/plots/simpleExact_ApproxPlot_T0.5.csv};
                %
                \addplot+[discard if not={L}{3},discard if not={contrast}{2.5},line width=1.1pt,mark=None] table [x=x, y=y, col sep=comma] {../data/plots/simpleExact_ApproxPlot_T0.5.csv};
                %
                \addplot+[discard if not={L}{4},discard if not={contrast}{2.5},line width=1.1pt,mark=None] table [x=x, y=y, col sep=comma] {../data/plots/simpleExact_ApproxPlot_T0.5.csv};
                

                \draw[dashed,gray,very thick] (0.5,-2) -- (0.5,2);
                \node [draw=none] at (axis description cs:0.45,0.23) {\color{gray} $\Gamma$};

                \legend{exact,$L=1$, $L=2$, $L=3$, $L=4$}
                \nextgroupplot[width=9cm,height=7cm,domain=0:1,xmode=linear,ymode=linear, xlabel={$x$}, ylabel={}, title={$u$ vs. $u_h$ ($c_1 = 5.5$)}, %cycle list name=paulcolors, 
                legend pos=north west, %yticklabels={,,},
                ymin = -2.25, ymax = 2.25,
                ]
               
                \addplot[black,line width=3pt,mark=None] table [x=x, y=y, col sep=comma] {../data/plots/exact_plotSJ_contrast5.5_time0.25.csv};

                \addplot+[discard if not={L}{1},discard if not={contrast}{5.5},line width=1.1pt,mark=None] table [x=x, y=y, col sep=comma] {../data/plots/simpleExact_ApproxPlot_T0.5.csv};
                %
                \addplot+[discard if not={L}{2},discard if not={contrast}{5.5},line width=1.1pt,mark=None] table [x=x, y=y, col sep=comma] {../data/plots/simpleExact_ApproxPlot_T0.5.csv};
                %
                \addplot+[discard if not={L}{3},discard if not={contrast}{5.5},line width=1.1pt,mark=None] table [x=x, y=y, col sep=comma] {../data/plots/simpleExact_ApproxPlot_T0.5.csv};
                %
                \addplot+[discard if not={L}{4},discard if not={contrast}{5.5},line width=1.1pt,mark=None] table [x=x, y=y, col sep=comma] {../data/plots/simpleExact_ApproxPlot_T0.5.csv};
            

                \draw[dashed,gray,very thick] (0.5,-3) -- (0.5,3);
                \node [draw=none] at (axis description cs:0.45,0.23) {\color{gray} $\Gamma$};

                %Colors
                \definecolor{L1Color}{RGB}{55,126,184}
                \definecolor{L2Color}{RGB}{77,175,74}
                \definecolor{L3Color}{RGB}{152,78,163}
                \definecolor{L4Color}{RGB}{255,127,0}

                %L = 1
                \draw[color=L1Color,thick] (0.75,-2) -- (1,-2);
                \draw[color=L1Color,thick] (0.75,-2.05) -- (0.75,-1.95);
                \draw[color=L1Color,thick] (1.0,-2.05) -- (1.0,-1.95);

                %L = 2
                \draw[color=L2Color,thick] (0.75,-1.75) -- (1,-1.75);
                \draw[color=L2Color,thick] (0.75,-1.70) -- (0.75,-1.8);
                \draw[color=L2Color,thick] (0.875,-1.70) -- (0.875,-1.8);
                \draw[color=L2Color,thick] (1.0,-1.70) -- (1.0,-1.8);
                
                %L = 3
                \draw[color=L3Color,thick] (0.75,-1.5) -- (1,-1.5);
                \draw[color=L3Color,thick] (0.75,-1.45) -- (0.75,-1.55);
                \draw[color=L3Color,thick] (0.8125,-1.45) -- (0.8125,-1.55);
                \draw[color=L3Color,thick] (0.875,-1.45) -- (0.875,-1.55);
                \draw[color=L3Color,thick] (0.9375,-1.45) -- (0.9375,-1.55);
                \draw[color=L3Color,thick] (1.0,-1.45) -- (1.0,-1.55);
            
                %L = 4
                \draw[color=L4Color,thick] (0.75,-1.25) -- (1,-1.25);
                \draw[color=L4Color,thick] (0.75,-1.20) -- (0.75,-1.30);
                \draw[color=L4Color,thick] (0.78125,-1.20) -- (0.78125,-1.30);
                \draw[color=L4Color,thick] (0.8125,-1.20) -- (0.8125,-1.30);
                \draw[color=L4Color,thick] (0.84375,-1.20) -- (0.84375,-1.30);
                \draw[color=L4Color,thick] (0.875,-1.20) -- (0.875,-1.30);
                \draw[color=L4Color,thick] (0.90625,-1.20) -- (0.90625,-1.30);
                \draw[color=L4Color,thick] (0.9375,-1.20) -- (0.9375,-1.30);
                \draw[color=L4Color,thick] (0.96875,-1.20) -- (0.96875,-1.30);
                \draw[color=L4Color,thick] (1.0,-1.20) -- (1.0,-1.30);

                %\legend{exact, $L=1$, $L=2$, $L=3$, $L=4$}
                
            \end{groupplot}
    \node at ($(current bounding box.south)+(0,-0.5cm)$) {\pgfplotslegendfromname{named}};
\end{tikzpicture}
\end{document}
